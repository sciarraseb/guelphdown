% This is the Reed College LaTeX thesis template. Most of the work
% for the document class was done by Sam Noble (SN), as well as this
% template. Later comments etc. by Ben Salzberg (BTS). Additional
% restructuring and APA support by Jess Youngberg (JY).
% Your comments and suggestions are more than welcome; please email
% them to cus@reed.edu
%
% See https://www.reed.edu/cis/help/LaTeX/index.html for help. There are a
% great bunch of help pages there, with notes on
% getting started, bibtex, etc. Go there and read it if you're not
% already familiar with LaTeX.
%
% Any line that starts with a percent symbol is a comment.
% They won't show up in the document, and are useful for notes
% to yourself and explaining commands.
% Commenting also removes a line from the document;
% very handy for troubleshooting problems. -BTS

%%
%% Preamble
\documentclass[
12pt, % The default document font size, options: 10pt, 11pt, 12pt
twoside,
english]{guelphthesis}
%----------------------------------------------------------------------------------------
% PACKAGES
%----------------------------------------------------------------------------------------

\usepackage{tocloft} %needed for table of contents, list of figures, list of tables, list of appendices
\usepackage{graphicx,latexsym}
\usepackage{amsmath}
\usepackage{amssymb,amsthm}

\usepackage{longtable,booktabs,setspace}
\usepackage{lmodern}
\usepackage{float}
\usepackage{etoolbox}
\floatplacement{figure}{H}
% Thanks, @Xyv
\usepackage{calc}
% End of CII addition
\usepackage{rotating}
\usepackage{tocbibind} %includes list of figures, list of tables, and table of contents in table of contents
\usepackage{indentfirst} %needed so that first paragraph after each section titles has indent
\usepackage{lineno} %allows option for line numbering
\usepackage{draftwatermark} %for draft watermark
\SetWatermarkText{} %ensures draft is not printed when draft:false
%\usepackage[backend=biber]{biblatex}


% Syntax highlighting #22

% To pass between YAML and LaTeX the dollar signs are added by CII
\title{Is Timing Everything? Measurement Timing and the Ability to Accurately Model Longitudinal Data}
\author{Sebastian L.V. Sciarra}
\year{2022}
\date{October, 2022}
\advisor{David Stanley}
\institution{University of Guelph}
\degree{Doctorate of Philosophy}



\department{Psychology}


% From {rticles}
\newlength{\cslhangindent}
\setlength{\cslhangindent}{1cm} %indentation of hanging lines
% for Pandoc 2.8 to 2.10.1
\newenvironment{cslreferences}%
  {}%
  {\par}

% For Pandoc 2.11+
% As noted by @mirh [2] is needed instead of [3] for 2.12
\newenvironment{CSLReferences}[2] % #1 hanging-ident, #2 entry spacing
 {% don't indent paragraphs
  \setlength{\parindent}{0pt}
  % turn on hanging indent if param 1 is 1
  \ifodd #1 \everypar{\setlength{\hangindent}{\cslhangindent}}\ignorespaces\fi
  % set entry spacing
  \ifnum #2 > 0
  \setlength{\parskip}{\linespacing{2}}
  \fi
 }%
 {}


\urlstyle{rm}

%----------------------------------------------------------------------------------------
% CUSTOM COMMANDS
%----------------------------------------------------------------------------------------
%numbers lines before equations
%taken from https://tex.stackexchange.com/questions/43648/why-doesnt-lineno-number-a-paragraph-when-it-is-followed-by-an-align-equation
\newcommand*\patchAmsMathEnvironmentForLineno[1]{%
  \expandafter\let\csname old#1\expandafter\endcsname\csname #1\endcsname
  \expandafter\let\csname oldend#1\expandafter\endcsname\csname end#1\endcsname
  \renewenvironment{#1}%
     {\linenomath\csname old#1\endcsname}%
     {\csname oldend#1\endcsname\endlinenomath}}%
\newcommand*\patchBothAmsMathEnvironmentsForLineno[1]{%
  \patchAmsMathEnvironmentForLineno{#1}%
  \patchAmsMathEnvironmentForLineno{#1*}}%
\AtBeginDocument{%
\patchBothAmsMathEnvironmentsForLineno{equation}%
\patchBothAmsMathEnvironmentsForLineno{align}%
\patchBothAmsMathEnvironmentsForLineno{flalign}%
\patchBothAmsMathEnvironmentsForLineno{alignat}%
\patchBothAmsMathEnvironmentsForLineno{gather}%
\patchBothAmsMathEnvironmentsForLineno{multline}%
}


%nest all the \frontmatter functions in \oldfrontmatter, which allows us to redefine \frontmatter as everything it was with one modification to the
%draft watermark
\let\oldfrontmatter\frontmatter
%set page numbering to bottom center for \frontmatter
\fancypagestyle{frontmatter}{%
 \fancyhf{}% clear all header and footer fields
  \renewcommand{\headrulewidth}{0pt}
  \fancyhead[R]{\roman{page}}% Roman page number in footer centre

  }

\renewcommand{\frontmatter}{
  \oldfrontmatter
     \SetWatermarkLightness{0.8} %shading of draft watermark
  \SetWatermarkText{DRAFT}
  
   %set page number font to Arial if ArialFont: false in YAML header
  
   \pagestyle{frontmatter} % add this to center page numbers
}

%set page numbering to bottom center for \mainmatter
\fancypagestyle{mainmatter}{%
 \fancyhf{}% clear all header and footer fields
  \renewcommand{\headrulewidth}{0pt}
  \fancyfoot[C]{\arabic{page}}% Roman page number in footer centre

 \hypersetup{pdfpagemode={UseOutlines},
    bookmarksopen=true,
    hypertexnames=true,
    colorlinks = true,
    citecolor = true,
    linkcolor = blue,
    urlcolor= blue,
    anchorcolor = blue,
    pdfstartview={FitV},
    breaklinks=true,
    hyperindex = true, backref=page}

  

}

%nest all the \mainmatter functions in \oldmainmatter, which allows us to redefine \mainmatter as everything it was with one modification to the
%page numbering format
\newcommand{\setMainMatterLinespacing}{
 \setstretch{2} %default line spacing

  %change line spacing if specified in YAML header
        \setstretch{2}
  }

\let\oldmainmatter\mainmatter
\renewcommand{\mainmatter}{
  \oldmainmatter

  %change line spacing if specified in YAML header
  \setMainMatterLinespacing

      \linenumbers
  
  \pagestyle{mainmatter} % add this to center page numbers

}

%code below is important for linespacing to remain unaffected when kableExtra::landscape() is used andthe margin is specifically defined. Otherwise,
%linespacing for entire document goes to singlespacing for the text that follows the table.
\let\oldRestoreGeometry\restoregeometry
\renewcommand{\restoregeometry}{
  \oldRestoreGeometry

  %change line spacing if specified in YAML header
  \setMainMatterLinespacing
}

%change footnote and page number font to arial if desired

%----------------------------------------------------------------------------------------
%	TABLE OF CONTENTS, LIST OF FIGURES, & LIST OF TABLES
%----------------------------------------------------------------------------------------
%TABLE OF CONTENTS
\setlength{\cftbeforetoctitleskip}{0cm} %remove vertical space above table of contents

%two lines below ensure centered title for toc
%needed so that table of contents entry is not indented
\renewcommand{\contentsname}{Table of Contents} %change title for toc
\renewcommand{\cfttoctitlefont}{\hfill\fontsize{14}{14}\selectfont\bfseries\MakeUppercase}
\renewcommand{\cftaftertoctitle}{\hfill\hfill} %sometimes another \hfill is needed; depends on some setting in abovce code

%fonts for all entry level titles
\renewcommand\cftchapfont{\mdseries} %eliminate bolded chapter titles in toc
\renewcommand\cftsecfont{\mdseries} %eliminate bolded chapter titles in toc
\renewcommand\cftsubsecfont{\mdseries} %eliminate bolded chapter titles in toc
\renewcommand\cftsubsubsecfont{\mdseries} %eliminate bolded chapter titles in toc
\renewcommand\cftparafont{\mdseries} %eliminate bolded chapter titles in toc
\renewcommand\cftsubparafont{\mdseries} %eliminate bolded chapter titles in toc

%fonts for all entry level page numbers
\renewcommand{\cftchappagefont}{\mdseries} %remove bolding of page numbers for chapter headers in toc
\renewcommand\cftsecpagefont{\mdseries} %eliminate bolded chapter titles in toc
\renewcommand\cftsubsecpagefont{\mdseries} %eliminate bolded chapter titles in toc
\renewcommand\cftsubsubsecpagefont{\mdseries} %eliminate bolded chapter titles in toc
\renewcommand\cftparapagefont{\mdseries} %eliminate bolded chapter titles in toc
\renewcommand\cftsubparapagefont{\mdseries} %eliminate bolded chapter titles in toc

\renewcommand{\cftchapleader}{\cftdotfill{0.1}} %remove chapter bolding + modif dot spacing
\renewcommand{\cftdotsep}{0.1} %make dots in toc closer together

%spacing between toc items (should be all equal)
\setlength{\cftbeforechapskip}{0cm} %removes spacing before each chapter element
\renewcommand{\cftchapafterpnum}{\vskip6pt}
\renewcommand{\cftsecafterpnum}{\vskip6pt}
\renewcommand{\cftsubsecafterpnum}{\vskip6pt}
\renewcommand{\cftsubsubsecafterpnum}{\vskip6pt}
\renewcommand{\cftparaafterpnum}{\vskip6pt}
\renewcommand{\cftsubparaafterpnum}{\vskip6pt}

%remove header that appears in table of contents after first page
\renewcommand{\cftmarktoc}{}

%commands need to be redefined so that leading dots go all the way to the page numbers for all header levels (chap, sec, subsec, subsubsec, para, subpara
%%%general framework for commands below: cftXfillnum sets the format for the leading dots (\cftchapleader) and the page number (\cftchappagefont) such that leading dots proceed all the way to the page number with no spaces between dots and page number (\nobreak) at which wpoint paragraph mode ends (\par) and vertical spacing (defined  above) after item entry is inserted
%chapter (level 0)
\renewcommand{\cftchapfillnum}[1]{%
  {\cftchapleader}\nobreak
  {\cftchappagefont #1}\par\cftchapafterpnum
}

%sec (level 1)
\renewcommand{\cftsecfillnum}[1]{%
  {\cftsecleader}\nobreak
  {\cftsecpagefont #1}\par\cftsecafterpnum
}

%subsec (level 2)
\renewcommand{\cftsubsecfillnum}[1]{%
  {\cftsubsecleader}\nobreak
  {\cftsubsecpagefont #1}\par\cftsubsecafterpnum
}

%subsubsec (level 3)
\renewcommand{\cftsubsubsecfillnum}[1]{%
  {\cftsubsubsecleader}\nobreak
  {\cftsubsubsecpagefont #1}\par\cftsubsubsecafterpnum
}

%para (level 4)
\renewcommand{\cftparafillnum}[1]{%
  {\cftparaleader}\nobreak
  {\cftparapagefont #1}\par\cftparaafterpnum
}

%subpara (level 5)
\renewcommand{\cftsubparafillnum}[1]{%
  {\cftsubparaleader}\nobreak
  {\cftsubparapagefont #1}\par\cftsubparaafterpnum
}

%LIST OF TABLES
\renewcommand{\cfttabfont}{\mdseries} %set font for entries in lot
\renewcommand{\cfttabpagefont}{\mdseries} %set front for page numbers

\setlength{\cftbeforelottitleskip}{0cm} %remove vertical space above table of contents
\setlength{\cftafterlottitleskip}{0.5cm} %space between title for list of tables and list entries
%two lines below ensure centered title for toc
%needed so that table of contents entry is not indented
\renewcommand{\cftlottitlefont}{\hfill\fontsize{14}{14}\selectfont\bfseries\MakeUppercase}
\renewcommand{\cftafterlottitle}{\hfill} %sometimes another \hfill is needed; depends on some setting in abovce code

%commands need to be redefined so that leading dots go all the way to the page numbers for tables
%%%general framework for command below: cftfigfillnum sets the format for the leading dots (\cftfigleader) and the page number (\cftfigpagefont) such that leading dots proceed all the way to the page number with no spaces between dots and page number (\nobreak) at which point paragraph mode ends (\par) and vertical spacing (defined  below) after item entry is inserted
\setlength{\cftbeforetabskip}{0cm} %removes spacing before each chapter element
\renewcommand{\cfttabafterpnum}{\vskip6pt}

\renewcommand{\cfttabfillnum}[1]{%
  {\cfttableader}\nobreak
  {\cfttabpagefont #1}\par\cfttabafterpnum
}

%remove header that appears in list of tables after first page
\renewcommand{\cftmarklot}{}

%LIST OF FIGURES
\renewcommand{\cftfigfont}{\mdseries} %set font for entries in lot
\renewcommand{\cftfigpagefont}{\mdseries} %set front for page numbers

\setlength{\cftbeforeloftitleskip}{0cm} %remove vertical space above table of contents
\setlength{\cftafterloftitleskip}{0.5cm} %space between title for list of figures and list entries

%two lines below ensure centered title for toc
%needed so that table of contents entry is not indented
\renewcommand{\cftloftitlefont}{\hfill\fontsize{14}{14}\selectfont\bfseries\MakeUppercase}
\renewcommand{\cftafterloftitle}{\hfill} %sometimes another \hfill is needed; depends on some setting in abovce code

%commands need to be redefined so that leading dots go all the way to the page numbers for figures
%%%general framework for command below: cftfigfillnum sets the format for the leading dots (\cftfigleader) and the page number (\cftfigpagefont) such that leading dots proceed all the way to the page number with no spaces between dots and page number (\nobreak) at which wpoint paragraph mode ends (\par) and vertical spacing (defined  below) after item entry is inserted
\setlength{\cftbeforefigskip}{0cm} %removes spacing before each chapter element
\renewcommand{\cftfigafterpnum}{\vskip6pt}

\renewcommand{\cftfigfillnum}[1]{%
  {\cftfigleader}\nobreak
  {\cftfigpagefont #1}\par\cftfigafterpnum
}

%remove header that appears in list of figures after first page
\renewcommand{\cftmarklof}{}

%----------------------------------------------------------------------------------------
% LIST OF APPENDICES
%----------------------------------------------------------------------------------------
\newcommand{\listappname}{List of Appendices}
\newlistof[chapter]{app}{loa}{\listappname} %creates a new appendix counter that will be reset at the start of each \chapter

\setcounter{loadepth}{5} %loa will  go to depth of level 5
\setlength{\cftbeforeloatitleskip}{0cm} %remove vertical space above loa
\setlength{\cftafterloatitleskip}{0.5cm} %space between title for loa and list entries
\renewcommand{\cftmarkloa}{} %remove header titles

%two lines below ensure centered title for toc
%needed so that table of contents entry is not indented
\renewcommand{\cftloatitlefont}{\hfill\fontsize{14}{14}\selectfont\bfseries\MakeUppercase}
\renewcommand{\cftafterloatitle}{\hfill\hfill} %sometimes another \hfill is needed; depends on some setting in above code


%APPENDIX (level 0)
\renewcommand{\theapp}{\Alph{app}} %sets alphabetic counter for appendix
\renewcommand{\cftappfont}{\mdseries} %set font for level 0 entry in loa
\renewcommand{\cftapppagefont}{\mdseries} %set front for page numbers

\renewcommand{\cftapppresnum}{Appendix\space}
\renewcommand{\cftappaftersnum}{:\space}
\settowidth{\cftappnumwidth}{\cftapppresnum\theapp\cftappaftersnum\space}

\setlength{\cftbeforeappskip}{0cm} %removes vertical spacing before each chapter element
\renewcommand{\cftappafterpnum}{\vskip6pt}

%updates appendix counter, modifies chapter title such so that it is Appendix _letter_: #1
\newcommand{\app}[1]{%
  \refstepcounter{app}\pdfbookmark[-1]{\cftapppresnum\theapp\cftappaftersnum#1}{#1\theapp}%
  \chapter*{\fontsize{16}{16}\selectfont\bfseries\cftapppresnum\theapp\cftappaftersnum #1} %formats entry in document
  \addcontentsline{loa}{app}{{\cftapppresnum\theapp\cftappaftersnum}#1}%
  \par
}

% figure and table counting in appendix
\usepackage{chngcntr}


%leading dots for appendix (end immediately before page number)
\renewcommand{\cftappfillnum}[1]{%
 {\cftappleader}\nobreak{\cftapppagefont #1}\par\cftappafterpnum
}

%SECAPPENDIX (level 1; format A.1 : title)
\newlistentry[app]{secapp}{loa}{1}
\renewcommand{\thesecapp}{\theapp.\arabic{secapp}}
\renewcommand{\cftsecappfont}{\mdseries} %set font for level 1 entry in loa
\renewcommand{\cftsecapppagefont}{\mdseries} %set front for page numbers

\renewcommand{\cftsecapppresnum}{} %remove word 'Appendix'
\renewcommand{\cftsecappaftersnum}{\hspace{0.5cm}}  %replicate toc format for sub-level-0 headers \thesubappendix (i.e., A.1   title )

\setlength{\cftbeforesecappskip}{0cm} %removes vertical spacing before each chapter element
\renewcommand{\cftsecappafterpnum}{\vskip6pt}
\setlength{\cftsecappindent}{1.55em} %indentation in loa
\settowidth{\cftsecappnumwidth}{\cftsecapppresnum\thesecapp\cftsecappaftersnum\hspace{0.3cm}}

%updates appendix counter, modifies chapter title such so that it is Appendix _letter_: #1
\newcommand{\secapp}[1]{%
  \refstepcounter{secapp}\pdfbookmark[0]{#1}{#1\thesubapp}%
  \section*{\thesecapp\hspace{0.3cm} #1} %spacing between section number and title in text
  \addcontentsline{loa}{secapp}{{\thesecapp\cftsecappaftersnum}#1}%
  \par
}

%leading dots for appendix (end immediately before page number)
\renewcommand{\cftsecappfillnum}[1]{%
 {\cftsecappleader}\nobreak{\cftsecapppagefont #1}\par\cftsecappafterpnum
}


%SUBAPPENDIX (level 2; format A.1.1 : title)
\newlistentry[app]{subapp}{loa}{1}
\renewcommand{\thesubapp}{\thesecapp.\arabic{subapp}}
\renewcommand{\cftsubappfont}{\mdseries} %set font for level 2 entry in loa
\renewcommand{\cftsubapppagefont}{\mdseries} %set front for page numbers

\renewcommand{\cftsubapppresnum}{} %remove word 'Appendix'
\renewcommand{\cftsubappaftersnum}{\hspace{0.5cm}}  %replicate toc format for sub-level-0 headers \thesubappendix (i.e., A.1   title )

\setlength{\cftbeforesubappskip}{0cm} %removes vertical spacing before each chapter element
\renewcommand{\cftsubappafterpnum}{\vskip6pt}
\setlength{\cftsubappindent}{3.10em} %indentation in loa
%\renewcommand{\cftsubappnumwidth}{1.47cm}
\settowidth{\cftsubappnumwidth}{\thesubapp\cftsubappaftersnum\hspace{0.3cm}}

%updates appendix counter, modifies chapter title such so that it is Appendix _letter_: #1
\newcommand{\subapp}[1]{%
  \refstepcounter{subapp}\pdfbookmark[1]{#1}{#1\thesubapp}%
  \subsection*{\thesubapp\hspace{0.3cm} #1}%
  \addcontentsline{loa}{subapp}{{\thesubapp\cftsubappaftersnum}#1}%
  \par
}

%leading dots for appendix (end immediately before page number)
\renewcommand{\cftsubappfillnum}[1]{%
 {\cftsubappleader}\nobreak{\cftsubapppagefont #1}\par\cftsubappafterpnum
}


% SUBSUBAPPENDIX (level 3; format A.1.1.1  title)
\newlistentry[app]{subsubapp}{loa}{1}
\renewcommand{\thesubsubapp}{\thesubapp.\arabic{subsubapp}}
\renewcommand{\cftsubsubappfont}{\mdseries} %set font for level 3 entry in loa
\renewcommand{\cftsubsubapppagefont}{\mdseries} %set front for page numbers


\renewcommand{\cftsubsubapppresnum}{} %remove word 'Appendix'
\renewcommand{\cftsubsubappaftersnum}{\hspace{0.5cm}}  %space after subsubapp title

\setlength{\cftbeforesubsubappskip}{0cm} %removes vertical spacing before each chapter element
\renewcommand{\cftsubsubappafterpnum}{\vskip6pt}
\setlength{\cftsubsubappindent}{4.65em} %indentation in loa (1.55 *2)
\settowidth{\cftsubsubappnumwidth}{\thesubsubapp\cftsubsubappaftersnum\hspace{0.3cm}}

%updates appendix counter, modifies chapter title such so that it is Appendix _letter_: #1
\newcommand{\subsubapp}[1]{%
  \refstepcounter{subsubapp}\pdfbookmark[2]{#1}{#1\thesubsubapp}%
  \subsubsection*{\thesubsubapp\hspace{0.3cm} #1}%
  \addcontentsline{loa}{subsubapp}{{\thesubsubapp\cftsubsubappaftersnum}#1}%
  \par
}

%leading dots for appendix (end immediately before page number)
\renewcommand{\cftsubsubappfillnum}[1]{%
 {\cftsubsubappleader}\nobreak{\cftsubsubapppagefont #1}\par\cftsubsubappafterpnum
}

% PARA (level 4; format A.1.1.1.1  title)
\newlistentry[app]{paraapp}{loa}{1}
\renewcommand{\theparaapp}{\thesubsubapp.\arabic{paraapp}}
\renewcommand{\cftparaappfont}{\mdseries} %set font for level 4 entry in loa
\renewcommand{\cftparaapppagefont}{\mdseries} %set front for page numbers

\renewcommand{\cftparaapppresnum}{} %remove word 'Appendix'
\renewcommand{\cftparaappaftersnum}{\hspace{0.5cm}}  %space after paraapp title

\setlength{\cftbeforeparaappskip}{0cm} %removes vertical spacing before each chapter element
\renewcommand{\cftparaappafterpnum}{\vskip6pt}
\setlength{\cftparaappindent}{6.2em} %indentation in loa (1.55 *2)
\settowidth{\cftparaappnumwidth}{\theparaapp\cftparaappaftersnum\hspace{0.3cm}}

%updates appendix counter, modifies chapter title such so that it is Appendix _letter_: #1
\newcommand{\paraapp}[1]{%
  \refstepcounter{paraapp}\pdfbookmark[3]{#1}{#1\theparaapp}%
  \paragraph*{\theparaapp\hspace{0.3cm} #1}%
  \addcontentsline{loa}{paraapp}{{\theparaapp\cftparaappaftersnum}#1}%
  \par
}

%leading dots for appendix (end immediately before page number)
\renewcommand{\cftparaappfillnum}[1]{%
 {\cftparaappleader}\nobreak{\cftparaapppagefont #1}\par\cftparaappafterpnum
}

% SUBPARA (level 5; format A.1.1.1.1  title)
\newlistentry[app]{subparaapp}{loa}{1}
\renewcommand{\thesubparaapp}{\theparaapp.\arabic{subparaapp}}
\renewcommand{\cftsubparaappfont}{\mdseries} %set font for level 5 entry in loa
\renewcommand{\cftsubparaapppagefont}{\mdseries} %set front for page numbers

\renewcommand{\cftsubparaapppresnum}{} %remove word 'Appendix'
\renewcommand{\cftsubparaappaftersnum}{\hspace{0.5cm}}  %space after subparaapp title

\setlength{\cftbeforesubparaappskip}{0cm} %removes vertical spacing before each chapter element
\renewcommand{\cftsubparaappafterpnum}{\vskip6pt}
\setlength{\cftsubparaappindent}{7.75em} %indentation in loa (1.55 *2)
\settowidth{\cftsubparaappnumwidth}{\thesubparaapp\cftsubparaappaftersnum\hspace{0.3cm}}

%updates appendix counter, modifies chapter title such so that it is Appendix _letter_: #1
\newcommand{\subparaapp}[1]{%
  \refstepcounter{subparaapp}\pdfbookmark[4]{#1}{#1\thesubparaapp}%
  \paragraph*{\thesubparaapp\hspace{0.3cm} #1} %paragraph is used because subparagraph has weird numbering problem
  \addcontentsline{loa}{subparaapp}{{\thesubparaapp\cftsubparaappaftersnum}#1}%
  \par
}

%SUBSUBPARA (level 6; format A.1.1.1.1.1  title)
\newlistentry[app]{subsubparaapp}{loa}{1}
\renewcommand{\thesubsubparaapp}{\thesubparaapp.\arabic{subsubparaapp}}

\renewcommand{\cftsubsubparaapppresnum}{} %remove word 'Appendix'
\renewcommand{\cftsubsubparaappaftersnum}{\hspace{0.5cm}}  %space after subparaapp title

\setlength{\cftbeforesubsubparaappskip}{0cm} %removes vertical spacing before each chapter element
\renewcommand{\cftsubsubparaappafterpnum}{\vskip6pt}
\setlength{\cftsubsubparaappindent}{9.3em} %indentation in loa (1.55 *2)
\settowidth{\cftsubsubparaappnumwidth}{\thesubsubparaapp\cftsubsubparaappaftersnum\hspace{0.3cm}}

%updates appendix counter, modifies chapter title such so that it is Appendix _letter_: #1
\newcommand{\subsubparaapp}[1]{%
  \refstepcounter{subsubparaapp}\pdfbookmark[5]{#1}{#1\thesubsubparaapp}%
  \subparagraph*{\thesubsubparaapp\hspace{0.3cm} #1} %paragraph is used because subparagraph has weird numbering problem
  \addcontentsline{loa}{subsubparaapp}{{\thesubsubparaapp\cftsubsubparaappaftersnum}#1}%
  \par
}

%leading dots for appendix (end immediately before page number)
\renewcommand{\cftsubsubparaappfillnum}[1]{%
 {\cftsubsubparaappleader}\nobreak{\cftsubsubparaapppagefont #1}\par\cftsubsubparaappafterpnum
}




%load additional latex packages needed within document
	\usepackage{booktabs}
\usepackage{longtable}
\usepackage{array}
\usepackage{multirow}
\usepackage{wrapfig}
\usepackage{float}
\usepackage{colortbl}
\usepackage{pdflscape}
\usepackage{tabu}
\usepackage{threeparttable}
\usepackage{threeparttablex}
\usepackage[normalem]{ulem}
\usepackage{makecell}
\usepackage{xcolor}



 % Required for customising links and the PDF
\hypersetup{pdfpagemode={UseOutlines},
bookmarksopen=true, %allows bookmarks in pdf
hypertexnames=true, %enables counting when referencing to sections
colorlinks = true, % Set to true to enable coloring links, a nice option, false to turn them off
citecolor = blue, % The color of citations
linkcolor = blue, % The color of references to document elements (sections, figures, etc)
urlcolor= blue,
anchorcolor = blue, % The color of hyperlinks (URLs)
pdfstartview={FitV},
breaklinks=true, backref=page
}




% BEGIN DOCUMENT
\begin{document}
\frontmatter %pages will be numbered with roman numerals

  \maketitle

\setcounter{page}{2} %ensures abstract page number starts at roman numberal ii

\thispagestyle{empty} %removes page number only for abstract page
  \begin{abstract}{2}{The preface pretty much says it all. This is additional content. The preface pretty much says it all. This is additional content. The preface pretty much says it all. This is additional content. The preface pretty much says it all. This is additional content. The preface pretty much says it all. This is additional content.}  %[linespacing][abstract][

  \end{abstract}

% notice how yaml variables are indexed with dollar signs and then passed into second argument of preambleItem environments
  \begin{preambleItem}{2}{Dedication}{You can have a dedication here if you wish. You can have a dedication here if you wish.You can have a dedication here if you wish.You can have a dedication here if you wish.You can have a dedication here if you wish.You can have a dedication here if you wish.You can have a dedication here if you wish.}
  \end{preambleItem}
   \begin{preambleItem}{2}{Acknowledgements}{I want to thank a few people.You can have a dedication here if you wish. You can have a dedication here if you wish.You can have a dedication here if you wish.You can have a dedication here if you wish.You can have a dedication here if you wish.You can have a dedication here if you wish.You can have a dedication here if you wish.}
  \end{preambleItem}


%move page numbers to top right for list of tables, figures, and tables
\fancypagestyle{plain}{%
  \fancyhf{}% clear all header and footer fields
  \renewcommand{\headrulewidth}{0pt}
  \fancyhead[R]{\thepage}

   }

%table of contents
  \hypersetup{linkcolor = black, pdfborder= 0 0 0} %remove red borders around toc items
  \setcounter{secnumdepth}{5}
  \setcounter{tocdepth}{5}
  \tableofcontents
  \newpage

%list of tables
  \listoftables
  \newpage

%list of figures
  \listoffigures
  \newpage

%list of appendices
  \phantomsection
  \addcontentsline{toc}{chapter}{\listappname}
  \listofapp

  \newpage

\mainmatter % here the regular arabic numbering starts

\hypertarget{introduction}{%
\chapter{Introduction}\label{introduction}}
\begin{quote}
    ``Neither the behavior of human beings nor the activities of organizations can be defined without reference to time, and temporal aspects are critical for understanding them" (Navarro et al., 2015, p. 136).
\end{quote}
The topic of time has received considerable attention in organizational psychology over the past 20 years. Examples of well-received articles published around the beginning of the 21\textsuperscript{st} century discuss how investigating time is important for
understanding patterns of change and boundary conditions of theory
(\protect\hyperlink{ref-zaheer1999}{Zaheer et al., 1999}), how longitudinal research is necessary for disentangling
different types of causality (\protect\hyperlink{ref-mitchell2001}{Mitchell \& James, 2001}), and explicate a pattern
of organizational change (or institutionalization; \protect\hyperlink{ref-lawrence2001}{Lawrence et al., 2001}).
Since then, articles have emphasized the need to address time in
specific areas such as performance (\protect\hyperlink{ref-dalal2014}{Dalal et al., 2014}; \protect\hyperlink{ref-fisher2008}{Fisher, 2008}), teams (\protect\hyperlink{ref-roe2012}{Roe et al., 2012}), and goal setting (\protect\hyperlink{ref-fried2004}{Fried \& Slowik, 2004}) and, more generally, throughout organizational research (\protect\hyperlink{ref-aguinis2021}{Aguinis \& Bakker, 2021}; \protect\hyperlink{ref-george2000}{George \& Jones, 2000}; \protect\hyperlink{ref-kunisch2017}{Kunisch et al., 2017}; \protect\hyperlink{ref-navarro2015}{Navarro et al., 2015}; \protect\hyperlink{ref-ployhart2010}{Ployhart \& Vandenberg, 2010}; \protect\hyperlink{ref-roe2008}{Roe, 2008}; \protect\hyperlink{ref-shipp2015}{Shipp \& Cole, 2015}; \protect\hyperlink{ref-sonnentag2012}{Sonnentag, 2012}; \protect\hyperlink{ref-vantilborgh2018}{Vantilborgh et al., 2018}).

The importance of time has also been recognized in organizational theory. In defining a theoretical contribution, Whetten (\protect\hyperlink{ref-whetten1989}{1989}) discussed that time must be discussed in regard to setting boundary conditions (i.e., under what circumstances does the theory apply) and in specifying relations between variables over time (\protect\hyperlink{ref-george2000}{George \& Jones, 2000}; \protect\hyperlink{ref-mitchell2001}{Mitchell \& James, 2001}). Even if a considerable number of organizational theories do not adhere to the definition of Whetten (\protect\hyperlink{ref-whetten1989}{1989}), theoretical models in organizational psychology consist of path diagrams that delineate the causal underpinnings of a process. Given that temporal precedence is a necessary condition for establishing causality (\protect\hyperlink{ref-mill2011}{Mill, 2011}), time has a role, whether implicitly or explicitly, in organizational theory.





Despite the considerable attention given towards investigating processes over time and its ubiquity in organizational theory, the prevalence of longitudinal research has historically remained low. One study examined the prevalence of longitudinal research from 1970--2006 across five organizational psychology journals and found that 4\% of articles used longitudinal designs (Roe, 2014). Another survey of two applied psychology journals in 2005 found that approximately 10\% (10 of 105 studies) of studies used longitudinal designs (\protect\hyperlink{ref-roe2008}{Roe, 2008}). Similarly, two surveys of studies employing longitudinal designs with mediation analysis found that, across five journals, only about 10\% (7 of 72 studies) did so in 2005 (\protect\hyperlink{ref-maxwell2007}{Maxwell \& Cole, 2007}) and approximately 16\% (15 of 92 studies) did so in 2006 (\protect\hyperlink{ref-mitchell2013}{Mitchell \& Maxwell, 2013}).\footnote{Note that the definition of a longitudinal design in Maxwell \& Cole (\protect\hyperlink{ref-maxwell2007}{2007}) and Mitchell \& Maxwell (\protect\hyperlink{ref-mitchell2013}{2013}) required that measurements be taken over at least three time points so that measurements of the predictor, mediator, and outcome variables were separated over time.} Thus, the prevalence of longitudinal research has remained low.

In the seven sections that follow, I will explain why longitudinal research is necessary and the factors that must be considered when conducting such research. In the first section, I will explain why conducting longitudinal research is essential for understanding the dynamics of psychological processes. In the second section, I will overview patterns of change that are likely to emerge over time. In the third and fourth sections, I will, respectively, discuss some methods for modelling nonlinear change and the frameworks in which they can be used. In the fifth section, I will overview design and analytical issues involved in designing longitudinal studies. In the sixth section, I will explain how design and analytical issues encountered in conducting longitudinal research can be investigated. Finally, in the seventh section, I will provide a systematic review of the research that has investigated design and analytical issues involved in conducting longitudinal research. A summary of the three simulation experiments that I conducted in my dissertation will then be provided.

\hypertarget{the-need-to-conduct-longitudinal-research}{%
\section{The Need to Conduct Longitudinal Research}\label{the-need-to-conduct-longitudinal-research}}

Longitudinal research provides substantial advantages over cross-sectional research. Unfortunately, researchers commonly discuss the results of cross-sectional analyses as if they have been obtained with a longitudinal design. However, cross-sectional and longitudinal analyses often produce different results. One example of the assumption that cross-sectional findings are equivalent to longitudinal findings comes from the large number of studies employing mediation analysis. Given that mediation is used to understand chains of causality in psychological processes (\protect\hyperlink{ref-baron1986}{Baron \& Kenny, 1986}), it would thus make sense to pair mediation analysis with a longitudinal design because understanding causality, after all, requires temporal precedence. Unfortunately, the majority of studies that have used mediation analysis have done so using cross-sectional designs---with estimates of approximately 90\% (\protect\hyperlink{ref-maxwell2007}{Maxwell \& Cole, 2007}) and 84\% (\protect\hyperlink{ref-mitchell2013}{Mitchell \& Maxwell, 2013})---and have often discussed the results as if they were longitudinal. Investigations into whether mediation results remain equivalent across cross-sectional and longitudinal designs have repeatedly concluded that using mediation analysis on cross-sectional data can return different, and sometimes completely opposite, results from using it on longitudinal data (\protect\hyperlink{ref-cole2003}{Cole \& Maxwell, 2003}; \protect\hyperlink{ref-maxwell2011}{Maxwell et al., 2011}; \protect\hyperlink{ref-maxwell2007}{Maxwell \& Cole, 2007}; \protect\hyperlink{ref-mitchell2013}{Mitchell \& Maxwell, 2013}; \protect\hyperlink{ref-olaughlin2018}{O'Laughlin et al., 2018}). Therefore, mediation analyses based on cross-sectional analyses may be misleading.

The non-equivalence of cross-sectional and longitudinal results that occurs with mediation analysis is, unfortunately, not due to a specific set of circumstances that only arise with mediation analysis, but a consequence of a broader systematic cause that affects the results of many analyses. The concept of ergodicity explains why cross-sectional and longitudinal analyses seldom yield similar results. To understand ergodicity, it is first important to realize that variance is central to many statistical analyses---correlation, regression, factor analysis, and mediation are some examples. Thus, if variance remains unchanged across cross-sectional and longitudinal data sets, then analyses of either data set would return the same results. Importantly, variance only remains equal across cross-sectional and longitudinal data sets if two conditions put forth by ergodic theory are satisfied (homogeneity and stationarity; \protect\hyperlink{ref-molenaar2004}{Molenaar, 2004}; \protect\hyperlink{ref-molenaar2009}{Molenaar \& Campbell, 2009}). If these two conditions are met, then a process is said to be ergodic. Unfortunately, the two conditions required for ergodicity are highly unlikely to be satisfied and so cross-sectional findings will frequently deviate from longitudinal findings (for a detailed discussion, see Appendix \ref{ergodicity}).



Given that cross-sectional and longitudinal analyses are, in general, unlikely to return equivalent findings, it is unsurprising that several investigations in organizational research---and psychology as a whole---have found these analyses to return different results. Beginning with an example from Curran \& Bauer (\protect\hyperlink{ref-curran2011}{2011}), heart attacks are less likely to occur in people who exercise regularly (longitudinal finding), but more likely to happen when exercising (cross-sectional finding). Correlational studies find differences in correlation magnitudes between cross-sectional and longitudinal data sets Fisher et al. (\protect\hyperlink{ref-fisher2018}{2018}).\footnote{Note that Fisher et al. (\protect\hyperlink{ref-fisher2018}{2018}) also found the variability of longitudinal correlations to be considerably larger than the variability of cross-sectional correlations.} Moving on to perhaps the most commonly employed analysis in organizational research of mediation, several articles have highlighted cross-sectional data can return different, and sometimes completely opposite, results to longitudinal data (\protect\hyperlink{ref-cole2003}{Cole \& Maxwell, 2003}; \protect\hyperlink{ref-maxwell2011}{Maxwell et al., 2011}; \protect\hyperlink{ref-maxwell2007}{Maxwell \& Cole, 2007}; \protect\hyperlink{ref-olaughlin2018}{O'Laughlin et al., 2018}). Factor analysis is perhaps the most interesting example: The well-documented five-factor model of personality seldom arises when analyzing person-level data that was obtained by measuring personality on 90 consecutive days (\protect\hyperlink{ref-hamaker2005}{Hamaker et al., 2005}). Therefore, cross-sectional analyses are rarely equivalent to longitudinal analyses.

Fortunately, technological advancements have allowed researchers to more easily conduct longitudinal research in two ways. First, the use of the experience sampling method (\protect\hyperlink{ref-beal2015}{Beal, 2015}) in conjunction with modern information transmission technologies---whether through phone applications or short message services---allows data to sometimes be sampled over time with relative ease. Second, the development of analyses for longitudinal data (along with their integration in commonly used software) that enable person-level data to be modelled such as multilevel models (\protect\hyperlink{ref-raudenbush2002}{Raudenbush \& Bryk, 2002}), growth mixture models (\protect\hyperlink{ref-wang2007}{Mo Wang \& Bodner, 2007}), and dynamic factor analysis (\protect\hyperlink{ref-ram2013}{Ram et al., 2013}) provide researchers with avenues to explore the temporal dynamics of psychological processes. With one recent survey estimating that 43.3\% of mediation studies (26 of 60 studies) used a longitudinal design (\protect\hyperlink{ref-olaughlin2018}{O'Laughlin et al., 2018}), it appears that the prevalence of longitudinal research has increased from the 9.5\% (\protect\hyperlink{ref-roe2008}{Roe, 2008}) and 16.3\% (\protect\hyperlink{ref-mitchell2013}{Mitchell \& Maxwell, 2013}) values estimated at the beginning of the 21\textsuperscript{st} century. Although the frequency of longitudinal research appears to have increased over the past 20 years, several avenues exist where the quality of longitudinal research can be improved, and in my dissertation, I focus on investigating these avenues.

\hypertarget{understanding-patterns-of-change-that-emerge-over-time}{%
\section{Understanding Patterns of Change That Emerge Over Time}\label{understanding-patterns-of-change-that-emerge-over-time}}

Change can occur in many ways over time. One pattern of change commonly assumed to occur over time is that of linear change. When change follows a linear pattern, the rate of change over time remains constant. Unfortunately, a linear pattern places demanding restrictions on the possible trajectories of change. If change were to follow a linear pattern, then any pauses in change (or plateaus) or changes in direction could not occur: Change would simply grow over time. Unfortunately, effect sizes have been shown to diminish over time (for meta-analytic examples, see \protect\hyperlink{ref-cohen1993}{Cohen, 1993}; \protect\hyperlink{ref-griffeth2000}{Griffeth et al., 2000}; \protect\hyperlink{ref-hom1992}{Hom et al., 1992}; \protect\hyperlink{ref-riketta2008}{Riketta, 2008}; \protect\hyperlink{ref-steel1990}{Steel et al., 1990}; \protect\hyperlink{ref-steel1984}{Steel \& Ovalle, 1984}). Moreover, many variables display cyclic patterns of change over time, with mood (\protect\hyperlink{ref-larsen1990}{Larsen \& Kasimatis, 1990}), daily stress (\protect\hyperlink{ref-bodenmann2010}{Bodenmann et al., 2010}), and daily drinking behaviour (\protect\hyperlink{ref-huh2015}{Huh et al., 2015}) as some examples. Therefore, change over is unlikely to follow a linear pattern.

A more realistic pattern of change to occur over time is a nonlinear pattern (for a review, see \protect\hyperlink{ref-cudeck2007}{Cudeck \& Harring, 2007}). Nonlinear change allows the rate of change to be nonconstant; that is, change may occur more rapidly during certain periods of time, stop altogether, or reverse direction. When looking at patterns of change observed across psychology, several examples of nonlinear change have been found in the declining rate of speech errors throughout child development (\protect\hyperlink{ref-burchinal1991}{Burchinal \& Appelbaum, 1991}), rates of forgetting (\protect\hyperlink{ref-murre2015}{Murre \& Dros, 2015}), development of habits (\protect\hyperlink{ref-fournier2017}{Fournier et al., 2017}), and the formation of opinions (\protect\hyperlink{ref-xia2020}{Xia et al., 2020}). Given that nonlinear change appears more likely than linear change, my dissertation will assume change over time to be nonlinear.

\hypertarget{modelling-change}{%
\section{Methods of Modelling Nonlinear Patterns of Change Over Time}\label{modelling-change}}




Given that, unlike modelling linear change, several methods exist for modelling nonlinear change, it is important to discuss these methods. On this note, I will provide an overview of two commonly employed methods for modelling nonlinear change: 1) the polynomial approach and 2) the nonlinear function approach.\footnote{It should be noted that nonlinear change can be modelled in a variety of ways, with latent change score models (e.g., \protect\hyperlink{ref-orourke2021}{O'Rourke et al., 2021}) and spline models (e.g., \protect\hyperlink{ref-fine2020}{Fine \& Grimm, 2020}) offering some examples.}\textsuperscript{,}\footnote{The definition of a nonlinear function is mathematical in nature. Specifically, a nonlinear function contains at least one parameter that exists in the corresponding partial derivative. For example, in the logistic function $\uptheta + \frac{\upalpha - \uptheta}{1 + exp^(\frac{\upbeta - t}{\upgamma}}$ is nonlinear because $\upbeta$ exists in $\frac{\partial y}{\partial \upbeta}$ (in addition to $\upgamma$ existing in its corresponding partial derivative). The $n^{th}$ order polynomial function of $y = a + bx + cx^2 + ... + nx^n$ is linear because  the partial derivatives with respect to the parameters (i.e., $1, x^2, ..., x^n$) do not contain the associated parameter.} Importantly, the simulation experiments in my dissertation will use the nonlinear function approach to model nonlinear change.

\hypertarget{refs}{}
\begin{CSLReferences}{1}{0}
\leavevmode\vadjust pre{\hypertarget{ref-aguinis2021}{}}%
Aguinis, H., \& Bakker, R. M. (2021). Time is of the essence: Improving the conceptualization and measurement of time. \emph{Human Resource Management Review}, \emph{31}(2), 100763. \url{https://doi.org/10.1016/j.hrmr.2020.100763}

\leavevmode\vadjust pre{\hypertarget{ref-baron1986}{}}%
Baron, R. M., \& Kenny, D. A. (1986). The moderator{\textendash}mediator variable distinction in social psychological research: Conceptual, strategic, and statistical considerations. \emph{Journal of Personality and Social Psychology}, \emph{51}(6), 1173--1182. \url{https://doi.org/10.1037/0022-3514.51.6.1173}

\leavevmode\vadjust pre{\hypertarget{ref-beal2015}{}}%
Beal, D. J. (2015). ESM 2.0: State of the art and future potential of experience sampling methods in organizational research. \emph{Annual Review of Organizational Psychology and Organizational Behavior}, \emph{2}(1), 383--407. \url{https://doi.org/10.1146/annurev-orgpsych-032414-111335}

\leavevmode\vadjust pre{\hypertarget{ref-bodenmann2010}{}}%
Bodenmann, G., Atkins, D. C., Schär, M., \& Poffet, V. (2010). The association between daily stress and sexual activity. \emph{Journal of Family Psychology}, \emph{24}(3), 271--279. \url{https://doi.org/10.1037/a0019365}

\leavevmode\vadjust pre{\hypertarget{ref-burchinal1991}{}}%
Burchinal, M., \& Appelbaum, M. I. (1991). Estimating individual developmental functions: Methods and their assumptions. \emph{Child Development}, \emph{62}(1), 23--42. \url{https://doi.org/10.2307/1130702}

\leavevmode\vadjust pre{\hypertarget{ref-cohen1993}{}}%
Cohen, A. (1993). Organizational commitment and turnover: A meta-analysis. \emph{Academy of Management Journal}, \emph{36}(5), 1140--1157. \url{https://doi.org/10.2307/256650}

\leavevmode\vadjust pre{\hypertarget{ref-cole2003}{}}%
Cole, D. A., \& Maxwell, S. E. (2003). Testing mediational models with longitudinal data: Questions and tips in the use of structural equation modeling. \emph{Journal of Abnormal Psychology}, \emph{112}(4), 558--577. \url{https://doi.org/10.1037/0021-843x.112.4.558}

\leavevmode\vadjust pre{\hypertarget{ref-cudeck2007}{}}%
Cudeck, R., \& Harring, J. R. (2007). Analysis of nonlinear patterns of change with random coefficient models. \emph{Annual Review of Psychology}, \emph{58}(1), 615--637. \url{https://doi.org/10.1146/annurev.psych.58.110405.085520}

\leavevmode\vadjust pre{\hypertarget{ref-curran2011}{}}%
Curran, P. J., \& Bauer, D. J. (2011). The disaggregation of within-person and between-person effects in longitudinal models of change. \emph{Annual Review of Psychology}, \emph{62}(1), 583--619. \url{https://doi.org/10.1146/annurev.psych.093008.100356}

\leavevmode\vadjust pre{\hypertarget{ref-dalal2014}{}}%
Dalal, R. S., Bhave, D. P., \& Fiset, J. (2014). Within-person variability in job performance. \emph{Journal of Management}, \emph{40}(5), 1396--1436. \url{https://doi.org/10.1177/0149206314532691}

\leavevmode\vadjust pre{\hypertarget{ref-fine2020}{}}%
Fine, K. L., \& Grimm, K. J. (2020). Examination of nonlinear and functional mixed-effects models with nonparametrically generated data. \emph{Multivariate Behavioral Research}, 1--18. \url{https://doi.org/10.1080/00273171.2020.1754746}

\leavevmode\vadjust pre{\hypertarget{ref-fisher2008}{}}%
Fisher, C. D. (2008). What if we took within-person variability seriously? \emph{Industrial and Organizational Psychology}, \emph{1}(2), 185--189. \url{https://doi.org/10.1111/j.1754-9434.2008.00036.x}

\leavevmode\vadjust pre{\hypertarget{ref-fisher2018}{}}%
Fisher, J., Medaglia, J. D., \& Jeronimus, B. F. (2018). Lack of group-to-individual generalizability is a threat to human subjects research. \emph{Proceedings of the National Academy of Sciences}, \emph{115}(27). \url{https://doi.org/10.1073/pnas.1711978115}

\leavevmode\vadjust pre{\hypertarget{ref-fournier2017}{}}%
Fournier, M., d'Arripe-Longueville, F., Rovere, C., Easthope, C. S., Schwabe, L., El Methni, J., \& Radel, R. (2017). Effects of circadian cortisol on the development of a health habit. \emph{Health Psychology}, \emph{36}(11), 1059--1064. \url{https://doi.org/10.1037/hea0000510}

\leavevmode\vadjust pre{\hypertarget{ref-fried2004}{}}%
Fried, Y., \& Slowik, L. H. (2004). Enriching goal-setting theory with time: An integrated approach. \emph{Academy of Management Review}, \emph{29}(3), 404--422. \url{https://doi.org/10.5465/amr.2004.13670973}

\leavevmode\vadjust pre{\hypertarget{ref-george2000}{}}%
George, J. M., \& Jones, G. R. (2000). The role of time in theory and theory building. \emph{Journal of Management}, \emph{26}(4), 657--684. \url{https://doi.org/10.1177/014920630002600404}

\leavevmode\vadjust pre{\hypertarget{ref-griffeth2000}{}}%
Griffeth, R. W., Hom, P. W., \& Gaertner, S. (2000). A meta-analysis of antecedents and correlates of employee turnover: Update, moderator tests, and research implications for the next millennium. \emph{Journal of Management}, \emph{26}(3), 463--488. \url{https://doi.org/10.1177/014920630002600305}

\leavevmode\vadjust pre{\hypertarget{ref-hamaker2005}{}}%
Hamaker, E. L., Dolan, C. V., \& Molenaar, P. C. M. (2005). Statistical modeling of the individual: Rationale and application of multivariate stationary time series analysis. \emph{Multivariate Behavioral Research}, \emph{40}(2), 207--233. \url{https://doi.org/10.1207/s15327906mbr4002_3}

\leavevmode\vadjust pre{\hypertarget{ref-hom1992}{}}%
Hom, P. W., Caranikas-Walker, F., Prussia, G. E., \& Griffeth, R. W. (1992). A meta-analytical structural equations analysis of a model of employee turnover. \emph{Journal of Applied Psychology}, \emph{77}(6), 890--909. \url{https://doi.org/10.1037/0021-9010.77.6.890}

\leavevmode\vadjust pre{\hypertarget{ref-huh2015}{}}%
Huh, D., Kaysen, D. L., \& Atkins, D. C. (2015). Modeling cyclical patterns in daily college drinking data with many zeroes. \emph{Multivariate Behavioral Research}, \emph{50}(2), 184--196. \url{https://doi.org/10.1080/00273171.2014.977433}

\leavevmode\vadjust pre{\hypertarget{ref-kunisch2017}{}}%
Kunisch, S., Bartunek, J. M., Mueller, J., \& Huy, Q. N. (2017). Time in strategic change research. \emph{Academy of Management Annals}, \emph{11}(2), 1005--1064. \url{https://doi.org/10.5465/annals.2015.0133}

\leavevmode\vadjust pre{\hypertarget{ref-larsen1990}{}}%
Larsen, R. J., \& Kasimatis, M. (1990). Individual differences in entrainment of mood to the weekly calendar. \emph{Journal of Personality and Social Psychology}, \emph{58}(1), 164--171. \url{https://doi.org/10.1037/0022-3514.58.1.164}

\leavevmode\vadjust pre{\hypertarget{ref-lawrence2001}{}}%
Lawrence, T. B., Winn, M. I., \& Jennings, P. D. (2001). The temporal dynamics of institutionalization. \emph{Academy of Management Review}, \emph{26}(4), 624--644. \url{https://doi.org/10.5465/amr.2001.5393901}

\leavevmode\vadjust pre{\hypertarget{ref-maxwell2007}{}}%
Maxwell, S. E., \& Cole, D. A. (2007). Bias in cross-sectional analyses of longitudinal mediation. \emph{Psychological Methods}, \emph{12}(1), 23--44. \url{https://doi.org/10.1037/1082-989x.12.1.23}

\leavevmode\vadjust pre{\hypertarget{ref-maxwell2011}{}}%
Maxwell, S. E., Cole, D. A., \& Mitchell, M. A. (2011). Bias in cross-sectional analyses of longitudinal mediation: Partial and complete mediation under an autoregressive model. \emph{Multivariate Behavioral Research}, \emph{46}(5), 816--841. \url{https://doi.org/10.1080/00273171.2011.606716}

\leavevmode\vadjust pre{\hypertarget{ref-mill2011}{}}%
Mill, J. S. (2011). Of the law of universal causation. In \emph{A system of logic, tatiocinative and inductive: Being a connected view of the principles of evidence, and the methods of scientific investigation} (Vol. 1, pp. 392--424). Cambridge University Press. (Original work published in 1843). \url{https://doi.org/10.1017/cbo9781139149839.021}

\leavevmode\vadjust pre{\hypertarget{ref-mitchell2013}{}}%
Mitchell, M. A., \& Maxwell, S. E. (2013). A comparison of the cross-sectional and sequential designs when assessing longitudinal mediation. \emph{Multivariate Behavioral Research}, \emph{48}(3), 301--339. \url{https://doi.org/10.1080/00273171.2013.784696}

\leavevmode\vadjust pre{\hypertarget{ref-mitchell2001}{}}%
Mitchell, T. R., \& James, L. R. (2001). Building better theory: Time and the specification of when things happen. \emph{Academy of Management Review}, \emph{26}(4), 530--547. \url{https://doi.org/10.5465/amr.2001.5393889}

\leavevmode\vadjust pre{\hypertarget{ref-wang2007}{}}%
Mo Wang, \& Bodner, T. E. (2007). Growth mixture modeling. \emph{Organizational Research Methods}, \emph{10}(4), 635--656. \url{https://doi.org/10.1177/1094428106289397}

\leavevmode\vadjust pre{\hypertarget{ref-molenaar2004}{}}%
Molenaar, P. C. M. (2004). A manifesto on psychology as idiographic science: Bringing the person back into scientific psychology, this time forever. \emph{Measurement: Interdisciplinary Research \& Perspective}, \emph{2}(4), 201--218. \url{https://doi.org/10.1207/s15366359mea0204_1}

\leavevmode\vadjust pre{\hypertarget{ref-molenaar2009}{}}%
Molenaar, P. C. M., \& Campbell, C. G. (2009). The new person-specific paradigm in psychology. \emph{Current Directions in Psychological Science}, \emph{18}(2), 112--117. \url{https://doi.org/10.1111/j.1467-8721.2009.01619.x}

\leavevmode\vadjust pre{\hypertarget{ref-murre2015}{}}%
Murre, J. M. J., \& Dros, J. (2015). Replication and analysis of Ebbinghaus{'} forgetting curve. \emph{PLOS ONE}, \emph{10}(7), e0120644. \url{https://doi.org/10.1371/journal.pone.0120644}

\leavevmode\vadjust pre{\hypertarget{ref-navarro2015}{}}%
Navarro, J., Roe, R. A., \& Artiles, M. I. (2015). Taking time seriously: Changing practices and perspectives in work/organizational psychology. \emph{Journal of Work and Organizational Psychology}, \emph{31}(3), 135--145. \url{https://doi.org/10.1016/j.rpto.2015.07.002}

\leavevmode\vadjust pre{\hypertarget{ref-nixon2011}{}}%
Nixon, A. E., Mazzola, J. J., Bauer, J., Krueger, J. R., \& Spector, P. E. (2011). Can work make you sick? A meta-analysis of the relationships between job stressors and physical symptoms. \emph{Work \& Stress}, \emph{25}(1), 1--22. \url{https://doi.org/10.1080/02678373.2011.569175}

\leavevmode\vadjust pre{\hypertarget{ref-olaughlin2018}{}}%
O'Laughlin, K. D., Martin, M. J., \& Ferrer, E. (2018). Cross-sectional analysis of longitudinal mediation processes. \emph{Multivariate Behavioral Research}, \emph{53}(3), 375--402. \url{https://doi.org/10.1080/00273171.2018.1454822}

\leavevmode\vadjust pre{\hypertarget{ref-orourke2021}{}}%
O'Rourke, H. P., Fine, K. L., Grimm, K. J., \& MacKinnon, D. P. (2021). The importance of time metric precision when implementing bivariate latent change score models. \emph{Multivariate Behavioral Research}, 1--19. \url{https://doi.org/10.1080/00273171.2021.1874261}

\leavevmode\vadjust pre{\hypertarget{ref-ployhart2010}{}}%
Ployhart, R. E., \& Vandenberg, R. J. (2010). Longitudinal research: The theory, design, and analysis of change. \emph{Journal of Management}, \emph{36}(1), 94--120. \url{https://doi.org/10.1177/0149206309352110}

\leavevmode\vadjust pre{\hypertarget{ref-ram2013}{}}%
Ram, N., Brose, A., \& Molenaar, P. C. M. (2013). \emph{Dynamic factor analysis: Modeling person-specific process} (T. D. Little, Ed.; 1st ed., Vol. 2, pp. 441--457). Oxford University Press. \url{https://doi.org/10.1093/oxfordhb/9780199934898.013.0021}

\leavevmode\vadjust pre{\hypertarget{ref-raudenbush2002}{}}%
Raudenbush, S. W., \& Bryk, A. S. (2002). \emph{Hierarchical linear models: Applications and data analysis methods} (2nd ed., Vol. 1). SAGE Publications. \href{https://shorturl.at/imFN7}{shorturl.at/imFN7}

\leavevmode\vadjust pre{\hypertarget{ref-riketta2008}{}}%
Riketta, M. (2008). The causal relation between job attitudes and performance: A meta-analysis of panel studies. \emph{Journal of Applied Psychology}, \emph{93}(2), 472--481. \url{https://doi.org/10.1037/0021-9010.93.2.472}

\leavevmode\vadjust pre{\hypertarget{ref-roe2008}{}}%
Roe, R. A. (2008). Time in applied psychology. \emph{European Psychologist}, \emph{13}(1), 37--52. \url{https://doi.org/10.1027/1016-9040.13.1.37}

\leavevmode\vadjust pre{\hypertarget{ref-roe2014}{}}%
Roe, R. A. (2014). Test validity from a temporal perspective: Incorporating time in validation research. \emph{European Journal of Work and Organizational Psychology}, \emph{23}(5), 754--768. \url{https://doi.org/10.1080/1359432x.2013.804177}

\leavevmode\vadjust pre{\hypertarget{ref-roe2012}{}}%
Roe, R. A., Gockel, C., \& Meyer, B. (2012). Time and change in teams: Where we are and where we are moving. \emph{European Journal of Work and Organizational Psychology}, \emph{21}(5), 629--656. \url{https://doi.org/10.1080/1359432x.2012.729821}

\leavevmode\vadjust pre{\hypertarget{ref-shipp2015}{}}%
Shipp, A. J., \& Cole, M. S. (2015). Time in individual-level organizational studies: What is it, how is it used, and why isn{'}t it exploited more often? \emph{Annual Review of Organizational Psychology and Organizational Behavior}, \emph{2}(1), 237--260. \url{https://doi.org/10.1146/annurev-orgpsych-032414-111245}

\leavevmode\vadjust pre{\hypertarget{ref-sonnentag2012}{}}%
Sonnentag, S. (2012). Time in organizational research: Catching up on a long neglected topic in order to improve theory. \emph{Organizational Psychology Review}, \emph{2}(4), 361--368. \url{https://doi.org/10.1177/2041386612442079}

\leavevmode\vadjust pre{\hypertarget{ref-steel1990}{}}%
Steel, R. P., Hendrix, W. H., \& Balogh, S. P. (1990). Confounding effects of the turnover base rate on relations between time lag and turnover study outcomes: An extension of meta-analysis findings and conclusions. \emph{Journal of Organizational Behavior}, \emph{11}(3), 237--242. \url{https://doi.org/10.1002/job.4030110306}

\leavevmode\vadjust pre{\hypertarget{ref-steel1984}{}}%
Steel, R. P., \& Ovalle, N. K. (1984). A review and meta-analysis of research on the relationship between behavioral intentions and employee turnover. \emph{Journal of Applied Psychology}, \emph{69}(4), 673--686. \url{https://doi.org/10.1037/0021-9010.69.4.673}

\leavevmode\vadjust pre{\hypertarget{ref-vantilborgh2018}{}}%
Vantilborgh, T., Hofmans, J., \& Judge, T. A. (2018). The time has come to study dynamics at work. \emph{Journal of Organizational Behavior}, \emph{39}(9), 1045--1049. \url{https://doi.org/10.1002/job.2327}

\leavevmode\vadjust pre{\hypertarget{ref-whetten1989}{}}%
Whetten, D. A. (1989). What constitutes a theoretical contribution? \emph{Academy of Management Review}, \emph{14}(4), 490--495. \url{https://doi.org/10.5465/amr.1989.4308371}

\leavevmode\vadjust pre{\hypertarget{ref-xia2020}{}}%
Xia, W., Ye, M., Liu, J., Cao, M., \& Sun, X.-M. (2020). Analysis of a nonlinear opinion dynamics model with biased assimilation. \emph{Automatica}, \emph{120}, 109113. \url{https://doi.org/10.1016/j.automatica.2020.109113}

\leavevmode\vadjust pre{\hypertarget{ref-zaheer1999}{}}%
Zaheer, S., Albert, S., \& Zaheer, A. (1999). Time scales and organizational theory. \emph{Academy of Management Review}, \emph{24}(4), 725--741. \url{https://doi.org/10.5465/amr.1999.2553250}

\end{CSLReferences}
\end{document}
