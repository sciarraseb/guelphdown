% This is the Reed College LaTeX thesis template. Most of the work
% for the document class was done by Sam Noble (SN), as well as this
% template. Later comments etc. by Ben Salzberg (BTS). Additional
% restructuring and APA support by Jess Youngberg (JY).
% Your comments and suggestions are more than welcome; please email
% them to cus@reed.edu
%
% See https://www.reed.edu/cis/help/LaTeX/index.html for help. There are a
% great bunch of help pages there, with notes on
% getting started, bibtex, etc. Go there and read it if you're not
% already familiar with LaTeX.
%
% Any line that starts with a percent symbol is a comment.
% They won't show up in the document, and are useful for notes
% to yourself and explaining commands.
% Commenting also removes a line from the document;
% very handy for troubleshooting problems. -BTS

%%
%% Preamble
\documentclass[
12pt, % The default document font size, options: 10pt, 11pt, 12pt
twoside,
english]{guelphthesis}

%----------------------------------------------------------------------------------------
% PACKAGES
%----------------------------------------------------------------------------------------
\usepackage{hyperref}
\usepackage{tocloft} %needed for table of contents, list of figures, list of tables, list of appendices
\usepackage{graphicx,latexsym}
\usepackage{amsmath}
\usepackage{amssymb,amsthm}
\usepackage{longtable,booktabs,setspace}


\usepackage{lmodern}
\usepackage{float}
\usepackage{etoolbox}
\floatplacement{figure}{H}
% Thanks, @Xyv
\usepackage{calc}
% End of CII addition
\usepackage{rotating}
\usepackage{tocbibind} %includes list of figures, list of tables, and table of contents in table of contents
\usepackage{indentfirst} %needed so that first paragraph after each section titles has indent
\usepackage{lineno} %allows option for line numbering
\usepackage{draftwatermark} %for draft watermark
\SetWatermarkText{} %ensures draft is not printed when draft:false
\usepackage[backend=biber, style=authoryear]{biblatex}

% Syntax highlighting #22

% To pass between YAML and LaTeX the dollar signs are added by CII
\title{Is Timing Everything? Measurement Timing and the Ability to Accurately Model Longitudinal Data}
\author{Sebastian L.V. Sciarra}
\year{2022}
\date{October, 2022}
\advisor{David Stanley}
\institution{University of Guelph}
\degree{Doctorate of Philosophy}



\department{Psychology}


% From {rticles}
\newlength{\cslhangindent}
\setlength{\cslhangindent}{1cm} %indentation of hanging lines
% for Pandoc 2.8 to 2.10.1
\newenvironment{cslreferences}%
  {}%
  {\par}

% For Pandoc 2.11+
% As noted by @mirh [2] is needed instead of [3] for 2.12
\newenvironment{CSLReferences}[2] % #1 hanging-ident, #2 entry spacing
 {% don't indent paragraphs
  \setlength{\parindent}{0pt}
  % turn on hanging indent if param 1 is 1
  \ifodd #1 \everypar{\setlength{\hangindent}{\cslhangindent}}\ignorespaces\fi
  % set entry spacing
  \ifnum #2 > 0
  \setlength{\parskip}{\linespacing{2}}
  \fi
 }%
 {}



\urlstyle{rm}

%----------------------------------------------------------------------------------------
% CUSTOM COMMANDS
%----------------------------------------------------------------------------------------
%numbers lines before equations
%taken from https://tex.stackexchange.com/questions/43648/why-doesnt-lineno-number-a-paragraph-when-it-is-followed-by-an-align-equation
\newcommand*\patchAmsMathEnvironmentForLineno[1]{%
  \expandafter\let\csname old#1\expandafter\endcsname\csname #1\endcsname
  \expandafter\let\csname oldend#1\expandafter\endcsname\csname end#1\endcsname
  \renewenvironment{#1}%
     {\linenomath\csname old#1\endcsname}%
     {\csname oldend#1\endcsname\endlinenomath}}%
\newcommand*\patchBothAmsMathEnvironmentsForLineno[1]{%
  \patchAmsMathEnvironmentForLineno{#1}%
  \patchAmsMathEnvironmentForLineno{#1*}}%
\AtBeginDocument{%
\patchBothAmsMathEnvironmentsForLineno{equation}%
\patchBothAmsMathEnvironmentsForLineno{align}%
\patchBothAmsMathEnvironmentsForLineno{flalign}%
\patchBothAmsMathEnvironmentsForLineno{alignat}%
\patchBothAmsMathEnvironmentsForLineno{gather}%
\patchBothAmsMathEnvironmentsForLineno{multline}%
}


%nest all the \frontmatter functions in \oldfrontmatter, which allows us to redefine \frontmatter as everything it was with one modification to the
%draft watermark
\let\oldfrontmatter\frontmatter
%set page numbering to bottom center for \frontmatter
\fancypagestyle{frontmatter}{%
 \fancyhf{}% clear all header and footer fields
  \renewcommand{\headrulewidth}{0pt}
  \fancyhead[R]{\roman{page}}% Roman page number in footer centre

  }

\renewcommand{\frontmatter}{
  \oldfrontmatter
     \SetWatermarkLightness{0.8} %shading of draft watermark
  \SetWatermarkText{DRAFT}
  
   %set page number font to Arial if ArialFont: false in YAML header
  
   \pagestyle{frontmatter} % add this to center page numbers
}

%set page numbering to bottom center for \mainmatter
\fancypagestyle{mainmatter}{%
 \fancyhf{}% clear all header and footer fields
  \renewcommand{\headrulewidth}{0pt}
  \fancyfoot[C]{\arabic{page}}% Roman page number in footer centre

  \hypersetup{pdfpagemode={UseOutlines},
    bookmarksopen=true,
    hypertexnames=true,
    colorlinks = true,
    citecolor = blue,
    linkcolor = blue,
    urlcolor= blue,
    anchorcolor = blue,
    pdfstartview={FitV},
    breaklinks=true}

  

}

%nest all the \mainmatter functions in \oldmainmatter, which allows us to redefine \mainmatter as everything it was with one modification to the
%page numbering format
\newcommand{\setMainMatterLinespacing}{
 \setstretch{2} %default line spacing

  %change line spacing if specified in YAML header
        \setstretch{2}
  }

\let\oldmainmatter\mainmatter
\renewcommand{\mainmatter}{
  \oldmainmatter

  %change line spacing if specified in YAML header
  \setMainMatterLinespacing

      \linenumbers
  
  \pagestyle{mainmatter} % add this to center page numbers

}

%code below is important for linespacing to remain unaffected when kableExtra::landscape() is used andthe margin is specifically defined. Otherwise,
%linespacing for entire document goes to singlespacing for the text that follows the table.
\let\oldRestoreGeometry\restoregeometry
\renewcommand{\restoregeometry}{
  \oldRestoreGeometry

  %change line spacing if specified in YAML header
  \setMainMatterLinespacing
}

%change footnote and page number font to arial if desired

%----------------------------------------------------------------------------------------
%	TABLE OF CONTENTS, LIST OF FIGURES, & LIST OF TABLES
%----------------------------------------------------------------------------------------
%TABLE OF CONTENTS
\setlength{\cftbeforetoctitleskip}{0cm} %remove vertical space above table of contents

%two lines below ensure centered title for toc
%needed so that table of contents entry is not indented
\renewcommand{\contentsname}{Table of Contents} %change title for toc
\renewcommand{\cfttoctitlefont}{\hfill\fontsize{14}{14}\selectfont\bfseries\MakeUppercase}
\renewcommand{\cftaftertoctitle}{\hfill\hfill} %sometimes another \hfill is needed; depends on some setting in abovce code

%fonts for all entry level titles
\renewcommand\cftchapfont{\mdseries} %eliminate bolded chapter titles in toc
\renewcommand\cftsecfont{\mdseries} %eliminate bolded chapter titles in toc
\renewcommand\cftsubsecfont{\mdseries} %eliminate bolded chapter titles in toc
\renewcommand\cftsubsubsecfont{\mdseries} %eliminate bolded chapter titles in toc
\renewcommand\cftparafont{\mdseries} %eliminate bolded chapter titles in toc
\renewcommand\cftsubparafont{\mdseries} %eliminate bolded chapter titles in toc

%fonts for all entry level page numbers
\renewcommand{\cftchappagefont}{\mdseries} %remove bolding of page numbers for chapter headers in toc
\renewcommand\cftsecpagefont{\mdseries} %eliminate bolded chapter titles in toc
\renewcommand\cftsubsecpagefont{\mdseries} %eliminate bolded chapter titles in toc
\renewcommand\cftsubsubsecpagefont{\mdseries} %eliminate bolded chapter titles in toc
\renewcommand\cftparapagefont{\mdseries} %eliminate bolded chapter titles in toc
\renewcommand\cftsubparapagefont{\mdseries} %eliminate bolded chapter titles in toc

\renewcommand{\cftchapleader}{\cftdotfill{0.1}} %remove chapter bolding + modif dot spacing
\renewcommand{\cftdotsep}{0.1} %make dots in toc closer together

%spacing between toc items (should be all equal)
\setlength{\cftbeforechapskip}{0cm} %removes spacing before each chapter element
\renewcommand{\cftchapafterpnum}{\vskip6pt}
\renewcommand{\cftsecafterpnum}{\vskip6pt}
\renewcommand{\cftsubsecafterpnum}{\vskip6pt}
\renewcommand{\cftsubsubsecafterpnum}{\vskip6pt}
\renewcommand{\cftparaafterpnum}{\vskip6pt}
\renewcommand{\cftsubparaafterpnum}{\vskip6pt}

%remove header that appears in table of contents after first page
\renewcommand{\cftmarktoc}{}

%commands need to be redefined so that leading dots go all the way to the page numbers for all header levels (chap, sec, subsec, subsubsec, para, subpara
%%%general framework for commands below: cftXfillnum sets the format for the leading dots (\cftchapleader) and the page number (\cftchappagefont) such that leading dots proceed all the way to the page number with no spaces between dots and page number (\nobreak) at which wpoint paragraph mode ends (\par) and vertical spacing (defined  above) after item entry is inserted
%chapter (level 0)
\renewcommand{\cftchapfillnum}[1]{%
  {\cftchapleader}\nobreak
  {\cftchappagefont #1}\par\cftchapafterpnum
}

%sec (level 1)
\renewcommand{\cftsecfillnum}[1]{%
  {\cftsecleader}\nobreak
  {\cftsecpagefont #1}\par\cftsecafterpnum
}

%subsec (level 2)
\renewcommand{\cftsubsecfillnum}[1]{%
  {\cftsubsecleader}\nobreak
  {\cftsubsecpagefont #1}\par\cftsubsecafterpnum
}

%subsubsec (level 3)
\renewcommand{\cftsubsubsecfillnum}[1]{%
  {\cftsubsubsecleader}\nobreak
  {\cftsubsubsecpagefont #1}\par\cftsubsubsecafterpnum
}

%para (level 4)
\renewcommand{\cftparafillnum}[1]{%
  {\cftparaleader}\nobreak
  {\cftparapagefont #1}\par\cftparaafterpnum
}

%subpara (level 5)
\renewcommand{\cftsubparafillnum}[1]{%
  {\cftsubparaleader}\nobreak
  {\cftsubparapagefont #1}\par\cftsubparaafterpnum
}

%LIST OF TABLES
\renewcommand{\cfttabfont}{\mdseries} %set font for entries in lot
\renewcommand{\cfttabpagefont}{\mdseries} %set front for page numbers

\setlength{\cftbeforelottitleskip}{0cm} %remove vertical space above table of contents
\setlength{\cftafterlottitleskip}{0.5cm} %space between title for list of tables and list entries
%two lines below ensure centered title for toc
%needed so that table of contents entry is not indented
\renewcommand{\cftlottitlefont}{\hfill\fontsize{14}{14}\selectfont\bfseries\MakeUppercase}
\renewcommand{\cftafterlottitle}{\hfill} %sometimes another \hfill is needed; depends on some setting in abovce code

%commands need to be redefined so that leading dots go all the way to the page numbers for tables
%%%general framework for command below: cftfigfillnum sets the format for the leading dots (\cftfigleader) and the page number (\cftfigpagefont) such that leading dots proceed all the way to the page number with no spaces between dots and page number (\nobreak) at which point paragraph mode ends (\par) and vertical spacing (defined  below) after item entry is inserted
\setlength{\cftbeforetabskip}{0cm} %removes spacing before each chapter element
\renewcommand{\cfttabafterpnum}{\vskip6pt}

\renewcommand{\cfttabfillnum}[1]{%
  {\cfttableader}\nobreak
  {\cfttabpagefont #1}\par\cfttabafterpnum
}

%remove header that appears in list of tables after first page
\renewcommand{\cftmarklot}{}

%LIST OF FIGURES
\renewcommand{\cftfigfont}{\mdseries} %set font for entries in lot
\renewcommand{\cftfigpagefont}{\mdseries} %set front for page numbers

\setlength{\cftbeforeloftitleskip}{0cm} %remove vertical space above table of contents
\setlength{\cftafterloftitleskip}{0.5cm} %space between title for list of figures and list entries

%two lines below ensure centered title for toc
%needed so that table of contents entry is not indented
\renewcommand{\cftloftitlefont}{\hfill\fontsize{14}{14}\selectfont\bfseries\MakeUppercase}
\renewcommand{\cftafterloftitle}{\hfill} %sometimes another \hfill is needed; depends on some setting in abovce code

%commands need to be redefined so that leading dots go all the way to the page numbers for figures
%%%general framework for command below: cftfigfillnum sets the format for the leading dots (\cftfigleader) and the page number (\cftfigpagefont) such that leading dots proceed all the way to the page number with no spaces between dots and page number (\nobreak) at which wpoint paragraph mode ends (\par) and vertical spacing (defined  below) after item entry is inserted
\setlength{\cftbeforefigskip}{0cm} %removes spacing before each chapter element
\renewcommand{\cftfigafterpnum}{\vskip6pt}

\renewcommand{\cftfigfillnum}[1]{%
  {\cftfigleader}\nobreak
  {\cftfigpagefont #1}\par\cftfigafterpnum
}

%remove header that appears in list of figures after first page
\renewcommand{\cftmarklof}{}

%----------------------------------------------------------------------------------------
% LIST OF APPENDICES
%----------------------------------------------------------------------------------------
\newcommand{\listappname}{List of Appendices}
\newlistof[chapter]{app}{loa}{\listappname} %creates a new appendix counter that will be reset at the start of each \chapter

\setcounter{loadepth}{5} %loa will  go to depth of level 5
\setlength{\cftbeforeloatitleskip}{0cm} %remove vertical space above loa
\setlength{\cftafterloatitleskip}{0.5cm} %space between title for loa and list entries
\renewcommand{\cftmarkloa}{} %remove header titles

%two lines below ensure centered title for toc
%needed so that table of contents entry is not indented
\renewcommand{\cftloatitlefont}{\hfill\fontsize{14}{14}\selectfont\bfseries\MakeUppercase}
\renewcommand{\cftafterloatitle}{\hfill\hfill} %sometimes another \hfill is needed; depends on some setting in above code


%APPENDIX (level 0)
\renewcommand{\theapp}{\Alph{app}} %sets alphabetic counter for appendix
\renewcommand{\cftappfont}{\mdseries} %set font for level 0 entry in loa
\renewcommand{\cftapppagefont}{\mdseries} %set front for page numbers

\renewcommand{\cftapppresnum}{Appendix\space}
\renewcommand{\cftappaftersnum}{:\space}
\settowidth{\cftappnumwidth}{\cftapppresnum\theapp\cftappaftersnum\space}

\setlength{\cftbeforeappskip}{0cm} %removes vertical spacing before each chapter element
\renewcommand{\cftappafterpnum}{\vskip6pt}

%updates appendix counter, modifies chapter title such so that it is Appendix _letter_: #1
\newcommand{\app}[1]{%
  \refstepcounter{app}\pdfbookmark[-1]{\cftapppresnum\theapp\cftappaftersnum#1}{#1\theapp}%
  \chapter*{\fontsize{16}{16}\selectfont\bfseries\cftapppresnum\theapp\cftappaftersnum #1} %formats entry in document
  \addcontentsline{loa}{app}{{\cftapppresnum\theapp\cftappaftersnum}#1}%
  \par
}

% figure and table counting in appendix
\usepackage{chngcntr}


%leading dots for appendix (end immediately before page number)
\renewcommand{\cftappfillnum}[1]{%
 {\cftappleader}\nobreak{\cftapppagefont #1}\par\cftappafterpnum
}

%SECAPPENDIX (level 1; format A.1 : title)
\newlistentry[app]{secapp}{loa}{1}
\renewcommand{\thesecapp}{\theapp.\arabic{secapp}}
\renewcommand{\cftsecappfont}{\mdseries} %set font for level 1 entry in loa
\renewcommand{\cftsecapppagefont}{\mdseries} %set front for page numbers

\renewcommand{\cftsecapppresnum}{} %remove word 'Appendix'
\renewcommand{\cftsecappaftersnum}{\hspace{0.5cm}}  %replicate toc format for sub-level-0 headers \thesubappendix (i.e., A.1   title )

\setlength{\cftbeforesecappskip}{0cm} %removes vertical spacing before each chapter element
\renewcommand{\cftsecappafterpnum}{\vskip6pt}
\setlength{\cftsecappindent}{1.55em} %indentation in loa
\settowidth{\cftsecappnumwidth}{\cftsecapppresnum\thesecapp\cftsecappaftersnum\hspace{0.3cm}}

%updates appendix counter, modifies chapter title such so that it is Appendix _letter_: #1
\newcommand{\secapp}[1]{%
  \refstepcounter{secapp}\pdfbookmark[0]{#1}{#1\thesubapp}%
  \section*{\thesecapp\hspace{0.3cm} #1} %spacing between section number and title in text
  \addcontentsline{loa}{secapp}{{\thesecapp\cftsecappaftersnum}#1}%
  \par
}

%leading dots for appendix (end immediately before page number)
\renewcommand{\cftsecappfillnum}[1]{%
 {\cftsecappleader}\nobreak{\cftsecapppagefont #1}\par\cftsecappafterpnum
}


%SUBAPPENDIX (level 2; format A.1.1 : title)
\newlistentry[app]{subapp}{loa}{1}
\renewcommand{\thesubapp}{\thesecapp.\arabic{subapp}}
\renewcommand{\cftsubappfont}{\mdseries} %set font for level 2 entry in loa
\renewcommand{\cftsubapppagefont}{\mdseries} %set front for page numbers

\renewcommand{\cftsubapppresnum}{} %remove word 'Appendix'
\renewcommand{\cftsubappaftersnum}{\hspace{0.5cm}}  %replicate toc format for sub-level-0 headers \thesubappendix (i.e., A.1   title )

\setlength{\cftbeforesubappskip}{0cm} %removes vertical spacing before each chapter element
\renewcommand{\cftsubappafterpnum}{\vskip6pt}
\setlength{\cftsubappindent}{3.10em} %indentation in loa
%\renewcommand{\cftsubappnumwidth}{1.47cm}
\settowidth{\cftsubappnumwidth}{\thesubapp\cftsubappaftersnum\hspace{0.3cm}}

%updates appendix counter, modifies chapter title such so that it is Appendix _letter_: #1
\newcommand{\subapp}[1]{%
  \refstepcounter{subapp}\pdfbookmark[1]{#1}{#1\thesubapp}%
  \subsection*{\thesubapp\hspace{0.3cm} #1}%
  \addcontentsline{loa}{subapp}{{\thesubapp\cftsubappaftersnum}#1}%
  \par
}

%leading dots for appendix (end immediately before page number)
\renewcommand{\cftsubappfillnum}[1]{%
 {\cftsubappleader}\nobreak{\cftsubapppagefont #1}\par\cftsubappafterpnum
}


% SUBSUBAPPENDIX (level 3; format A.1.1.1  title)
\newlistentry[app]{subsubapp}{loa}{1}
\renewcommand{\thesubsubapp}{\thesubapp.\arabic{subsubapp}}
\renewcommand{\cftsubsubappfont}{\mdseries} %set font for level 3 entry in loa
\renewcommand{\cftsubsubapppagefont}{\mdseries} %set front for page numbers


\renewcommand{\cftsubsubapppresnum}{} %remove word 'Appendix'
\renewcommand{\cftsubsubappaftersnum}{\hspace{0.5cm}}  %space after subsubapp title

\setlength{\cftbeforesubsubappskip}{0cm} %removes vertical spacing before each chapter element
\renewcommand{\cftsubsubappafterpnum}{\vskip6pt}
\setlength{\cftsubsubappindent}{4.65em} %indentation in loa (1.55 *2)
\settowidth{\cftsubsubappnumwidth}{\thesubsubapp\cftsubsubappaftersnum\hspace{0.3cm}}

%updates appendix counter, modifies chapter title such so that it is Appendix _letter_: #1
\newcommand{\subsubapp}[1]{%
  \refstepcounter{subsubapp}\pdfbookmark[2]{#1}{#1\thesubsubapp}%
  \subsubsection*{\thesubsubapp\hspace{0.3cm} #1}%
  \addcontentsline{loa}{subsubapp}{{\thesubsubapp\cftsubsubappaftersnum}#1}%
  \par
}

%leading dots for appendix (end immediately before page number)
\renewcommand{\cftsubsubappfillnum}[1]{%
 {\cftsubsubappleader}\nobreak{\cftsubsubapppagefont #1}\par\cftsubsubappafterpnum
}

% PARA (level 4; format A.1.1.1.1  title)
\newlistentry[app]{paraapp}{loa}{1}
\renewcommand{\theparaapp}{\thesubsubapp.\arabic{paraapp}}
\renewcommand{\cftparaappfont}{\mdseries} %set font for level 4 entry in loa
\renewcommand{\cftparaapppagefont}{\mdseries} %set front for page numbers

\renewcommand{\cftparaapppresnum}{} %remove word 'Appendix'
\renewcommand{\cftparaappaftersnum}{\hspace{0.5cm}}  %space after paraapp title

\setlength{\cftbeforeparaappskip}{0cm} %removes vertical spacing before each chapter element
\renewcommand{\cftparaappafterpnum}{\vskip6pt}
\setlength{\cftparaappindent}{6.2em} %indentation in loa (1.55 *2)
\settowidth{\cftparaappnumwidth}{\theparaapp\cftparaappaftersnum\hspace{0.3cm}}

%updates appendix counter, modifies chapter title such so that it is Appendix _letter_: #1
\newcommand{\paraapp}[1]{%
  \refstepcounter{paraapp}\pdfbookmark[3]{#1}{#1\theparaapp}%
  \paragraph*{\theparaapp\hspace{0.3cm} #1}%
  \addcontentsline{loa}{paraapp}{{\theparaapp\cftparaappaftersnum}#1}%
  \par
}

%leading dots for appendix (end immediately before page number)
\renewcommand{\cftparaappfillnum}[1]{%
 {\cftparaappleader}\nobreak{\cftparaapppagefont #1}\par\cftparaappafterpnum
}

% SUBPARA (level 5; format A.1.1.1.1  title)
\newlistentry[app]{subparaapp}{loa}{1}
\renewcommand{\thesubparaapp}{\theparaapp.\arabic{subparaapp}}
\renewcommand{\cftsubparaappfont}{\mdseries} %set font for level 5 entry in loa
\renewcommand{\cftsubparaapppagefont}{\mdseries} %set front for page numbers

\renewcommand{\cftsubparaapppresnum}{} %remove word 'Appendix'
\renewcommand{\cftsubparaappaftersnum}{\hspace{0.5cm}}  %space after subparaapp title

\setlength{\cftbeforesubparaappskip}{0cm} %removes vertical spacing before each chapter element
\renewcommand{\cftsubparaappafterpnum}{\vskip6pt}
\setlength{\cftsubparaappindent}{7.75em} %indentation in loa (1.55 *2)
\settowidth{\cftsubparaappnumwidth}{\thesubparaapp\cftsubparaappaftersnum\hspace{0.3cm}}

%updates appendix counter, modifies chapter title such so that it is Appendix _letter_: #1
\newcommand{\subparaapp}[1]{%
  \refstepcounter{subparaapp}\pdfbookmark[4]{#1}{#1\thesubparaapp}%
  \paragraph*{\thesubparaapp\hspace{0.3cm} #1} %paragraph is used because subparagraph has weird numbering problem
  \addcontentsline{loa}{subparaapp}{{\thesubparaapp\cftsubparaappaftersnum}#1}%
  \par
}

%SUBSUBPARA (level 6; format A.1.1.1.1.1  title)
\newlistentry[app]{subsubparaapp}{loa}{1}
\renewcommand{\thesubsubparaapp}{\thesubparaapp.\arabic{subsubparaapp}}

\renewcommand{\cftsubsubparaapppresnum}{} %remove word 'Appendix'
\renewcommand{\cftsubsubparaappaftersnum}{\hspace{0.5cm}}  %space after subparaapp title

\setlength{\cftbeforesubsubparaappskip}{0cm} %removes vertical spacing before each chapter element
\renewcommand{\cftsubsubparaappafterpnum}{\vskip6pt}
\setlength{\cftsubsubparaappindent}{9.3em} %indentation in loa (1.55 *2)
\settowidth{\cftsubsubparaappnumwidth}{\thesubsubparaapp\cftsubsubparaappaftersnum\hspace{0.3cm}}

%updates appendix counter, modifies chapter title such so that it is Appendix _letter_: #1
\newcommand{\subsubparaapp}[1]{%
  \refstepcounter{subsubparaapp}\pdfbookmark[5]{#1}{#1\thesubsubparaapp}%
  \subparagraph*{\thesubsubparaapp\hspace{0.3cm} #1} %paragraph is used because subparagraph has weird numbering problem
  \addcontentsline{loa}{subsubparaapp}{{\thesubsubparaapp\cftsubsubparaappaftersnum}#1}%
  \par
}

%leading dots for appendix (end immediately before page number)
\renewcommand{\cftsubsubparaappfillnum}[1]{%
 {\cftsubsubparaappleader}\nobreak{\cftsubsubparaapppagefont #1}\par\cftsubsubparaappafterpnum
}




%load additional latex packages needed within document
	\usepackage{booktabs}
\usepackage{longtable}
\usepackage{array}
\usepackage{multirow}
\usepackage{wrapfig}
\usepackage{float}
\usepackage{colortbl}
\usepackage{pdflscape}
\usepackage{tabu}
\usepackage{threeparttable}
\usepackage{threeparttablex}
\usepackage[normalem]{ulem}
\usepackage{makecell}
\usepackage{xcolor}


% BEGIN DOCUMENT
\begin{document}
\frontmatter %pages will be numbered with roman numerals

  \maketitle

\setcounter{page}{2} %ensures abstract page number starts at roman numberal ii

\thispagestyle{empty} %removes page number only for abstract page
  \begin{abstract}{2}{The preface pretty much says it all. This is additional content. The preface pretty much says it all. This is additional content. The preface pretty much says it all. This is additional content. The preface pretty much says it all. This is additional content. The preface pretty much says it all. This is additional content.}  %[linespacing][abstract][

  \end{abstract}

% notice how yaml variables are indexed with dollar signs and then passed into second argument of preambleItem environments
  \begin{preambleItem}{2}{Dedication}{You can have a dedication here if you wish. You can have a dedication here if you wish.You can have a dedication here if you wish.You can have a dedication here if you wish.You can have a dedication here if you wish.You can have a dedication here if you wish.You can have a dedication here if you wish.}
  \end{preambleItem}
   \begin{preambleItem}{2}{Acknowledgements}{I want to thank a few people.You can have a dedication here if you wish. You can have a dedication here if you wish.You can have a dedication here if you wish.You can have a dedication here if you wish.You can have a dedication here if you wish.You can have a dedication here if you wish.You can have a dedication here if you wish.}
  \end{preambleItem}


%move page numbers to top right for list of tables, figures, and tables
\fancypagestyle{plain}{%
  \fancyhf{}% clear all header and footer fields
  \renewcommand{\headrulewidth}{0pt}
  \fancyhead[R]{\thepage}

   }

%table of contents
  \hypersetup{linkcolor = black, pdfborder= 0 0 0} %remove red borders around toc items
  \setcounter{secnumdepth}{5}
  \setcounter{tocdepth}{5}
  \tableofcontents
  \newpage

%list of tables
  \listoftables
  \newpage

%list of figures
  \listoffigures
  \newpage

%list of appendices
  \phantomsection
  \addcontentsline{toc}{chapter}{\listappname}
  \listofapp

  \newpage

\mainmatter % here the regular arabic numbering starts

\hypertarget{thesisdownthesis_gitbook-default}{%
\chapter{thesisdown::thesis\_gitbook: default}\label{thesisdownthesis_gitbook-default}}

Placeholder

\hypertarget{the-need-to-conduct-longitudinal-research}{%
\section{The Need to Conduct Longitudinal Research}\label{the-need-to-conduct-longitudinal-research}}

\hypertarget{understanding-patterns-of-change-that-emerge-over-time}{%
\section{Understanding Patterns of Change That Emerge Over Time}\label{understanding-patterns-of-change-that-emerge-over-time}}

\hypertarget{challenges-involved-in-conducting-longitudinal-research}{%
\section{Challenges Involved in Conducting Longitudinal Research}\label{challenges-involved-in-conducting-longitudinal-research}}

\hypertarget{number-of-measurements}{%
\subsection{Number of Measurements}\label{number-of-measurements}}

\hypertarget{spacing-of-measurements}{%
\subsection{Spacing of Measurements}\label{spacing-of-measurements}}

\hypertarget{time-structuredness}{%
\subsection{Time Structuredness}\label{time-structuredness}}

\hypertarget{time-structured-data}{%
\subsubsection{Time-Structured Data}\label{time-structured-data}}

\hypertarget{time-unstructured-data}{%
\subsubsection{Time-Unstructured Data}\label{time-unstructured-data}}

\hypertarget{summary}{%
\subsection{Summary}\label{summary}}

\hypertarget{using-simulations-to-assess-modelling-accuracy}{%
\section{Using Simulations To Assess Modelling Accuracy}\label{using-simulations-to-assess-modelling-accuracy}}

\hypertarget{systematic-review-of-simulation-literature}{%
\section{Systematic Review of Simulation Literature}\label{systematic-review-of-simulation-literature}}

\hypertarget{systematic-review-methodology}{%
\subsection{Systematic Review Methodology}\label{systematic-review-methodology}}

\hypertarget{systematic-review-results}{%
\subsection{Systematic Review Results}\label{systematic-review-results}}

\hypertarget{next-steps}{%
\subsection{Next Steps}\label{next-steps}}

\hypertarget{methods-of-modelling-nonlinear-patterns-of-change-over-time}{%
\section{Methods of Modelling Nonlinear Patterns of Change Over Time}\label{methods-of-modelling-nonlinear-patterns-of-change-over-time}}

\hypertarget{overview-of-simulation-experiments}{%
\section{Overview of Simulation Experiments}\label{overview-of-simulation-experiments}}

\hypertarget{experiment-1}{%
\chapter{Experiment 1}\label{experiment-1}}

Placeholder

\hypertarget{methods}{%
\section{Methods}\label{methods}}

\hypertarget{variables-used-in-simulation-experiment}{%
\subsection{Variables Used in Simulation Experiment}\label{variables-used-in-simulation-experiment}}

\hypertarget{independent-variables}{%
\subsubsection{Independent Variables}\label{independent-variables}}

\hypertarget{spacing-measurements}{%
\paragraph{Spacing of Measurements}\label{spacing-measurements}}

\hypertarget{number-measurements}{%
\paragraph{Number of Measurements}\label{number-measurements}}

\hypertarget{population-values-set-for-the-fixed-effect-days-to-halfway-elevation-parameter-upbeta_fixed-nature-of-change}{%
\paragraph{\texorpdfstring{Population Values Set for The Fixed-Effect Days-to-Halfway Elevation Parameter \(\upbeta_{fixed}\) (Nature of Change)}{Population Values Set for The Fixed-Effect Days-to-Halfway Elevation Parameter \textbackslash upbeta\_\{fixed\} (Nature of Change)}}\label{population-values-set-for-the-fixed-effect-days-to-halfway-elevation-parameter-upbeta_fixed-nature-of-change}}

\hypertarget{constants}{%
\subsubsection{Constants}\label{constants}}

\hypertarget{dependent-variables}{%
\subsubsection{Dependent Variables}\label{dependent-variables}}

\hypertarget{convergence}{%
\paragraph{Convergence Success Rate}\label{convergence}}

\hypertarget{bias-comp}{%
\paragraph{Bias}\label{bias-comp}}

\hypertarget{pres-precision}{%
\paragraph{Precision}\label{pres-precision}}

\hypertarget{data-generation}{%
\subsection{Overview of Data Generation}\label{data-generation}}

\hypertarget{data-generation-1}{%
\subsubsection{Data Generation}\label{data-generation-1}}

\hypertarget{function-used-to-generate-each-data-set}{%
\paragraph{Function Used to Generate Each Data Set}\label{function-used-to-generate-each-data-set}}

\hypertarget{population-values-used-for-function-parameters}{%
\paragraph{Population Values Used for Function Parameters}\label{population-values-used-for-function-parameters}}

\hypertarget{data-modelling}{%
\subsection{Modelling of Each Generated Data Set}\label{data-modelling}}

\hypertarget{analysis-visualization}{%
\subsection{Analysis of Data Modelling Output and Accompanying Visualizations}\label{analysis-visualization}}

\hypertarget{convergence-analysis}{%
\subsubsection{Analysis of Convergence Success Rate}\label{convergence-analysis}}

\hypertarget{bias-analysis}{%
\subsubsection{Analysis and Visualization of Bias}\label{bias-analysis}}

\hypertarget{precision-analysis}{%
\subsubsection{Analysis and Visualization of Precision}\label{precision-analysis}}

\hypertarget{effect-size-computation-for-precision}{%
\paragraph{Effect Size Computation for Precision}\label{effect-size-computation-for-precision}}

\hypertarget{results-and-discussion}{%
\section{Results and Discussion}\label{results-and-discussion}}

\hypertarget{framework-for-interpreting-results}{%
\subsection{Framework for Interpreting Results}\label{framework-for-interpreting-results}}

\hypertarget{pre-processing-of-data-and-model-convergence}{%
\subsection{Pre-Processing of Data and Model Convergence}\label{pre-processing-of-data-and-model-convergence}}

\hypertarget{concise-tab}{%
\subsection{Equal Spacing}\label{concise-tab}}

\hypertarget{nature-change-equal-exp1}{%
\subsubsection{Nature of Change That Leads to Highest Modelling Accuracy}\label{nature-change-equal-exp1}}

\hypertarget{bias-equal-exp1}{%
\subsubsection{Bias}\label{bias-equal-exp1}}

\hypertarget{precision-equal-exp1}{%
\subsubsection{Precision}\label{precision-equal-exp1}}

\hypertarget{qualitative-equal-exp1}{%
\subsubsection{Qualitative Description}\label{qualitative-equal-exp1}}

\hypertarget{summary-of-results}{%
\subsubsection{Summary of Results}\label{summary-of-results}}

\hypertarget{time-interval-increasing-spacing}{%
\subsection{Time-Interval Increasing Spacing}\label{time-interval-increasing-spacing}}

\hypertarget{nature-change-time-inc-exp1}{%
\subsubsection{Nature of Change That Leads to Highest Modelling Accuracy}\label{nature-change-time-inc-exp1}}

\hypertarget{bias-time-inc-exp1}{%
\subsubsection{Bias}\label{bias-time-inc-exp1}}

\hypertarget{precision-time-inc-exp1}{%
\subsubsection{Precision}\label{precision-time-inc-exp1}}

\hypertarget{qualitative-time-inc-exp1}{%
\subsubsection{Qualitative Description}\label{qualitative-time-inc-exp1}}

\hypertarget{summary-of-results-1}{%
\subsubsection{Summary of Results}\label{summary-of-results-1}}

\hypertarget{time-interval-decreasing-spacing}{%
\subsection{Time-Interval Decreasing Spacing}\label{time-interval-decreasing-spacing}}

\hypertarget{nature-change-time-dec-exp1}{%
\subsubsection{Nature of Change That Leads to Highest Modelling Accuracy}\label{nature-change-time-dec-exp1}}

\hypertarget{bias-time-dec-exp1}{%
\subsubsection{Bias}\label{bias-time-dec-exp1}}

\hypertarget{precision-time-dec-exp1}{%
\subsubsection{Precision}\label{precision-time-dec-exp1}}

\hypertarget{qualitative-time-dec-exp1}{%
\subsubsection{Qualitative Description}\label{qualitative-time-dec-exp1}}

\hypertarget{summary-of-results-2}{%
\subsubsection{Summary of Results}\label{summary-of-results-2}}

\hypertarget{middle-and-extreme-spacing}{%
\subsection{Middle-and-Extreme Spacing}\label{middle-and-extreme-spacing}}

\hypertarget{nature-change-mid-ext-exp1}{%
\subsubsection{Nature of Change That Leads to Highest Modelling Accuracy}\label{nature-change-mid-ext-exp1}}

\hypertarget{bias-mid-ext-exp1}{%
\subsubsection{Bias}\label{bias-mid-ext-exp1}}

\hypertarget{precision-mid-ext-exp1}{%
\subsubsection{Precision}\label{precision-mid-ext-exp1}}

\hypertarget{qualitative-mid-ext-exp1}{%
\subsubsection{Qualitative Description}\label{qualitative-mid-ext-exp1}}

\hypertarget{summary-of-results-3}{%
\subsubsection{Summary of Results}\label{summary-of-results-3}}

\hypertarget{addressing-my-research-questions}{%
\subsection{Addressing My Research Questions}\label{addressing-my-research-questions}}

\hypertarget{does-placing-measurements-near-periods-of-change-increase-modelling-accuracy}{%
\subsubsection{Does Placing Measurements Near Periods of Change Increase Modelling Accuracy?}\label{does-placing-measurements-near-periods-of-change-increase-modelling-accuracy}}

\hypertarget{when-the-nature-of-change-is-unknown-how-should-measurements-be-spaced}{%
\subsubsection{When the Nature of Change is Unknown, How Should Measurements be Spaced?}\label{when-the-nature-of-change-is-unknown-how-should-measurements-be-spaced}}

\hypertarget{summary-of-experiment-1}{%
\section{Summary of Experiment 1}\label{summary-of-experiment-1}}

\hypertarget{experiment-2}{%
\chapter{Experiment 2}\label{experiment-2}}

Placeholder

\hypertarget{methods-1}{%
\section{Methods}\label{methods-1}}

\hypertarget{variables-used-in-simulation-experiment-1}{%
\subsection{Variables Used in Simulation Experiment}\label{variables-used-in-simulation-experiment-1}}

\hypertarget{independent-variables-1}{%
\subsubsection{Independent Variables}\label{independent-variables-1}}

\hypertarget{spacing-of-measurements-1}{%
\paragraph{Spacing of Measurements}\label{spacing-of-measurements-1}}

\hypertarget{number-of-measurements-1}{%
\paragraph{Number of Measurements}\label{number-of-measurements-1}}

\hypertarget{sample-size}{%
\paragraph{Sample Size}\label{sample-size}}

\hypertarget{constants-1}{%
\subsubsection{Constants}\label{constants-1}}

\hypertarget{dependent-variables-1}{%
\subsubsection{Dependent Variables}\label{dependent-variables-1}}

\hypertarget{convergence-success-rate}{%
\paragraph{Convergence Success Rate}\label{convergence-success-rate}}

\hypertarget{bias}{%
\paragraph{Bias}\label{bias}}

\hypertarget{precision}{%
\paragraph{Precision}\label{precision}}

\hypertarget{overview-of-data-generation}{%
\subsection{Overview of Data Generation}\label{overview-of-data-generation}}

\hypertarget{data-modelling-exp2}{%
\subsection{Modelling of Each Generated Data Set}\label{data-modelling-exp2}}

\hypertarget{analysis-of-data-modelling-output-and-accompanying-visualizations}{%
\subsection{Analysis of Data Modelling Output and Accompanying Visualizations}\label{analysis-of-data-modelling-output-and-accompanying-visualizations}}

\hypertarget{results-and-discussion-1}{%
\section{Results and Discussion}\label{results-and-discussion-1}}

\hypertarget{framework-for-interpreting-results-1}{%
\subsection{Framework for Interpreting Results}\label{framework-for-interpreting-results-1}}

\hypertarget{pre-processing-of-data-and-model-convergence-1}{%
\subsection{Pre-Processing of Data and Model Convergence}\label{pre-processing-of-data-and-model-convergence-1}}

\hypertarget{concise-example}{%
\subsection{Equal Spacing}\label{concise-example}}

\hypertarget{bias-equal-exp2}{%
\subsubsection{Bias}\label{bias-equal-exp2}}

\hypertarget{precision-equal-exp2}{%
\subsubsection{Precision}\label{precision-equal-exp2}}

\hypertarget{qualitative-equal-exp2}{%
\subsubsection{Qualitative Description}\label{qualitative-equal-exp2}}

\hypertarget{summary-of-results-4}{%
\subsubsection{Summary of Results}\label{summary-of-results-4}}

\hypertarget{time-interval-increasing-spacing-1}{%
\subsection{Time-Interval Increasing Spacing}\label{time-interval-increasing-spacing-1}}

\hypertarget{bias-time-inc-exp2}{%
\paragraph{Bias}\label{bias-time-inc-exp2}}

\hypertarget{precision-time-inc-exp2}{%
\paragraph{Precision}\label{precision-time-inc-exp2}}

\hypertarget{qualitative-time-inc-exp2}{%
\paragraph{Qualitative Description}\label{qualitative-time-inc-exp2}}

\hypertarget{summary-of-results-5}{%
\subsubsection{Summary of Results}\label{summary-of-results-5}}

\hypertarget{time-interval-decreasing-spacing-1}{%
\subsection{Time-Interval Decreasing Spacing}\label{time-interval-decreasing-spacing-1}}

\hypertarget{bias-time-dec-exp2}{%
\subsubsection{Bias}\label{bias-time-dec-exp2}}

\hypertarget{precision-time-dec-exp2}{%
\subsubsection{Precision}\label{precision-time-dec-exp2}}

\hypertarget{qualitative-time-dec-exp2}{%
\subsubsection{Qualitative Description}\label{qualitative-time-dec-exp2}}

\hypertarget{summary-of-results-6}{%
\subsubsection{Summary of Results}\label{summary-of-results-6}}

\hypertarget{middle-and-extreme-spacing-1}{%
\subsection{Middle-and-Extreme Spacing}\label{middle-and-extreme-spacing-1}}

\hypertarget{bias-mid-ext-exp2}{%
\paragraph{Bias}\label{bias-mid-ext-exp2}}

\hypertarget{precision-mid-ext-exp2}{%
\paragraph{Precision}\label{precision-mid-ext-exp2}}

\hypertarget{qualitative-mid-ext-exp2}{%
\paragraph{Qualitative Description}\label{qualitative-mid-ext-exp2}}

\hypertarget{summary-of-results-7}{%
\subsubsection{Summary of Results}\label{summary-of-results-7}}

\hypertarget{what-measurement-number-sample-size-pairings-should-be-used-with-each-spacing-schedule}{%
\section{What Measurement Number-Sample Size Pairings Should be Used With Each Spacing Schedule?}\label{what-measurement-number-sample-size-pairings-should-be-used-with-each-spacing-schedule}}

\hypertarget{experiment-3}{%
\chapter{Experiment 3}\label{experiment-3}}

In Experiment 3, I investigated how decreasing time structuredness affected modelling accuracy. Before presenting the results of Experiment 3, I will present my design and and analysis goals. For the design, I conducted a 3(time structuredness: time-structured data, time-unstructured data resulting from a fast response rate, time-unstructured data resulting from a slow response rate) x 4(number of measurements: 5, 7, 9, 11) x 6(sample size: 30, 50, 100, 200, 500, 1000) study. For the analysis, I examined whether the number of measurements and sample sizes needed to obtain high modelling accuracy (i.e., low bias, high precision) increased as time structuredness decreased.

\hypertarget{methods-2}{%
\section{Methods}\label{methods-2}}

\hypertarget{variables-used-in-simulation-experiment-2}{%
\subsection{Variables Used in Simulation Experiment}\label{variables-used-in-simulation-experiment-2}}

\hypertarget{independent-variables-2}{%
\subsubsection{Independent Variables}\label{independent-variables-2}}

\hypertarget{number-of-measurements-2}{%
\paragraph{Number of Measurements}\label{number-of-measurements-2}}

For the number of measurements, I used the same values as in Experiment 1 of 5, 7, 9, and 11 measurements (see \protect\hyperlink{number-measurements}{number of measurements} for more discussion)).

\hypertarget{sample-size-1}{%
\paragraph{Sample Size}\label{sample-size-1}}

For sample size, I used the same values as in Experiment 2 of 30, 50, 100, 200, 500, and 1000 (see \protect\hyperlink{sample-size}{sample size} for more discussion).

\hypertarget{time-structuredness-1}{%
\paragraph{Time Structuredness}\label{time-structuredness-1}}

\emph{Time structuredness} describes the extent to which, at each time point, data are obtained at the exact same time point. The manipulation of time structuredness was adopted from the manipulation used in Coulombe et al. (2016) with a slight modification. In Coulombe et al. (2016), time-unstructured data were generated according to an exponential pattern such that most data were obtained at the beginning of the response window, with a smaller amount of data being obtained towards the end of the response window. Importantly, Coulombe et al. (2016) employed a non-continuous function for generating time-unstructured data: A binning method was employed such that 80\% of the data were obtained within a time period equivalent to 12\% (fast response rate) or 30\% (slow response rate) of the entire response window. Using a response window length of 10 days with a fast response rate, the procedure employed by Coulombe et al. (2016) for generating time-unstructured data would have generated the following percentages of data in each of the four bins (note that, using the data generation procedure for Coulombe et al. (2016), the effective response window length for a fast response rate would be 4 days in the current example instead of 10 days):\footnote{The data generation procedure in (ref:coulombe2016) for a fast response rate assumed that all of the data were collected within the initial 40\% length of the nominal response window length (i.e., 4 days in the current example).}
\begin{enumerate}
\def\labelenumi{\arabic{enumi})}
\tightlist
\item
  Bin 1: 60\% of the data would be generated in the initial 10\% length of the response window (0--0.40 day).
\item
  Bin 2: 20\% of the data would be generated in the next 20\% length of the response response window (0.40--1.20 days).
\item
  Bin 3: 10\% of the data would be generated in the next 30\% length of the response window (1.20--2.40 days).
\item
  Bin 4: the remaining 10\% of the data would be generated in the remaining 40\% length of the response window (2.40--4.00 days).
\end{enumerate}
\noindent Note that, summing the data percentages and time durations from the first two bins yields an 80\% cumulative response rate that is obtained in the initial 12\% length of the full-length response window of 10 days (i.e., \((\frac{1.2}{10})100\% = 12\%\)). Also note that, in Coulombe et al. (2016), a data point in each bin was randomly assigned a measurement time within the bin's time range. In the current example where the full-length response window had a length of 10 days, a data point obtained in the first bin would be randomly assigned a measurement time between 0--0.40. Although Coulombe et al. (2016) generated time-unstructured data to resemble data collection conditions-----response rates have been shown to follow an exponential pattern (Dillman et al. (2014); Pan (2010))-----the use of a pseudo-continuous binning function for generating time-unstructured data lacked ecological validity because response patterns are more likely to follow a continuous function.
To improve on the time structuredness manipulation of Coulombe et al. (2016), I developed a more ecologically valid manipulation by using a continuous function. Specifically, I used the the exponential function shown below in Equation \ref{eq:exp-function} to generate time-unstructured data:
\begin{align}
y = M(1 - e^{-ax}),
\label{eq:exp-function} 
\end{align}
\noindent where \(x\) stores the time delay for a measurement at a particular time point, \(y\) represents the cumulative response percentage achieved at a given \(x\) time delay, \(a\) sets the rate of growth of the cumulative response percentage over time, and \(M\) sets the range of possible y values. Two important points need to be made with respect to the \(M\) parameter (range of possible \(y\) values) and the response window length used in the current simulations. First, because the range of possible values for the cumulative response percentage (\(y\)) is 0--1 (data can be collected from a 0\% to a maximum of 100\% of respondents; \(\{y : 0 \le y \le 1\}\)), the M parameter had a value of 1 (M = 1). Second, the response window length in the current simulations was 36 days, and so the range of possible time delay values was between 0--36 (\(\{x:0\le x \le36\}\)).\footnote{A value of 36 days was used because the generation of time-unstructured data had to remain independent of the manipulation of measurement number (i.e., the response window lengths used in generating time-unstructured data could not vary with the number of measurements). To ensure the manipulations of measurement number and time structuredness remained independent, the reponse window length had to remain constant for all measurement number conditions with equal spacing. Looking at Table \ref{tab:measurementDays}, the longest possible response window that fit within all measurement number conditions with equal spacing was the interval length of the 11-measurement condition (i.e., 36 days).}

To replicate the time structuredness manipulation in Coulombe et al. (2016) using the continuous exponential function of Equation \ref{eq:exp-function}, the growth rate parameter (\(a\)) had to be calibrated to achieve a cumulative response rate of 80\% after either 12\% or 30\% of the response window length of 36 days. The derivation below solves for \(a\), with Equation \ref{eq:growth-rate} showing the equation for computing \(a\).
\begin{align}
y  &= M(1 - e^{-ax}) \nonumber \\
y &= M - Me^{-ax} \nonumber \\
y &= 1 -e^{-ax} \nonumber \\
e^{-ax} &= 1-y \nonumber \\
-ax\log(e) &= \log(1 - y) \nonumber \\
a &= \frac{\log(1 - y)}{-x}
\label{eq:growth-rate}
\end{align}
\noindent Because the target response rate was 80\%, y took on a value of .80 (y = .80). Given that the response window length in the current simulations was 36 days, x took on a value of 4.32 (12\% of 36) when time-unstructured data were defined by a fast response rate and 10.80 (30\% of 36) when time-unstructured data were defined by a slow response rate. Using Equation \ref{eq:growth-rate} yielded the following growth rate parameter values for fast and slow response rates (\(a_{fast}\), \(a_{slow}\)):
\begin{align}
a_{fast} &= \frac{\log(1 - .80)}{-4.32} = 0.37 \nonumber \\
a_{slow} &= \frac{\log(1 - .80)}{-10.80} = 0.15 \nonumber
\end{align}
\noindent Therefore, to obtain 80\% of the data with a fast response rate (i.e., in 4.32 days), the growth parameter (\(a\)) needed to have a value of 0.37 (\(a_{fast}\) = 0.37) and, to obtain 80\% of the data with a slow response rate (i.e., in 10.80 days), the growth parameter (\(a\)) needed to have a value of 0.15 (\(a_{slow}\) = 0.15). Using the above growth rate values derived for the fast and slow response growth rate parameters (\(a_{fast}\), \(a_{slow}\)), the following functions were generated for fast and slow response rates:
\begin{align}
f_{fast}(x) = M(1 - e^{a_{fast}x}) = M(1 - e^{-0.37x}) \text{ and} \label{eq:cdf-fast}\\
f_{slow}(x) = M(1 - e^{a_{slow}x}) = M(1 - e^{-0.15x}).\label{eq:cdf-slow}
\end{align}
\noindent Using Equations \ref{eq:cdf-fast}--\ref{eq:cdf-slow}, Figure 10 shows the resulting cumulative distribution functions (CDF) for time-unstructured data that show the cumulative response percentage as a function of time. Panel A shows the cumulative distribution function for a fast response rate (Equation \ref{eq:cdf-fast}), where an 80\% response rate was obtained in 4.32 days. Panel B shows the cumulative distribution function for a slow response rate (Equation \ref{eq:cdf-slow}), where an 80\% response rate was obtained in 10.80 days.
\begin{apaFigure}
[portrait]
[samepage]
[0cm]
{Cumulative Distribution Functions (CDF) With Fast and Slow Response Rates}
{Figures/cdf_plot}
{.20}
{Figures/cdf_plot}
{Panel A: Cumulative distribution function for a fast response rate (Equation \ref{eq:cdf-fast}), where an 80\% response rate is obtained in 4.32 days. Panel B: Cumulative distribution function for a slow response rate (Equation \ref{eq:cdf-slow}), where an 80\% response rate is obtained in 10.80 days.}
\end{apaFigure}
\hypertarget{constants-2}{%
\subsubsection{Constants}\label{constants-2}}

Because the nature of change not manipulated in Experiment 3, I set it to have a constant value across all cells. To keep the nature of change constant across all cells, I set the fixed-effect days-to-halfway elevation parameter (\(\upbeta_{fixed}\)) to have a value of 180. Another variable set to a constant value across the cells was measurement spacing (equal spacing was used).

\hypertarget{dependent-variables-2}{%
\subsubsection{Dependent Variables}\label{dependent-variables-2}}

\hypertarget{convergence-success-rate-1}{%
\paragraph{Convergence Success Rate}\label{convergence-success-rate-1}}

The proportion of iterations in a cell where models converged defined
the \textbf{convergence success rate}.\footnote{Specifically, convergence was obtained if the convergence code returned by OpenMx was 0.} Equation \eqref{eq:convergence} below shows the calculation used to compute the convergence success rate:
\begin{align}
  \text{Convergence success rate} =  \frac{\text{Number of models that successfully converged in a cell}}{n},
  \label{eq:convergence} 
\end{align}
\noindent where \emph{n} represents the total number of models run in a cell.

\hypertarget{bias-1}{%
\paragraph{Bias}\label{bias-1}}

Bias was calculated to evaluate the accuracy with which each logistic
function parameter was estimated. As shown below in Equation
\eqref{eq:bias}, \emph{bias} was obtained by calculating the difference
between the population value set for a parameter and the average
estimated value in each cell.
\begin{align}
  \text{Bias} =  \text{Population value for parameter} - \text{Average estimated value}
  \label{eq:bias} 
\end{align}
\noindent Bias was calculated for the fixed- and random-effect parameters of the baseline (\(\uptheta_{fixed}\), \(\uptheta_{random}\)), maximal elevation (\(\upalpha_{fixed}\), \(\upalpha_{random}\)), days-to-halfway elevation (\(\upbeta_{fixed}\), \(\upbeta_{random}\)), and the halfway-triquarter delta parameters (\(\upgamma_{fixed}\), \(\upgamma_{random}\)).

\hypertarget{precision-1}{%
\paragraph{Precision}\label{precision-1}}

In addition to computing bias, precision was calculated to evaluate the confidence with which each parameter was estimated in a given cell. \emph{Precision} was obtained by computing the range of values covered by the middle 95\% of values estimated for a logistic parameter in each cell. By using the middle 95\% of estimated values, a plausible range of population estimates was obtained.

\hypertarget{overview-of-data-generation-1}{%
\subsection{Overview of Data Generation}\label{overview-of-data-generation-1}}

Data generation was computed the same way as in Experiment 1 (see \protect\hyperlink{data-generation}{data generation}) with one addition to the procedure needed for time structuredness. The section that follows details how time structuredness was simulated.

\hypertarget{simulation-procedure-for-time-structuredness}{%
\paragraph{Simulation Procedure for Time Structuredness}\label{simulation-procedure-for-time-structuredness}}

To simulate time-unstructured data, response rates at each collection
point followed an exponential pattern described by either a fast or slow
response rate (for a review, see \protect\hyperlink{time-structuredness}{time structuredness}). Importantly, data generated
for each person at each time point had to be sampled according to a
probability density function defined by either the fast or slow response
rate cumulative distribution function. In the current context, a
\emph{probability density function} describes the probability of sampling
any given time delay value \(x\) where the range of time delay values is
0--36 (\(\{x : 0 \le x \le 36 \}\)). To obtain the probability density functions
for fast and slow response rates, the response rate function shown in
Equation \eqref{eq:exponential} was differentiated with respect to \(x\) to
obtain the function shown below in Equation \ref{eq:pdf-function}\footnote{Euler's notation for differentiation is used to represent derivatives. In words, $\frac{\partial f(x)}{\partial x}$ means that the derivative of the function $f(x)$ is taken with respect to $x$.}:
\begin{align}
f^\prime = \frac{\partial f(x)}{\partial x} &= \frac{\partial}{\partial x}M(1 - e^{-ax}). \nonumber \\
&= M (e^{-ax}a)
\label{eq:pdf-function}
\end {align}

\noindent To compute the probability density function for the fast
response rate cumulative distribution function, the growth rate
parameter \(a\) was set to 0.37 in Equation \ref{eq:pdf-function} to
obtain the following function in Equation \ref{eq:fast-pdf-function}:
\begin{align}
f^\prime_{fast}(x) = M (e^{-a_{fast}x}a_{fast}) = M (e^{-0.37x}0.37). 
\label{eq:fast-pdf-function}
\end {align}

\noindent To compute the probability density function for the slow
response rate cumulative distribution function, the growth rate
parameter \(a\) was set to 0.15 in Equation \ref{eq:pdf-function} to
obtain the following function in Equation \ref{eq:slow-pdf-function}:
\begin{align}
f^\prime_{slow}(x) = M (e^{-0.15}a_{slow}) = M (e^{-0.15}0.15). 
\label{eq:slow-pdf-function}
\end {align}

Figure \ref{fig:cdf-pdf-plots} shows the fast and slow response
cumulative distribution functions (CDF) and their corresponding
probability density functions (PDF). Panel A shows the cumulative
distribution function for the fast response rate (with a growth
parameter value \(a\) set to 0.37; see Equation \ref{eq:cdf-fast}) and
Panel B shows the probability density function that results from
computing the derivative of the fast response rate cumulative
distribution function with respect to \(x\) (see Equation
\ref{eq:fast-pdf-function}). Panel C shows the cumulative distribution
function for the slow response rate (with a growth parameter value \(a\)
set to 0.15; see Equation \ref{eq:cdf-slow})) and Panel D shows the
probability density function that results from computing the derivative
of the slow response rate cumulative distribution function with respect
to \(x\) (see Equation \ref{eq:slow-pdf-function} and section on \protect\hyperlink{sec:time-structuredness}{time
structuredness} for more discussion). For the
fast response rate functions, an 80\% response rate is obtained after
4.32 days or, equivalently, 80\% of the area underneath the probability
density function is obtained at 4.32 days
(\(\int^{4.32}_{0} f_{fast}^\prime (x) = 0.80\); the integral from 0 to 4.32 of the probability density function for a fast response rate \(f^\prime(x)_{fast}\) is 0.80). For the slow response
rate functions, an 80\% response rate is obtained after 10.80 days or,
equivalently, 80\% of the area underneath the probability density
function is obtained at 10.80 days
(\(\int^{10.80}_{0} f_{slow}^\prime (x) = 0.80\); the integral from 0 to 10.80 of the probability density function for a slow response rate \(f^\prime(x)_{slow}\) is 0.80).
\begin{apaFigure}
[portrait]
[samepage]
[0cm]
{Cumulative Distribution Functions (CDF) and Probability Density Functions (PDF) for Fast and Slow Response Rates}
{cdf-pdf-plots}
{.38}
{Figures/cdf_pdf_plots}
{Panel A: Cumulative distribution function for the fast response rate (with a growth parameter value $a$ set to 0.37; see Equation \ref{eq:cdf-fast}). Panel B: Probability density function that results from computing the derivative of the fast response rate cumulative distribution function with respect to $x$ (see Equation \ref{eq:fast-pdf-function}). Panel C: Cumulative distribution function for the slow response rate (with a growth parameter value $a$ set to 0.15; see Equation \ref{eq:cdf-slow}). Panel D: Probability density function that results from computing the derivative of the slow response rate cumulative distribution function with respect to $x$ (see Equation \ref{eq:slow-pdf-function} and \nameref{time-structuredness} for more discussion on time structuredness). For the fast response rate functions, an 80\% response rate is obtained after 4.32 days or, equivalently, 80\% of the area underneath the probability density function is obtained at 4.32 days ($\int^{4.32}_{0} f_{fast}^\prime (x) = 0.80$). For the slow response rate functions, an 80\% response rate is obtained after 10.80 days or, equivalently, 80\% of the area underneath the probability density function is obtained at 10.80 days ($\int^{10.80}_{0} f_{slow}^\prime (x) = 0.80$).}
\end{apaFigure}
Having computed probability density functions for fast and slow response rates, time delays could be generated to create time-unstructured data. To generate time-unstructured data for a person at a given time point, a time delay was first
generated by sampling values according to the probability density function defined by either a fast or slow response rate (Equations \ref{eq:fast-pdf-function}--\ref{eq:slow-pdf-function}). The sampled time delay was then added to the value of the current measurement day, with the combined measurement day then being plugged into the logistic function (Equation \ref{eq:logFunction-generation}) along with a set of person-specific parameter values to generate an observed score at a given time point for a given person.

\hypertarget{data-modelling-exp3}{%
\subsection{Modelling of Each Generated Data Set}\label{data-modelling-exp3}}

Each generated data set was modelled using the structured latent growth curves outlined in Experiment 1 (see \protect\hyperlink{data-modelling}{data modelling}. For a detailed explanation of how the logistic function was fit into the structural equation modelling framework, see \protect\hyperlink{structured-latent}{Technical Appendix B}.

\hypertarget{analysis-of-data-modelling-output-and-accompanying-visualizations-1}{%
\subsection{Analysis of Data Modelling Output and Accompanying Visualizations}\label{analysis-of-data-modelling-output-and-accompanying-visualizations-1}}

Analysis and visualization was conducted as outlined in Experiment 1 (see \protect\hyperlink{analysis-visualization}{analysis and visualization}).

\hypertarget{results-and-discussion-2}{%
\section{Results and Discussion}\label{results-and-discussion-2}}

In the sections that follow, I organize the results by presenting them for each level of time structuredness (time-structured data, time-unstructured data resulting from a fast response rate, time-unstructured data resulting from a slow response rate). Importantly, only the results for the day-unit parameters will be presented (i.e., fixed- and random-effect days-to-halfway elevation and halfway-triquarter delta parameters {[}\(\upbeta_{fixed}\), \(\upbeta_{random}\), \(\upgamma_{fixed}\), \(\upgamma_{random}\), respectively{]}). The results for the likert-unit parameters (i.e., fixed- and random-effect baseline and maximal elevation parameters {[}\(\uptheta_{fixed}\), \(\uptheta_{random}\), \(\upalpha_{fixed}\), \(\upalpha_{random}\), respectively{]}) were largely trivial and so are presented in \protect\hyperlink{appendix-c}{Appendix C}).

For each level of time structuredness, I first provide a concise summary of the results and then provide a detailed report of the estimation accuracy of each day-unit parameter of the logistic function. Because the lengths of the detailed reports are considerable, I first provide concise summaries to establish a framework to interpret the detailed reports. The detailed report of each time structuredness level will summarize the results of each day-unit's bias/precision plot, report partial \(\upomega^2\) values, and then provide a qualitative summary.

\hypertarget{framework-for-interpreting-results-2}{%
\subsection{Framework for Interpreting Results}\label{framework-for-interpreting-results-2}}

To conduct Experiment 3, the three variables of number of measurements (4 levels), sample size (6 levels), and time structuredness (3 levels) were manipulated, which yielded a total of 72 cells. Importantly, within each cell, bias and precision values were also computed for each of the nine parameters estimated by the structured latent growth curve models (for a review, see \protect\hyperlink{modelling-data-sets}{modelling of each generated data set}). Thus, because the analysis of Experiment 3 computes values for many dependent variables, interpreting the results can become overwhelming. Therefore, I will provide a framework to help the reader efficiently navigate the results section.

Because I will present the results of Experiment 3 by each level of time structuredness, the framework I will describe in Figure \ref{fig:results-plot-primer} shows a template for the bias/precision plots that I will present for each level of time structuredness. The results presented for each time structuredness level contain a bias/precision plot for each of the nine estimated parameters. Each bias/precision plot shows the bias and precision for the estimation of one parameter across all measurement number and nature-of change levels. Within each bias/precision plot, dots indicate the average estimated value (which indicates bias bias) and error bars represent the middle 95\% range of estimated values (which indicates precision). Bias/precision plots with black outlines show the results for day-unit parameters and plots with gray outlines show the results for Likert-unit parameters. Importantly, only the results for the day-unit parameters will be presented (i.e., fixed- and random-effect days-to-halfway elevation and halfway-triquarter delta parameters {[}\(\upbeta_{fixed}\), \(\upbeta_{random}\), \(\upgamma_{fixed}\), \(\upgamma_{random}\), respectively{]}). The results for the Likert-unit parameters (i.e., fixed- and random-effect baseline and maximal elevation parameters {[}\(\uptheta_{fixed}\), \(\uptheta_{random}\), \(\upalpha_{fixed}\), \(\upalpha_{random}\), respectively{]}) were largely trivial and so are presented in \protect\hyperlink{appendix-b}{Appendix B}. Therefore, the results of time structuredness level will only present the bias/precision plots for four parameters (i.e., the day-unit parameters).
\begin{apaFigure}
[portrait]
[samepage]
[-0.2cm]
{Set of Bias/Precision Plots Constructed for Each Spacing Schedule in Experiment 2}
{results-plot-primer-exp2}
{.77}
{Figures/logistic_results_plot_exp2}
{A parameter estimation plot is constructed for each parameter of the logistic function (see Equation \ref{eq:logFunction-generation}). Note that each parameter of the logistic function is modelled as a fixed and random effect along with an error term ($\upepsilon$; for a review, see Figure \ref{fig:combined_plot}).}
\end{apaFigure}
\hypertarget{pre-processing-of-data-and-model-convergence-2}{%
\subsection{Pre-Processing of Data and Model Convergence}\label{pre-processing-of-data-and-model-convergence-2}}

After collecting the output from the simulations, non-converged models
(and their corresponding parameter estimates) were removed from
subsequent analyses. Table \ref{tab:conv-exp-3} in \protect\hyperlink{appendix-a-convergence-rates}{Appendix B} provides the convergence
success rates for each cell in Experiment 3. Model convergence was almost always above 90\% and convergence rates
rates below 90\% only occurred in two cells with five measurements.

\hypertarget{concise-example-exp3}{%
\subsection{Time-Structured Data}\label{concise-example-exp3}}

For time-structured data, Table \ref{tab:summary-table-time-struc-exp3} provides a concise summary of the results for the day-unit parameters (see Figure \ref{fig:exp3_plot_days_time_struc} for the corresponding parameter estimation plots). The sections that follow will present the results for each column of Table \ref{tab:summary-table-time-struc-exp3} and provide elaboration when necessary.

Before presenting the results for equal spacing, I provide a brief description of the concise summary table created for each spacing schedule and shown for equal spacing in Table \ref{tab:summary-table-time-struc-exp3}. ext in the `Unbiased' and `Precise' columns indicates the measurement number-sample size pairings that, respectively, result in unbiased and precise estimation. Emboldened text in the `Unbiased' and `Qualitative Description' columns indicates the measurement number-sample size pairing needed to, respectively, obtain unbiased estimates and the greatest improvements in bias and precision across all day-unit parameters (acceptable precision not achieved in the estimation of all day-unit parameters with equal spacing). The `Error Bar Length' column indicates the error bar length that results from using the lower-bounding measurement number-sample size pairing listed in the `Qualitative Description' column.

\newgeometry{margin=2.54cm}
\begin{landscape}
\begin{ThreePartTable}
\begin{TableNotes}
\item \textit{Note. }Text in the `Unbiased' and `Precise' columns indicates the measurement number-sample size pairings that, respectively, result in unbiased and precise estimation. Emboldened text in the `Unbiased' and `Qualitative Description' columns indicates the number of measurements needed to, respectively, obtain unbiased estimates and the greatest improvements in bias and precision across all day-unit parameters (acceptable precision not achieved in the estimation of all day-unit parameters with equal spacing). `Error Bar Length' column indicates the maximum error bar length that results from using the measurement number-sample size recommendation listed in the `Qualitative Description' column. Parameter names and population values are as follows: $\upbeta_{fixed}$ = fixed-effect days-to-halfway elevation parameter = 180; $\upgamma_{fixed}$ = fixed-effect halfway-triquarter delta parameter = 20; $\upbeta_{random}$ = random-effect days-to-halfway elevation parameter = 10; $\upgamma_{random}$ = random-effect halfway-triquarter delta parameter = 4. NM = number of measurements.
\end{TableNotes}
\begin{longtable}[l]{>{\raggedright\arraybackslash}p{3cm}>{\raggedright\arraybackslash}p{5cm}>{\raggedright\arraybackslash}p{5cm}>{\raggedright\arraybackslash}p{6.5cm}>{\raggedright\arraybackslash}p{3cm}}
\caption{\label{tab:summary-table-time-struc-exp3}Concise Summary of Results for Time-Structured Data in Experiment 3}\\
\toprule
\multicolumn{3}{c}{ } & \multicolumn{2}{c}{Description} \\
\cmidrule(l{3pt}r{3pt}){4-5}
Parameter & Unbiased & Precise & Qualitative Description & Error Bar Length\\
\midrule
\thead[lt]{$\upbeta_{fixed}$ \\ (Figure \ref{fig:exp3_plot_days_time_struc}A)} & All cells & All cells & Unbiased and precise estimation in all cells & 15.13\\
\thead[lt]{$\gamma_{fixed}$ \\ (Figure \ref{fig:exp3_plot_days_time_struc}B)} & All cells & NM $\ge$ 9 with \textit{N} = 500 & \thead[lt]{Largest improvements in precision \\ 
                                                      using \textbf{NM = 7 with \textit{N} $\ge$ 200} \vphantom{1} or \\
                                                      \textbf{NM = 9 with \textit{N} $\le$ 100}} & 9.79\\
\thead[lt]{$\upbeta_{random}$ \\ (Figure \ref{fig:exp3_plot_days_time_struc}C)} & All cells & No cells & Largest improvements in precision with NM = 7 & 17.22\\
\thead[lt]{$\upgamma_{random}$ \\ (Figure \ref{fig:exp3_plot_days_time_struc}D)} & \thead[lt]{\textbf{NM $\boldsymbol{\ge}$ 9 with \textit{N} $\ge$ 200}} & No cells & \thead[lt]{Largest improvements in precision \\ 
                                                      using \textbf{NM = 7 with \textit{N} $\ge$ 200} or \\
                                                      \textbf{NM = 9 with \textit{N} $\le$ 100}} & 10.08\\
\bottomrule
\insertTableNotes
\end{longtable}
\end{ThreePartTable}
\end{landscape}
\restoregeometry

\hypertarget{bias-time-struc-exp3}{%
\paragraph{Bias}\label{bias-time-struc-exp3}}

Before presenting the results for bias, I provide a description of the set of parameter estimation plots shown in Figure \ref{fig:exp3_plot_days_time_struc} and in the results sections for the other spacing schedules in Experiment 2. Figure \ref{fig:exp3_plot_days_time_struc} shows the parameter estimation plots for each day-unit parameter and Table \ref{tab:omega-exp2-equal} provides the partial \(\upomega^2\) values for each independent variable of each day-unit parameter. In Figure \ref{fig:exp3_plot_days_time_struc}, blue horizontal lines indicate the population values for each parameter (with population values of \(\upbeta_{fixed}\) = 180.00, \(\upbeta_{random}\) = 10.00, \(\upgamma_{fixed}\) = 20.00, and \(\upgamma_{random}\) = 4.00). Gray bands indicate the \(\pm 10\%\) margin of error for each parameter and unfilled dots indicate cells with average parameter estimates outside of the margin. Error bars represent the middle 95\% of estimated values, with light blue error bars indicating imprecise estimation. I considered dots that fell outside the gray bands as biased and error bar lengths with at least one whisker length exceeding the 10\% cutoff (i.e., or longer than the portion of the gray band underlying the whisker) as imprecise. Panels A--B show the parameter estimation plots for the fixed- and random-effect days-to-halfway elevation parameters (\(\upbeta_{fixed}\) and \(\upbeta_{random}\), respectively). Panels C--D show the parameter estimation plots for the fixed- and random-effect triquarter-halfway delta parameters (\(\upgamma_{fixed}\) and \(\upgamma_{random}\), respectively). Note that random-effect parameter units are in standard deviation units.

With respect to bias for time-structured data, estimates are biased (i.e., above the acceptable 10\% cutoff) for each day-unit parameter in the following cells:
\begin{itemize}
\tightlist
\item
  fixed-effect days-to-halfway elevation parameter (\(\upbeta_{fixed}\); Figure \ref{fig:exp3_plot_days_time_struc}A): no cells.
\item
  fixed-effect halfway-triquarter delta parameter (\(\upgamma_{fixed}\); Figure \ref{fig:exp3_plot_days_time_struc}B): no cells.
\item
  random-effect days-to-halfway elevation parameter (\(\upbeta_{random}\); Figure \ref{fig:exp3_plot_days_time_struc}C): no cells.
\item
  random-effect triquarter-halfway elevation parameter (\(\upgamma_{random}\); Figure \ref{fig:exp3_plot_days_time_struc}D): five and seven measurements across all sample sizes and nine and 11 measurements with \(N \le 100\).
\end{itemize}
In summary, with time-structured data, estimation of all the day-unit parameters across all manipulated nature-of-change values is unbiased using at least nine measurements with \(N \ge 200\), which is indicated by the emboldened text in the `Unbiased' column of Table \ref{tab:summary-table-time-struc-exp3}.
\begin{apaFigure}
[portrait]
[samepage]
[-0.2cm]
{Parameter Estimation Plots for Day-Unit Parameters With Time-Structured Data in Experiment 3}
{exp3_plot_days_time_struc}
{0.165}
{Figures/exp3_plot_days_time structured}
{Panel A: Parameter estimation plot for the fixed-effect days-to-halfway elevation parameter ($\upbeta_{fixed}$). Panel B: Parameter estimation plot for the fixed-effect triquarter-halfway elevation parameter ($\upgamma_{fixed}$). Panel C: Parameter estimation plot for the random-effect days-to-halfway elevation parameter ($\upbeta_{random}$). Panel D: Parameter estimation plot for the random-effect triquarter-halfway elevation parameter ($\upgamma_{random}$). Blue horizontal lines in each panel represent the population value for each parameter. Population values for each day-unit parameter are as follows: $\upbeta_{fixed}$ = 180.00, $\upbeta_{random}$ = 10.00, $\upgamma_{fixed}$ = 20.00, $\upgamma_{random}$ = 4.00. Gray bands indicate the $\pm 10\%$ margin of error for each parameter and unfilled dots indicate cells with average parameter estimates outside of the margin or biased estimates. Error bars represent the middle 95\% of estimated values, with light blue error bars indicating imprecise estimation. I considered dots that fell outside the gray bands as biased and error bar lengths with at least one whisker length exceeding the 10\% cutoff (i.e., or longer than the portion of the gray band underlying the whisker) as imprecise. Note that random-effect parameter units are in standard deviation units. See Table \ref{tab:param-exp-3} for specific values estimated for each parameter and Table \ref{tab:omega-exp3-time-struc} for $\upomega^2$ effect size values.}
\end{apaFigure}
\begin{ThreePartTable}
\begin{TableNotes}
\item NM = number of measurements (5, 7, 9, 11), S = sample size (30, 50, 100, 200, 500, 100), NM x S = interaction between number of measurements and sample size.
\end{TableNotes}
\begin{longtable}[l]{>{\raggedright\arraybackslash}p{6cm}ccc}
\caption{\label{tab:omega-exp3-time-struc}Partial $\upomega^2$ Values for Manipulated Variables With Time-Structured Data in Experiment 3}\\
\toprule
\multicolumn{1}{c}{ } & \multicolumn{3}{c}{Effect} \\
\cmidrule(l{3pt}r{3pt}){2-4}
Parameter & NM & S & NM x S\\
\midrule
$\upbeta_{fixed}$ (Figure \ref{fig:exp3_plot_days_time_struc}A) & 0.00 & 0.02 & 0.00\\
$\upbeta_{random}$ (Figure \ref{fig:exp3_plot_days_time_struc}B) & 0.14 & 0.27 & 0.03\\
$\upgamma_{fixed}$ (Figure \ref{fig:exp3_plot_days_time_struc}C) & 0.25 & 0.12 & 0.07\\
$\upgamma_{random}$ (Figure \ref{fig:exp3_plot_days_time_struc}D) & 0.18 & 0.03 & 0.01\\
\bottomrule
\insertTableNotes
\end{longtable}
\end{ThreePartTable}
\hypertarget{precision-time-struc-exp3}{%
\paragraph{Precision}\label{precision-time-struc-exp3}}

With respect to precision for time-structured data, estimates are imprecise (i.e., error bar length with at least one whisker length exceeding 10\% of a parameter's population value) in the following cells for each day-unit parameter:
\begin{itemize}
\tightlist
\item
  fixed-effect days-to-halfway elevation parameter (\(\upbeta_{fixed}\); Figure \ref{fig:exp3_plot_days_time_struc}A): no cells.
\item
  fixed-effect halfway-triquarter delta parameter (\(\upgamma_{fixed}\); Figure \ref{fig:exp3_plot_days_time_struc}B): five and seven measurements across all sample sizes and nine and 11 measurements with \(N \le 200\).
\item
  random-effect days-to-halfway elevation parameter (\(\upbeta_{random}\); Figure \ref{fig:exp3_plot_days_time_struc}C): all cells.
\item
  random-effect halfway-triquarter delta parameter {[}\(\upgamma_{random}\){]} in Figure \ref{fig:exp3_plot_days_time_struc}D): all cells.
\end{itemize}
In summary, with time-structured data, precise estimation can be obtained for the fixed-effect day-unit parameters using at least nine measurements with \(N \ge 500\), but no manipulated measurement number-sample size pairing results in precise estimation of the random-effect day-unit parameters (see the `Precise' column of Table \ref{tab:summary-table-time-struc-exp3}).

\hypertarget{qualitative-time-struc-exp3}{%
\paragraph{Qualitative Description}\label{qualitative-time-struc-exp3}}

For time-structured data in Figure \ref{fig:exp3_plot_days_time_struc}, although no manipulated measurement number results in precise estimation of all the day-unit parameters, the largest improvements in precision (and bias) result from using moderate measurement number-sample size pairings. With respect to bias under time-structured data, the largest improvements in bias result with the following measurement number-sample size pairings for the random-effect triquarter-halfway delta parameter (\(\upgamma_{fixed}\)):
\begin{itemize}
\tightlist
\item
  random-effect triquarter-halfway delta parameter (\(\upgamma_{random}\)): seven measurements with \(N \ge 100\) or nine measurements with \(N \le 50\).
\end{itemize}
\noindent With respect to precision under time-structured data, the largest improvements in precision for the estimation of all the day-unit parameters (except the fixed-effect days-to-halfway elevation parameter {[}\(\upbeta_{fixed}\){]}) result from using the following measurement number-sample size pairings:
\begin{itemize}
\tightlist
\item
  fixed-effect triquarter-halfway delta parameter (\(\upgamma_{fixed}\)): seven measurements with \(N \ge 200\) or nine measurements with \(N \le 100\), which results in a maximum error bar length of 9.79 days.
\item
  random-effect days-to-halfway elevation parameter (\(\upbeta_{random}\)): seven measurements across all manipulated sample sizes, which which results in a error bar length of 17.22 days.
\item
  random-effect triquarter-halfway delta parameter (\(\upgamma_{random}\)): seven measurements with \(N \ge 200\) or nine measurements with \(N \le 100\), which results in a maximum error bar length of 10.08 days.
\end{itemize}
For an applied researcher, one plausible question might be what measurement number-sample size pairing(s) results in the greatest improvements in bias and precision in the estimation of all day-unit parameters with time-structured data. In looking across the measurement number-sample size pairings in the above lists, it becomes apparent that greatest improvements in bias and precision in the estimation of all day-unit parameters result with the following measurement number-sample size pairing(s): seven measurements with \(N \ge 200\) or nine measurements with \(N \le 100\) (see the emboldened text in the `Qualitative Description' column of Table \ref{tab:summary-table-time-struc-exp3}).

\hypertarget{summary-of-results-8}{%
\subsubsection{Summary of Results}\label{summary-of-results-8}}

In summarizing the results for time-structured data, estimation of all day-unit parameters is unbiased using least nine measurements with \(N \ge 200\) (see \protect\hyperlink{bias-time-struc-exp3}{bias}). Precise estimation is never obtained in the estimation of all day-unit parameters with any manipulated measurement number-sample size pairing (see \protect\hyperlink{precision-time-struc-exp3}{precision}). Although it may be discouraging that no manipulated measurement number-sample size pairing under equal spacing results in precise estimation of all day-unit parameters, the largest improvements in precision (and bias) across all day-unit parameters are obtained with moderate measurement number-sample size pairings. With time-structured data, the largest improvements in bias and precision in the estimation of all day-unit parameters are obtained using seven measurements with \(N \ge 200\) or nine measurements with \(N \le 100\) (see \protect\hyperlink{qualitative-time-struc-exp3}{qualitiative description}).

\hypertarget{time-unstructured-data-characterized-by-a-fast-response-rate}{%
\subsection{Time-Unstructured Data Characterized by a Fast Response Rate}\label{time-unstructured-data-characterized-by-a-fast-response-rate}}

For time-unstructured data characterized by a fast response rate, Table \ref{tab:summary-table-fast-exp3} provides a concise summary of the results for the day-unit parameters (see Figure \ref{fig:exp3_plot_days_fast} for the corresponding parameter estimation plots). The sections that follow will present the results for each column of Table \ref{tab:summary-table-fast-exp3} and provide elaboration when necessary (for a description of Table \ref{tab:summary-table-fast-exp3}, see \protect\hyperlink{concise-example-exp3}{concise summary}).

\newgeometry{margin=2.54cm}
\begin{landscape}
\begin{ThreePartTable}
\begin{TableNotes}
\item \textit{Note. }Text in the `Unbiased' and `Precise' columns indicates the measurement number-sample size pairings that, respectively, result in unbiased and precise estimation. Emboldened text in the `Unbiased' and `Qualitative Description' columns indicates the number of measurements needed to, respectively, obtain unbiased estimates and the greatest improvements in bias and precision across all day-unit parameters (acceptable precision not achieved in the estimation of all day-unit parameters with equal spacing). `Error Bar Length' column indicates the maximum error bar length that results from using the measurement number-sample size recommendation listed in the `Qualitative Description' column. Parameter names and population values are as follows: $\upbeta_{fixed}$ = fixed-effect days-to-halfway elevation parameter = 180; $\upgamma_{fixed}$ = fixed-effect halfway-triquarter delta parameter = 20; $\upbeta_{random}$ = random-effect days-to-halfway elevation parameter = 10; $\upgamma_{random}$ = random-effect halfway-triquarter delta parameter = 4. NM = number of measurements.
\end{TableNotes}
\begin{longtable}[l]{>{\raggedright\arraybackslash}p{3cm}>{\raggedright\arraybackslash}p{5cm}>{\raggedright\arraybackslash}p{5cm}>{\raggedright\arraybackslash}p{6.5cm}>{\raggedright\arraybackslash}p{3cm}}
\caption{\label{tab:summary-table-fast-exp3}Concise Summary of Results for Time-Unstructured Data (Fast Response Rate) in Experiment 3}\\
\toprule
\multicolumn{3}{c}{ } & \multicolumn{2}{c}{Description} \\
\cmidrule(l{3pt}r{3pt}){4-5}
Parameter & Unbiased & Precise & Qualitative Description & Error Bar Length\\
\midrule
\thead[lt]{$\upbeta_{fixed}$ \\ (Figure \ref{fig:exp3_plot_days_fast}A)} & All cells & All cells & Unbiased and precise estimation in all cells & 15.35\\
\thead[lt]{$\gamma_{fixed}$ \\  (Figure \ref{fig:exp3_plot_days_fast}B)} & All cells & NM $\ge$ 9 with \textit{N} $\ge$ 500 & \thead[lt]{Largest improvements in precision \\ 
                                                      using \textbf{NM = 7 with \textit{N} $\ge$ 200} \vphantom{1} or \\
                                                      \textbf{NM = 9 with \textit{N} $\le$ 100}} & 10.25\\
\thead[lt]{$\upbeta_{random}$ \\ (Figure \ref{fig:exp3_plot_days_fast}C)} & All cells & No cells & Largest improvements in precision with NM = 7 & 17.47\\
\thead[lt]{$\upgamma_{random}$ \\ (Figure \ref{fig:exp3_plot_days_fast}D)} & \thead[lt]{
                                            \textbf{NM $\ge$ 7 with \textit{N} = 1000} or \\
                                            \textbf{NM $\ge$ 9 with \textit{N} $\ge$ 200} or \\
                                            \textbf{NM = 11 with \textit{N} = 100}} & No cells & \thead[lt]{Largest improvements in precision \\ 
                                                      using \textbf{NM = 7 with \textit{N} $\ge$ 200} or \\
                                                      \textbf{NM = 9 with \textit{N} $\le$ 100}} & 10.51\\
\bottomrule
\insertTableNotes
\end{longtable}
\end{ThreePartTable}
\end{landscape}
\restoregeometry

\hypertarget{bias-fast-exp3}{%
\paragraph{Bias}\label{bias-fast-exp3}}

With respect to bias for time-unstructured data characterized by a fast response rate, estimates are biased (i.e., above the acceptable 10\% cutoff) for each day-unit parameter in the following cells:
\begin{itemize}
\tightlist
\item
  fixed-effect days-to-halfway elevation parameter (\(\upbeta_{fixed}\); Figure \ref{fig:exp3_plot_days_fast}A): no cells.
\item
  fixed-effect halfway-triquarter delta parameter (\(\upgamma_{fixed}\); Figure \ref{fig:exp3_plot_days_fast}B): no cells.
\item
  random-effect days-to-halfway elevation parameter (\(\upbeta_{random}\); Figure \ref{fig:exp3_plot_days_fast}C): no cells.
\item
  random-effect triquarter-halfway elevation parameter (\(\upgamma_{random}\); Figure \ref{fig:exp3_plot_days_fast}D): five measurements across all sample sizes, seven measurements with \(N \le 500\), nine measurements with \(N \ge 100\), and 11 measurements with \(N \le 50\).
\end{itemize}
\noindent Note that, for the fixed-effect days-to-halfway elevation parameter (\(\upbeta_{fixed}\)), although bias is still within the acceptable margin of error, bias appears to be constant across all manipulated measurement number-sample size pairings.

In summary, with time-unstructured data characterized by a fast response rate, estimation of all the day-unit parameters across all manipulated nature-of-change values is unbiased using at least seven measurements with \(N = 1000\), nine measurements with \(N \ge 200\), or 11 measurements with \(N \ge 100\), which is indicated by the emboldened text in the `Unbiased' column of Table \ref{tab:summary-table-fast-exp3}.
\begin{apaFigure}
[portrait]
[samepage]
[-0.2cm]
{Parameter Estimation Plots for Day-Unit Parameters With Time-Unstructured Data Characterized by a Fast Response Rate in Experiment 3}
{exp3_plot_days_fast}
{0.165}
{Figures/exp3_plot_days_time unstructured (fast response)}
{Panel A: Parameter estimation plot for the fixed-effect days-to-halfway elevation parameter ($\upbeta_{fixed}$). Panel B: Parameter estimation plot for the fixed-effect triquarter-halfway elevation parameter ($\upgamma_{fixed}$). Panel C: Parameter estimation plot for the random-effect days-to-halfway elevation parameter ($\upbeta_{random}$). Panel D: Parameter estimation plot for the random-effect triquarter-halfway elevation parameter ($\upgamma_{random}$). Blue horizontal lines in each panel represent the population value for each parameter. Population values for each day-unit parameter are as follows: $\upbeta_{fixed}$ = 180.00, $\upbeta_{random}$ = 10.00, $\upgamma_{fixed}$ = 20.00, $\upgamma_{random}$ = 4.00. Gray bands indicate the $\pm 10\%$ margin of error for each parameter and unfilled dots indicate cells with average parameter estimates outside of the margin or biased estimates. Error bars represent the middle 95\% of estimated values, with light blue error bars indicating imprecise estimation. I considered dots that fell outside the gray bands as biased and error bar lengths with at least one whisker length exceeding the 10\% cutoff (i.e., or longer than the portion of the gray band underlying the whisker) as imprecise. Note that random-effect parameter units are in standard deviation units. See Table \ref{tab:param-exp-3} for specific values estimated for each parameter and Table \ref{tab:omega-exp3-fast} for $\upomega^2$ effect size values.}
\end{apaFigure}
\begin{ThreePartTable}
\begin{TableNotes}
\item NM = number of measurements (5, 7, 9, 11), S = sample size (30, 50, 100, 200, 500, 100), NM x S = interaction between number of measurements and sample size.
\end{TableNotes}
\begin{longtable}[l]{>{\raggedright\arraybackslash}p{6cm}ccc}
\caption{\label{tab:omega-exp3-fast}Partial $\upomega^2$ Values for Manipulated Variables With Time-Structured Data in Experiment 3}\\
\toprule
\multicolumn{1}{c}{ } & \multicolumn{3}{c}{Effect} \\
\cmidrule(l{3pt}r{3pt}){2-4}
Parameter & NM & S & NM x S\\
\midrule
$\upbeta_{fixed}$ (Figure \ref{fig:exp3_plot_days_fast}A) & 0.00 & 0.02 & 0.00\\
$\upbeta_{random}$ (Figure \ref{fig:exp3_plot_days_fast}B) & 0.15 & 0.27 & 0.03\\
$\upgamma_{fixed}$ (Figure \ref{fig:exp3_plot_days_fast}C) & 0.29 & 0.14 & 0.08\\
$\upgamma_{random}$ (Figure \ref{fig:exp3_plot_days_fast}D) & 0.17 & 0.04 & 0.01\\
\bottomrule
\insertTableNotes
\end{longtable}
\end{ThreePartTable}
\hypertarget{precision-fast-exp3}{%
\paragraph{Precision}\label{precision-fast-exp3}}

With respect to precision for time-unstructured data characterized by a fast response rate, estimates are imprecise (i.e., error bar length with at least one whisker length exceeding 10\% of a parameter's population value) in the following cells for each day-unit parameter:
\begin{itemize}
\tightlist
\item
  fixed-effect days-to-halfway elevation parameter (\(\upbeta_{fixed}\); Figure \ref{fig:exp3_plot_days_fast}A): no cells.
\item
  fixed-effect halfway-triquarter delta parameter (\(\upgamma_{fixed}\); Figure \ref{fig:exp3_plot_days_fast}B): five and seven measurements across all sample sizes and nine and 11 measurements with \(N \le 200\).
\item
  random-effect days-to-halfway elevation parameter (\(\upbeta_{random}\); Figure \ref{fig:exp3_plot_days_fast}C): all cells.
\item
  random-effect halfway-triquarter delta parameter {[}\(\upgamma_{random}\){]} in Figure \ref{fig:exp3_plot_days_fast}D): all cells.
\end{itemize}
In summary, with time-unstructured data characterized by a fast response rate, precise estimation can be obtained for the fixed-effect day-unit parameters using at least nine measurements with \(N \ge 500\), but no manipulated measurement number-sample size pairing results in precise estimation of the random-effect day-unit parameters (see the `Precise' column of Table \ref{tab:summary-table-fast-exp3}).

\hypertarget{qualitative-fast-exp3}{%
\paragraph{Qualitative Description}\label{qualitative-fast-exp3}}

For time-unstructured data characterized by a fast response rate (see Figure \ref{fig:exp3_plot_days_fast}), although no manipulated measurement number results in precise estimation of all the day-unit parameters, the largest improvements in precision (and bias) result from using moderate measurement number-sample size pairings. With respect to bias under time-unstructured data characterized by a fast response rate, the largest improvements in bias result with the following measurement number-sample size pairings for the random-effect triquarter-halfway delta parameter (\(\upgamma_{fixed}\)):
\begin{itemize}
\tightlist
\item
  random-effect triquarter-halfway delta parameter (\(\upgamma_{random}\)): seven measurements with \(N \ge 100\) or nine measurements with \(N \le 50\).
\end{itemize}
\noindent With respect to precision under time-unstructured data characterized by a fast response rate, the largest improvements in precision for the estimation of all the day-unit parameters (except the fixed-effect days-to-halfway elevation parameter {[}\(\upbeta_{fixed}\){]}) result from using the following measurement number-sample size pairings:
\begin{itemize}
\tightlist
\item
  fixed-effect triquarter-halfway delta parameter (\(\upgamma_{fixed}\)): seven measurements with \(N \ge 200\) or nine measurements with \(N \le 100\), which results in a maximum error bar length of 10.25 days.
\item
  random-effect days-to-halfway elevation parameter (\(\upbeta_{random}\)): seven measurements across all manipulated sample sizes, which which results in a error bar length of 17.47 days.
\item
  random-effect triquarter-halfway delta parameter (\(\upgamma_{random}\)): seven measurements with \(N \ge 200\) or nine measurements with \(N \le 100\), which results in a maximum error bar length of 10.51 days.
\end{itemize}
For an applied researcher, one plausible question might be what measurement number-sample size pairing(s) results in the greatest improvements in bias and precision in the estimation of all day-unit parameters with time-unstructured data characterized by a fast response rate. In looking across the measurement number-sample size pairings in the above lists, it becomes apparent that greatest improvements in bias and precision in the estimation of all day-unit parameters result with the following measurement number-sample size pairing(s): seven measurements with \(N \ge 200\) or nine measurements with \(N \le 100\) (see the emboldened text in the `Qualitative Description' column of Table \ref{tab:summary-table-fast-exp3}).

\hypertarget{summary-of-results-9}{%
\subsubsection{Summary of Results}\label{summary-of-results-9}}

In summarizing the results for time-unstructured data characterized by a fast response rate, estimation of all day-unit parameters is unbiased using least seven measurements with \(N = 1000\), nine measurements with \(N \ge 200\), or 11 measurements with \(N \ge 100\) (see \protect\hyperlink{bias-fast-exp3}{bias}). Importantly, bias for some day-unit parameters is constant across manipulated measurement number-sample size pairings. Precise estimation is never obtained in the estimation of all day-unit parameters with any manipulated measurement number-sample size pairing (see \protect\hyperlink{precision-fast-exp3}{precision}). Although it may be discouraging that no manipulated measurement number-sample size pairing under time-unstructured data characterized by a fast response rate results in precise estimation of all day-unit parameters, the largest improvements in precision (and bias) across all day-unit parameters are obtained with moderate measurement number-sample size pairings. With time-unstructured data characterized by a fast response rate, the largest improvements in bias and precision in the estimation of all day-unit parameters are obtained using seven measurements with \(N \ge 200\) or nine measurements with \(N \le 100\) (see \protect\hyperlink{qualitative-fast-exp3}{qualitiative description}).

\hypertarget{time-unstructured-data-characterized-by-a-slow-response-rate}{%
\subsection{Time-Unstructured Data Characterized by a Slow Response Rate}\label{time-unstructured-data-characterized-by-a-slow-response-rate}}

For time-unstructured data characterized by a slow response rate, Table \ref{tab:summary-table-slow-exp3} provides a concise summary of the results for the day-unit parameters (see Figure \ref{fig:exp3_plot_days_slow} for the corresponding parameter estimation plots). The sections that follow will present the results for each column of Table \ref{tab:summary-table-slow-exp3} and provide elaboration when necessary (for a description of Table \ref{tab:summary-table-slow-exp3}, see \protect\hyperlink{concise-example-exp3}{concise summary}).

\newgeometry{margin=2.54cm}
\begin{landscape}
\begin{ThreePartTable}
\begin{TableNotes}
\item \textit{\textit{Note.}\hspace{-1.1pc}} 
\item Bolded text in the `Low Bias' and `Qualitative Summary' columns indicates the measurement number-sample size pairing needed to, respectively, achieve low bias and the greatest improvements in bias and precision across all day-unit parameters (high precision not achieved in the estimation of all day-unit parameters with time-unstructured data characterized by a slow response rate). `Error Bar Length' indicates the longest error bar length that results from using the measurement number-sample size pairings in the `Qualitative Summary` column. Parameter names and population values are as follows: $\upbeta_{fixed}$ = fixed-effect days-to-halfway elevation parameter = 180; $\upgamma_{fixed}$ = fixed-effect halfway-\newline triquarter delta parameter = 20; $\upbeta_{random}$ = random-effect days-to-halfway elevation parameter = 10; $\upgamma_{random}$ = random-effect halfway-triquarter delta parameter = 4.
\end{TableNotes}
\begin{longtable}[l]{>{\raggedright\arraybackslash}p{2cm}>{\raggedright\arraybackslash}p{5cm}>{\raggedright\arraybackslash}p{4cm}>{\raggedright\arraybackslash}p{6.5cm}>{\raggedright\arraybackslash}p{2.5cm}}
\caption{\label{tab:summary-table-slow-exp3}Concise Summary of Results for Time-Unstructured Data (Slow Response Rate) in Experiment 3}\\
\toprule
\multicolumn{3}{c}{ } & \multicolumn{2}{c}{Summary} \\
\cmidrule(l{3pt}r{3pt}){4-5}
Parameter & Unbiased & Precise & Qualitative Summary & Error Bar Length\\
\midrule
\thead[lt]{$\upbeta_{fixed}$ \\ (Figure \ref{fig:exp3_plot_days_slow}A)} & All cells & All cells & Low bias and high precision in all cells & 16.68\\
\thead[lt]{$\gamma_{fixed}$ \\ (Figure \ref{fig:exp3_plot_days_slow}B)} & All cells except NM = 5 with \textit{N} = 50 & \thead[lt]{NM = 7 with \textit{N} = 200 or \\ 
                                            NM = 9 with \textit{N} $\le$ 500} & \thead[lt]{Largest improvements in precision \\ 
                                                      using \textbf{NM = 7 with \textit{N} $\ge$ 200} or \\
                                                      \textbf{NM = 9 with \textit{N} $\le$ 100}} & 10.53\\
\thead[lt]{$\upbeta_{random}$ \\ (Figure \ref{fig:exp3_plot_days_slow}C)} & No cells except NM = 5 with \textit{N} = 30 and NM = 11 with \textit{N} $\le$ 50 & No cells & Largest improvements in precision with NM = 7 & 18.44\\
\thead[lt]{$\upgamma_{random}$ \\ (Figure \ref{fig:exp3_plot_days_slow}D)} & No cells & No cells & \thead[lt]{Largest improvements in bias and \\
                                                      precision using NM = 7 with \textit{N} $\boldsymbol{\ge}$ 200 or \\
                                                      M = 9 with \textit{N} $\boldsymbol{\le}$ 100} & 10.9\\
\bottomrule
\insertTableNotes
\end{longtable}
\end{ThreePartTable}
\end{landscape}
\restoregeometry

\hypertarget{bias-slow-exp3}{%
\paragraph{Bias}\label{bias-slow-exp3}}

With respect to bias for time-unstructured data characterized by a slow response rate, estimates are biased (i.e., above the acceptable 10\% cutoff) for each day-unit parameter in the following cells:
\begin{itemize}
\tightlist
\item
  fixed-effect days-to-halfway elevation parameter (\(\upbeta_{fixed}\); Figure \ref{fig:exp3_plot_days_slow}A): no cells.
\item
  fixed-effect halfway-triquarter delta parameter (\(\upgamma_{fixed}\); Figure \ref{fig:exp3_plot_days_slow}B): no cells.
\item
  random-effect days-to-halfway elevation parameter (\(\upbeta_{random}\); Figure \ref{fig:exp3_plot_days_slow}C): no cells.
\item
  random-effect triquarter-halfway elevation parameter (\(\upgamma_{random}\); Figure \ref{fig:exp3_plot_days_slow}D): five measurements across all sample sizes, seven measurements with \(N \le 500\), nine measurements with \(N \ge 100\), and 11 measurements with \(N \le 50\).
\end{itemize}
\noindent Note that, for all parameters except the halfway-triquarter delta parameter (\(\upgamma_{fixed}\)), bias appears to be constant across all manipulated measurement number-sample size pairings Liu et al. (2021)

In summary, with time-unstructured data characterized by a slow response rate, estimation of all the day-unit parameters across all manipulated nature-of-change values is unbiased using at least seven measurements with \(N = 1000\), nine measurements with \(N \ge 200\), or 11 measurements with \(N \ge 100\), which is indicated by the emboldened text in the `Unbiased' column of Table \ref{tab:summary-table-slow-exp3}.
\begin{apaFigure}
[portrait]
[samepage]
[-0.2cm]
{Parameter Estimation Plots for Day-Unit Parameters With Time-Unstructured Data Characterized by a Slow Response Rate in Experiment 3}
{exp3_plot_days_slow}
{0.165}
{Figures/exp3_plot_days_time unstructured (slow response)}
{Panel A: Parameter estimation plot for the fixed-effect days-to-halfway elevation parameter ($\upbeta_{fixed}$). Panel B: Parameter estimation plot for the fixed-effect triquarter-halfway elevation parameter ($\upgamma_{fixed}$). Panel C: Parameter estimation plot for the random-effect days-to-halfway elevation parameter ($\upbeta_{random}$). Panel D: Parameter estimation plot for the random-effect triquarter-halfway elevation parameter ($\upgamma_{random}$). Blue horizontal lines in each panel represent the population value for each parameter. Population values for each day-unit parameter are as follows: $\upbeta_{fixed}$ = 180.00, $\upbeta_{random}$ = 10.00, $\upgamma_{fixed}$ = 20.00, $\upgamma_{random}$ = 4.00. Gray bands indicate the $\pm 10\%$ margin of error for each parameter and unfilled dots indicate cells with average parameter estimates outside of the margin or biased estimates. Error bars represent the middle 95\% of estimated values, with light blue error bars indicating imprecise estimation. I considered dots that fell outside the gray bands as biased and error bar lengths with at least one whisker length exceeding the 10\% cutoff (i.e., or longer than the portion of the gray band underlying the whisker) as imprecise. Note that random-effect parameter units are in standard deviation units. See Table \ref{tab:param-exp-3} for specific values estimated for each parameter and Table \ref{tab:omega-exp3-slow} for $\upomega^2$ effect size values.}
\end{apaFigure}
\begin{ThreePartTable}
\begin{TableNotes}
\item NM = number of measurements (5, 7, 9, 11), S = sample size (30, 50, 100, 200, 500, 100), NM x S = interaction between number of measurements and sample size.
\end{TableNotes}
\begin{longtable}[l]{>{\raggedright\arraybackslash}p{6cm}ccc}
\caption{\label{tab:omega-exp3-slow}Partial $\upomega^2$ Values for Manipulated Variables With Time-Unstructured Data Characterized by a Slow Response Rate in Experiment 3}\\
\toprule
\multicolumn{1}{c}{ } & \multicolumn{3}{c}{Effect} \\
\cmidrule(l{3pt}r{3pt}){2-4}
Parameter & NM & S & NM x S\\
\midrule
$\upbeta_{fixed}$ (Figure \ref{fig:exp3_plot_days_slow}A) & 0.00 & 0.02 & 0.00\\
$\upbeta_{random}$ (Figure \ref{fig:exp3_plot_days_slow}B) & 0.15 & 0.27 & 0.03\\
$\upgamma_{fixed}$ (Figure \ref{fig:exp3_plot_days_slow}C) & 0.29 & 0.14 & 0.08\\
$\upgamma_{random}$ (Figure \ref{fig:exp3_plot_days_slow}D) & 0.17 & 0.04 & 0.01\\
\bottomrule
\insertTableNotes
\end{longtable}
\end{ThreePartTable}
\hypertarget{precision-slow-exp3}{%
\paragraph{Precision}\label{precision-slow-exp3}}

With respect to precision for time-unstructured data characterized by a slow response rate, estimates are imprecise (i.e., error bar length with at least one whisker length exceeding 10\% of a parameter's population value) in the following cells for each day-unit parameter:
\begin{itemize}
\tightlist
\item
  fixed-effect days-to-halfway elevation parameter (\(\upbeta_{fixed}\); Figure \ref{fig:exp3_plot_days_slow}A): no cells.
\item
  fixed-effect halfway-triquarter delta parameter (\(\upgamma_{fixed}\); Figure \ref{fig:exp3_plot_days_slow}B): five and seven measurements across all sample sizes and nine and 11 measurements with \(N \le 200\).
\item
  random-effect days-to-halfway elevation parameter (\(\upbeta_{random}\); Figure \ref{fig:exp3_plot_days_slow}C): all cells.
\item
  random-effect halfway-triquarter delta parameter {[}\(\upgamma_{random}\){]} in Figure \ref{fig:exp3_plot_days_slow}D): all cells.
\end{itemize}
In summary, with time-unstructured data characterized by a slow response rate, precise estimation can be obtained for the fixed-effect day-unit parameters using at least nine measurements with \(N \ge 500\), but no manipulated measurement number-sample size pairing results in precise estimation of the random-effect day-unit parameters (see the `Precise' column of Table \ref{tab:summary-table-slow-exp3}).

\hypertarget{qualitative-slow-exp3}{%
\paragraph{Qualitative Description}\label{qualitative-slow-exp3}}

For time-unstructured data characterized by a slow response rate (see Figure \ref{fig:exp3_plot_days_slow}), although no manipulated measurement number results in precise estimation of all the day-unit parameters, the largest improvements in precision (and bias) result from using moderate measurement number-sample size pairings. With respect to bias under time-unstructured data characterized by a slow response rate, the largest improvements in bias result with the following measurement number-sample size pairings for the random-effect triquarter-halfway delta parameter (\(\upgamma_{fixed}\)):
\begin{itemize}
\tightlist
\item
  random-effect triquarter-halfway delta parameter (\(\upgamma_{random}\)): seven measurements with \(N \ge 100\) or nine measurements with \(N \le 50\).
\end{itemize}
\noindent With respect to precision under time-unstructured data characterized by a slow response rate, the largest improvements in precision for the estimation of all the day-unit parameters (except the fixed-effect days-to-halfway elevation parameter {[}\(\upbeta_{fixed}\){]}) result from using the following measurement number-sample size pairings:
\begin{itemize}
\tightlist
\item
  fixed-effect triquarter-halfway delta parameter (\(\upgamma_{fixed}\)): seven measurements with \(N \ge 200\) or nine measurements with \(N \le 100\), which results in a maximum error bar length of 10.53 days.
\item
  random-effect days-to-halfway elevation parameter (\(\upbeta_{random}\)): seven measurements across all manipulated sample sizes, which which results in a error bar length of 18.44 days.
\item
  random-effect triquarter-halfway delta parameter (\(\upgamma_{random}\)): seven measurements with \(N \ge 200\) or nine measurements with \(N \le 100\), which results in a maximum error bar length of 10.9 days.
\end{itemize}
For an applied researcher, one plausible question might be what measurement number-sample size pairing(s) results in the greatest improvements in bias and precision in the estimation of all day-unit parameters with time-unstructured data characterized by a fast response rate. In looking across the measurement number-sample size pairings in the above lists, it becomes apparent that greatest improvements in bias and precision in the estimation of all day-unit parameters result with the following measurement number-sample size pairing(s): seven measurements with \(N \ge 200\) or nine measurements with \(N \le 100\) (see the emboldened text in the `Qualitative Description' column of Table \ref{tab:summary-table-slow-exp3}).

\hypertarget{summary-of-results-10}{%
\subsubsection{Summary of Results}\label{summary-of-results-10}}

In summarizing the results for time-unstructured data characterized by a slow response rate, estimation of all day-unit parameters is least seven measurements with \(N = 1000\), nine measurements with \(N \ge 200\), or 11 measurements with \(N \ge 100\) (see \protect\hyperlink{bias-slow-exp3}{bias}). Importantly, bias for most day-unit parameters is constant across manipulated measurement number-sample size pairings. Precise estimation is never obtained in the estimation of all day-unit parameters with any manipulated measurement number-sample size pairing (see \protect\hyperlink{precision-slow-exp3}{precision}). Although it may be discouraging that no manipulated measurement number-sample size pairing under time-unstructured data characterized by a slow response rate results in precise estimation of all day-unit parameters, the largest improvements in precision (and bias) across all day-unit parameters are obtained with moderate measurement number-sample size pairings. With time-unstructured data characterized by a slow response rate, the largest improvements in bias and precision in the estimation of all day-unit parameters are obtained using seven measurements with \(N \ge 200\) or nine measurements with \(N \le 100\) (see \protect\hyperlink{qualitative-slow-exp3}{qualitiative description}).

\hypertarget{how-does-time-structuredness-affect-modelling-accuracy}{%
\subsection{How Does Time Structuredness Affect Modelling Accuracy?}\label{how-does-time-structuredness-affect-modelling-accuracy}}

In Experiment 3, I was interested in how decreasing time structuredness affected modelling accuracy. Table \ref{tab:summary-table-exp3} summarizes the results for each spacing schedule in Experiment 3. Text within the `Unbiased' and `Precise' columns indicates the measurement number-sample size pairing needed to, respectively, obtain unbiased an precise estimation for all the day-unit parameters. The `Error Bar Length' column indicates longest error bar lengths that result in the estimation of each day-unit parameter from using the measurement number-sample size pairings listed in the `Qualitative Description' column. In looking at the `Qualitative Description' column, the greatest improvements in bias and precision for all time structuredness levels result from using either seven measurements with \(N \ge 200\) or nine measurements with \(N \le 100\).

Although the same measurement number-sample size pairing can be used to obtain the greatest improvements in modelling accuracy under any time structuredness level, two results suggest that modelling accuracy decreases as the time structuredness decreases. First, the error bar lengths in Table \ref{tab:summary-table-ex3} increase as time structuredness decreases. As an example, the error bar length of the fixed-effect days-to-halfway elevation parameter is 15.13 days with time-structured data and increases to 16.68 days with time-unstructured data characterized by a slow response rate. Second, and more alarming, the bias incurred as time structuredness decreases is constant across all measurement number-sample size pairings (see Figure \ref{exp3_plot_days_slow}). That is, the increase in bias that results from time-unstructured data cannot be reduced by increasing the number of measurements or sample size. An an example, the fixed-effect days-to-halfway elevation parameter is underestimated by roughly 6 days across all measurement number-sample size pairings (\(\upbeta_{fixed}\); see Figure \ref{exp3_plot_days_slow}A).

\newgeometry{margin=2.54cm}
\begin{landscape}
\begin{ThreePartTable}
\begin{TableNotes}
\item \textit{Note. }`Qualitative Description' column indicates the number of measurements that obtains the greatest improvements in bias and precision across all day-unit parameters. `Error Bar Summary' columns list the error bar lengths that result for each day-unit parameter using the measurement number listed in the `Qualitative Description' column. Parameter names and population values are as follows: $\upbeta_{fixed}$ = fixed-effect days-to-halfway elevation parameter $\in$ \{80, 180, 280\}; $\upgamma_{fixed}$ = fixed-effect halfway-triquarter delta parameter = 20; $\upbeta_{random}$ = random-effect days-to-halfway elevation parameter = 10; $\upgamma_{random}$ = random-effect halfway-triquarter delta parameter = 4. NM = number of measurements.
\end{TableNotes}
\begin{longtable}[l]{>{\raggedright\arraybackslash}p{5cm}>{\raggedright\arraybackslash}p{4.5cm}>{\raggedright\arraybackslash}p{2cm}>{\raggedright\arraybackslash}p{5.5cm}>{\centering\arraybackslash}p{1cm}>{\centering\arraybackslash}p{1cm}>{\centering\arraybackslash}p{1cm}>{\centering\arraybackslash}p{1cm}}
\caption{\label{tab:summary-table-exp3}Concise Summary of Results Across All Time Structuredness Levels in Experiment 3}\\
\toprule
\multicolumn{4}{c}{ } & \multicolumn{4}{c}{Error Bar Summary} \\
\cmidrule(l{3pt}r{3pt}){5-8}
Time Structuredness & Unbiased & Precise & Qualitative Description & $\upbeta_{fixed}$ & $\upgamma_{fixed}$ & $\upbeta_{random}$ & $\upgamma_{random}$\\
\midrule
\thead[lt]{Time structured \\ (see Figure \ref{fig:exp3_plot_days_time_struc} and Table \ref{tab:summary-table-time-struc-exp3})} & \thead[lt]{NM $\ge$ 9 with \textit{N} $\ge$ 200} & No cells & \thead[lt]{Largest improvements in precision \\
                                                          using \textbf{NM = 7 with \textit{N} $\ge$ 200} \vphantom{1} or \\
                                                          \textbf{NM = 9 with \textit{N} $\le$ 100}} & 15.13 & 9.79 & 17.22 & 10.08\\
\cmidrule{1-8}
Time unstructured (fast response rate; see Figure \ref{fig:exp3_plot_days_fast} and Table \ref{tab:summary-table-fast-exp3}) & \thead[lt]{NM $\ge$ 7 with \textit{N} = 1000 or \\
                                            NM $\ge$ 9 with \textit{N} $\ge$ 200 or \\
                                            NM = 11 with \textit{N} = 100} & No cells & \thead[lt]{Largest improvements in precision \\ 
                                                      using \textbf{NM = 7 with \textit{N} $\ge$ 200} or \\
                                                      \textbf{NM = 9 with \textit{N} $\le$ 100}} & 15.35 & 10.25 & 17.47 & 10.51\\
\cmidrule{1-8}
Time unstructured (slow response rate; see Figure \ref{fig:exp3_plot_days_slow} and Table \ref{tab:summary-table-slow-exp3}) & No cells & No cells & \thead[lt]{Largest improvements in precision \\ 
                                                      using \textbf{NM = 7 with \textit{N} $\ge$ 200} or \\
                                                      \textbf{NM = 9 with \textit{N} $\le$ 100}} & 16.68 & 10.53 & 18.44 & 10.90\\
\bottomrule
\insertTableNotes
\end{longtable}
\end{ThreePartTable}
\end{landscape}
\restoregeometry

To understand why bias is constant as time structure decreases, it is important to first understand latent growth curve models more deeply. By default, latent growth curve models assume time-structured data. As a reminder, data are time structured when participants provide data at the exact same moment at each time point (e.g., if a study collects data on the first day of each month for a year, then time-structured data would only be obtained if participants all provide their data at the exact same moment each time data are collected). Consider a random-intercept-random-slope model shown in Figure \ref{fig:ex-latent-growth} that is used to model stress ratings collected on the first day of each month over the course of five months from \(j\) people. Stress ratings at each \(i\) time point for each \(j\) person are predicted by person-specific intercepts (\(b_{0j}\)) and slopes (\(b_{1j}\); in addition to a residual term {[}\(\upepsilon_{ij}\){]}) as shown below in Equation \ref{eq:stressLevel1} (which is often called Level-1 equation):
\begin{align}
  Stress_{ij} = b_{0j} + b_{1j}(Stress_{ij}) + \upepsilon_{ij}.
  \label{eq:stressLevel1}
\end{align}
\noindent The person-specific intercepts and slopes are the sum of a fixed-effect parameter whose value is constant across all people (\(\upgamma_{00}\) and \(\upgamma_{10}\)) and a random-effect parameter that represents the variance of the person-specific variables (i.e., \(\upsigma_{00}\) and \(\upsigma_{10}\)). The fixed-effect intercept and slope, respectively, represent the mean starting stress value (i.e., average stress value at Time = 0) and the average slope value. Importantly, by estimating a random-effect parameter (in addition to the fixed-effect parameters), deviations from the mean intercept an slope values can be obtained for each \(j\) person (\(\upsigma_{0j}\) and \(\upsigma_{1j}\)) and these values then allow the person-specific intercepts and slopes to be computed as shown in Equations \ref{eq:intLevel2}--\ref{eq:slopeLevel2} (which are often called Level-2 equations):
\begin{align}
  b_{0j} = \hat{\upgamma_{00}} + \upsigma_{0j} \label{eq:intLevel2} \\
  b_{1j} = \hat{\upgamma_{10}} + \upsigma_{1j} \label{eq:slopeLevel2}
\end{align}
\noindent Note that the fixed- and random-effect parameters in Figure \ref{fig:latent-growth} are superscribed with a caret (\(\hat{\phantom{\beta}}\)) to indicate that the values of these parameters are estimated by the latent growth curve model. Also note that, in Figure \ref{fig:latent-growth}, circles indicate latent variables, triangles indicate constants, and squares indicate observed (or manifest variables).
\begin{apaFigure}
[portrait]
[samepage]
[-0.2cm]
{Path Diagram for a Random-Intercept-Random-Slope Latent Growth Curve Model}
{latent-growth}
{0.65}
{Figures/lgc_path_diagram}
{Stress at each $i$ time point for each $j$ person is predicted by a person-specific slope ($b_{0j}$), person-specific intercept ($b_{1j}$), and residual ($\upepsilon_{ij}$; see Equation \ref{eq:stressLevel1} [Level-1 equation]). The person-specific effects are also called \textit{random effects} and each is the sum of a fixed-effect parameter whose value is constant across all people ($\upgamma_{00}$ and $\upgamma_{10}$) and a random-effect parameter that represents the variance of the person-specific variables (i.e., $\upsigma_{00}$ and $\upsigma_{10}$; see Equations \ref{eq:intLevel2}--\ref{eq:slopeLevel2} [Level-2 equations]). Note that the fixed- and random-effect parameters are superscribed with a caret ($\hat{\phantom{\beta}}$) to indicate that the values of these parameters are estimated by the latent growth curve model. Also note that circles indicate latent variables, triangles indicate constants, and squares indicate observed (or manifest variables).}
\end{apaFigure}
To understand why bias in parameter estimation increases as time structuredness decreases, it is important to discuss one component of the latent growth curve model not yet discussed: loadings. In latent variable models, \emph{loadings} comprise numbers that indicate how a latent variable should be modelled. The numbers in loadings satisfy two needs of latent variables. First, loadings give latent variables a unit; latent variables are inherently unitless, and so require a unit so that they can be meaningfully interpreted. By fixing at least one pathway between a latent and observed variable with a loading, the latent variable takes on the units of the observed variable. In the current example, the intercept and slope latent variables take on the units of the stress ratings (e.g., Likert units). Second, in latent growth curve models, latent variables need their effect to be specified, and loadings satisfy this need. In the current example, the intercept has a constant effect at each time point, and this is represented by setting its loadings at each time point to 1. The slope represents linearly increasing change over time, and so its loadings are set to increase by an integer value of 1 after each time point.

Although loadings allow latent variables to model change over time, their values are constant across participants and it is this characteristic that causes modelling accuracy to decrease as time structuredness decreases. In focusing on the slope variable in Figure \ref{fig:latent-growth}, the loadings of 0, 1, 2, 3, and 4 assume that only one response pattern describes how each participant provides their data over the five-month period. Specifically, the loadings assume that each participant provides data on the first day of each month, which is indicated by the gray rectangles (along with the loading number above each gray rectangle) in each panel of Figure \ref{fig:time-structure}. With time-structured data, constant loadings do not decrease modelling accuracy because each participant provides their data on the first day of each month. As examples of modelling accuracy with time-structured data, panels A and C of Figure \ref{fig:time-structure} show the predicted and actual patterns for individual participants with linear and logistic patterns of change, respectively. Because each individual participant displays a response pattern identical to the one specified by the loadings, the predicted and actual patterns of change are identical. With time-unstructured data, the predicted and actual patterns of change no longer overlap because response patterns in participants differ from the one assumed by the loadings. As examples of modelling accuracy with time-unstructured data, panels B and D of Figure \ref{fig:time-structure} show the predicted and actual patterns for individual participants with linear and logistic patterns of change, respectively. Although each participant provides data many days after the first day of each month, the constant loadings set in the model lead the it to assume that data were collected on the first day of each month. Because the model misattributes the time at which data are recorded, the predicted patterns of change are shifted leftward, leading to a decrease in modelling accuracy. In Figure \ref{fig:time-structure}B, the intercept (\(b_{0j}\)) increases due to time-unstructured data. In Figure \ref{fig:time-structure}D, the fixed-effect days-to-halfway elevation parameter (\(\upbeta_{fixed}\)) decreases due to time-unstructured data. Therefore, the loading structured specified by default in latent growth curve model causes modelling accuracy to decrease when data are time unstructured.
\begin{apaFigure}
[portrait]
[samepage]
[-0.2cm]
{Modelling Accuracy Decreases as Time Structuredness Decreases}
{time_structure}
{0.165}
{Figures/time_structure_plots}
{Panel A: Predicted and actual linear patterns of change are identical because of time-structured data. Panel B: Predicted and actual linear patterns of change are different because of time-untructured data decreases modelling accuracy. Panel C: Predicted and actual logistic patterns of change are identical because of time-structured data. Panel D: Predicted and actual logistic patterns of change differ because of time-unstructured data decreases modelling accuracy. Shaded vertical rectangles indicate the response pattern expected across all participants by the loadings set in the latent growth curve model depicted in Figure \ref{fig:latent-growth}.}
\end{apaFigure}
\hypertarget{eliminating-the-bias-caused-by-time-unstructuredness-using-definition-variables}{%
\subsection{Eliminating the Bias Caused by Time Unstructuredness: Using Definition Variables}\label{eliminating-the-bias-caused-by-time-unstructuredness-using-definition-variables}}
\begin{apaFigure}
[portrait]
[samepage]
[-0.2cm]
{Parameter Estimation Plots for Day-Unit Parameters When Using Definition Variables To Model Time-Unstructured Data Characterized by a Slow Response Rate}
{exp3_plot_days_def}
{0.165}
{Figures/exp3_defplot_days_time unstructured (slow response)}
{Panel A: Parameter estimation plot for the fixed-effect days-to-halfway elevation parameter ($\upbeta_{fixed}$). Panel B: Parameter estimation plot for the fixed-effect triquarter-halfway elevation parameter ($\upgamma_{fixed}$). Panel C: Parameter estimation plot for the random-effect days-to-halfway elevation parameter ($\upbeta_{random}$). Panel D: Parameter estimation plot for the random-effect triquarter-halfway elevation parameter ($\upgamma_{random}$). Blue horizontal lines in each panel represent the population value for each parameter. Population values for each day-unit parameter are as follows: $\upbeta_{fixed}$ = 180.00, $\upbeta_{random}$ = 10.00, $\upgamma_{fixed}$ = 20.00, $\upgamma_{random}$ = 4.00. Gray bands indicate the $\pm 10\%$ margin of error for each parameter and unfilled dots indicate cells with average parameter estimates outside of the margin or biased estimates. Error bars represent the middle 95\% of estimated values, with light blue error bars indicating imprecise estimation. I considered dots that fell outside the gray bands as biased and error bar lengths with at least one whisker length exceeding the 10\% cutoff (i.e., or longer than the portion of the gray band underlying the whisker) as imprecise. Note that random-effect parameter units are in standard deviation units. See Table \ref{tab:param-exp-3-def} for specific values estimated for each parameter and Table \ref{tab:omega-exp3-def} for $\upomega^2$ effect size values.}
\end{apaFigure}
\hypertarget{summary-1}{%
\section{Summary}\label{summary-1}}

\newpage

\hypertarget{references}{%
\chapter{References}\label{references}}

\begingroup

\hypertarget{refs}{}
\begin{CSLReferences}{1}{0}
\leavevmode\vadjust pre{\hypertarget{ref-coulombe2016}{}}%
Coulombe, P., Selig, J. P., \& Delaney, H. D. (2016). Ignoring individual differences in times of assessment in growth curve modeling. \emph{International Journal of Behavioral Development}, \emph{40}(1), 76--86. \url{https://doi.org/10.1177/0165025415577684}

\leavevmode\vadjust pre{\hypertarget{ref-dillman2014}{}}%
Dillman, D. A., Smyth, J. D., \& Christian, L. M. (2014). \emph{Internet, phone, mail, and mixed-mode surveys: The tailored design method}. John Wiley \& Sons.

\leavevmode\vadjust pre{\hypertarget{ref-liu2022}{}}%
Liu, J., Perera, R. A., Kang, L., Sabo, R. T., \& Kirkpatrick, R. M. (2021). Obtaining interpretable parameters from reparameterized longitudinal models: Transformation Mmtrices between growth factors in Tto parameter spaces. \emph{Journal of Educational and Behavioral Statistics}, \emph{47}(2), 167--201. \url{https://doi.org/10.3102/10769986211052009}

\leavevmode\vadjust pre{\hypertarget{ref-pan2010}{}}%
Pan, B. (2010). Online travel surveys and response patterns. \emph{Journal of Travel Research}, \emph{49}(1), 121--135. \url{https://doi.org/10.1177/0047287509336467}

\end{CSLReferences}
\endgroup

\end{document}
