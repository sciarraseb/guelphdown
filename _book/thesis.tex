% This is the Reed College LaTeX thesis template. Most of the work
% for the document class was done by Sam Noble (SN), as well as this
% template. Later comments etc. by Ben Salzberg (BTS). Additional
% restructuring and APA support by Jess Youngberg (JY).
% Your comments and suggestions are more than welcome; please email
% them to cus@reed.edu
%
% See https://www.reed.edu/cis/help/LaTeX/index.html for help. There are a
% great bunch of help pages there, with notes on
% getting started, bibtex, etc. Go there and read it if you're not
% already familiar with LaTeX.
%
% Any line that starts with a percent symbol is a comment.
% They won't show up in the document, and are useful for notes
% to yourself and explaining commands.
% Commenting also removes a line from the document;
% very handy for troubleshooting problems. -BTS

%%
%% Preamble
\documentclass[
12pt, % The default document font size, options: 10pt, 11pt, 12pt
twoside,
english]{guelphthesis}
%----------------------------------------------------------------------------------------
% PACKAGES
%----------------------------------------------------------------------------------------

\usepackage{tocloft} %needed for table of contents, list of figures, list of tables, list of appendices
\usepackage{graphicx,latexsym}
\usepackage{amsmath}
\usepackage{amssymb,amsthm}

\usepackage{longtable,booktabs,setspace}
\usepackage{lmodern}
\usepackage{float}
\usepackage{etoolbox}
\floatplacement{figure}{H}
% Thanks, @Xyv
\usepackage{calc}
% End of CII addition
\usepackage{rotating}
\usepackage{tocbibind} %includes list of figures, list of tables, and table of contents in table of contents
\usepackage{indentfirst} %needed so that first paragraph after each section titles has indent
\usepackage{lineno} %allows option for line numbering
\usepackage{draftwatermark} %for draft watermark
\SetWatermarkText{} %ensures draft is not printed when draft:false
%\usepackage[backend=biber]{biblatex}


% Syntax highlighting #22



% To pass between YAML and LaTeX the dollar signs are added by CII
\title{Is Timing Everything? Measurement Timing and the Ability to Accurately Model Longitudinal Data}
\author{Sebastian L.V. Sciarra}
\year{2022}
\date{October, 2022}
\advisor{David Stanley}
\institution{University of Guelph}
\degree{Doctorate of Philosophy}



\department{Psychology}



  \let\cleardoublepage\clearpage

% From {rticles}
%


\urlstyle{rm}

%----------------------------------------------------------------------------------------
% CUSTOM COMMANDS
%----------------------------------------------------------------------------------------
%numbers lines before equations
%taken from https://tex.stackexchange.com/questions/43648/why-doesnt-lineno-number-a-paragraph-when-it-is-followed-by-an-align-equation
\newcommand*\patchAmsMathEnvironmentForLineno[1]{%
  \expandafter\let\csname old#1\expandafter\endcsname\csname #1\endcsname
  \expandafter\let\csname oldend#1\expandafter\endcsname\csname end#1\endcsname
  \renewenvironment{#1}%
     {\linenomath\csname old#1\endcsname}%
     {\csname oldend#1\endcsname\endlinenomath}}%
\newcommand*\patchBothAmsMathEnvironmentsForLineno[1]{%
  \patchAmsMathEnvironmentForLineno{#1}%
  \patchAmsMathEnvironmentForLineno{#1*}}%
\AtBeginDocument{%
\patchBothAmsMathEnvironmentsForLineno{equation}%
\patchBothAmsMathEnvironmentsForLineno{align}%
\patchBothAmsMathEnvironmentsForLineno{flalign}%
\patchBothAmsMathEnvironmentsForLineno{alignat}%
\patchBothAmsMathEnvironmentsForLineno{gather}%
\patchBothAmsMathEnvironmentsForLineno{multline}%
}


%nest all the \frontmatter functions in \oldfrontmatter, which allows us to redefine \frontmatter as everything it was with one modification to the
%draft watermark
\let\oldfrontmatter\frontmatter
%set page numbering to bottom center for \frontmatter
\fancypagestyle{frontmatter}{%
 \fancyhf{}% clear all header and footer fields
  \renewcommand{\headrulewidth}{0pt}
  \fancyhead[R]{\roman{page}}% Roman page number in footer centre

  }

\renewcommand{\frontmatter}{
  \oldfrontmatter
     \SetWatermarkLightness{0.8} %shading of draft watermark
  \SetWatermarkText{DRAFT}
  
   %set page number font to Arial if ArialFont: false in YAML header
  
   \pagestyle{frontmatter} % add this to center page numbers
}

%set page numbering to bottom center for \mainmatter
\fancypagestyle{mainmatter}{%
 \fancyhf{}% clear all header and footer fields
  \renewcommand{\headrulewidth}{0pt}
  \fancyfoot[C]{\arabic{page}}% Roman page number in footer centre

   \hypersetup{pdfpagemode={UseOutlines},
    bookmarksopen=true,
    hypertexnames=true,
    colorlinks = true,
    allcolors = blue,
    %linkcolor = blue,
    %urlcolor= blue,
    %anchorcolor = blue,
    pdfstartview={FitV},
    breaklinks=true,
    hyperindex = true,
    backref=page}

  \cleardoublepage

  


}

%nest all the \mainmatter functions in \oldmainmatter, which allows us to redefine \mainmatter as everything it was with one modification to the
%page numbering format
\newcommand{\setMainMatterLinespacing}{
 \setstretch{2} %default line spacing

  %change line spacing if specified in YAML header
        \setstretch{2}
  }

\let\oldmainmatter\mainmatter
\renewcommand{\mainmatter}{
  \oldmainmatter

  %change line spacing if specified in YAML header
  \setMainMatterLinespacing

      \linenumbers
  
  \pagestyle{mainmatter} % add this to center page numbers

}

%code below is important for linespacing to remain unaffected when kableExtra::landscape() is used andthe margin is specifically defined. Otherwise,
%linespacing for entire document goes to singlespacing for the text that follows the table.
\let\oldRestoreGeometry\restoregeometry
\renewcommand{\restoregeometry}{
  \oldRestoreGeometry

  %change line spacing if specified in YAML header
  \setMainMatterLinespacing
}

%change footnote and page number font to arial if desired

%----------------------------------------------------------------------------------------
%	TABLE OF CONTENTS, LIST OF FIGURES, & LIST OF TABLES
%----------------------------------------------------------------------------------------
%TABLE OF CONTENTS
\setlength{\cftbeforetoctitleskip}{0cm} %remove vertical space above table of contents

%two lines below ensure centered title for toc
%needed so that table of contents entry is not indented
\renewcommand{\contentsname}{Table of Contents} %change title for toc
\renewcommand{\cfttoctitlefont}{\hfill\fontsize{14}{14}\selectfont\bfseries\MakeUppercase}
\renewcommand{\cftaftertoctitle}{\hfill\hfill} %sometimes another \hfill is needed; depends on some setting in abovce code

%fonts for all entry level titles
\renewcommand\cftchapfont{\mdseries} %eliminate bolded chapter titles in toc
\renewcommand\cftsecfont{\mdseries} %eliminate bolded chapter titles in toc
\renewcommand\cftsubsecfont{\mdseries} %eliminate bolded chapter titles in toc
\renewcommand\cftsubsubsecfont{\mdseries} %eliminate bolded chapter titles in toc
\renewcommand\cftparafont{\mdseries} %eliminate bolded chapter titles in toc
\renewcommand\cftsubparafont{\mdseries} %eliminate bolded chapter titles in toc

%fonts for all entry level page numbers
\renewcommand{\cftchappagefont}{\mdseries} %remove bolding of page numbers for chapter headers in toc
\renewcommand\cftsecpagefont{\mdseries} %eliminate bolded chapter titles in toc
\renewcommand\cftsubsecpagefont{\mdseries} %eliminate bolded chapter titles in toc
\renewcommand\cftsubsubsecpagefont{\mdseries} %eliminate bolded chapter titles in toc
\renewcommand\cftparapagefont{\mdseries} %eliminate bolded chapter titles in toc
\renewcommand\cftsubparapagefont{\mdseries} %eliminate bolded chapter titles in toc

\renewcommand{\cftchapleader}{\cftdotfill{0.1}} %remove chapter bolding + modif dot spacing
\renewcommand{\cftdotsep}{0.1} %make dots in toc closer together

%spacing between toc items (should be all equal)
\setlength{\cftbeforechapskip}{0cm} %removes spacing before each chapter element
\renewcommand{\cftchapafterpnum}{\vskip6pt}
\renewcommand{\cftsecafterpnum}{\vskip6pt}
\renewcommand{\cftsubsecafterpnum}{\vskip6pt}
\renewcommand{\cftsubsubsecafterpnum}{\vskip6pt}
\renewcommand{\cftparaafterpnum}{\vskip6pt}
\renewcommand{\cftsubparaafterpnum}{\vskip6pt}

%remove header that appears in table of contents after first page
\renewcommand{\cftmarktoc}{}

%commands need to be redefined so that leading dots go all the way to the page numbers for all header levels (chap, sec, subsec, subsubsec, para, subpara
%%%general framework for commands below: cftXfillnum sets the format for the leading dots (\cftchapleader) and the page number (\cftchappagefont) such that leading dots proceed all the way to the page number with no spaces between dots and page number (\nobreak) at which wpoint paragraph mode ends (\par) and vertical spacing (defined  above) after item entry is inserted
%chapter (level 0)
\renewcommand{\cftchapfillnum}[1]{%
  {\cftchapleader}\nobreak
  {\cftchappagefont #1}\par\cftchapafterpnum
}

%sec (level 1)
\renewcommand{\cftsecfillnum}[1]{%
  {\cftsecleader}\nobreak
  {\cftsecpagefont #1}\par\cftsecafterpnum
}

%subsec (level 2)
\renewcommand{\cftsubsecfillnum}[1]{%
  {\cftsubsecleader}\nobreak
  {\cftsubsecpagefont #1}\par\cftsubsecafterpnum
}

%subsubsec (level 3)
\renewcommand{\cftsubsubsecfillnum}[1]{%
  {\cftsubsubsecleader}\nobreak
  {\cftsubsubsecpagefont #1}\par\cftsubsubsecafterpnum
}

%para (level 4)
\renewcommand{\cftparafillnum}[1]{%
  {\cftparaleader}\nobreak
  {\cftparapagefont #1}\par\cftparaafterpnum
}

%subpara (level 5)
\renewcommand{\cftsubparafillnum}[1]{%
  {\cftsubparaleader}\nobreak
  {\cftsubparapagefont #1}\par\cftsubparaafterpnum
}

%LIST OF TABLES
\renewcommand{\cfttabfont}{\mdseries} %set font for entries in lot
\renewcommand{\cfttabpagefont}{\mdseries} %set front for page numbers

\setlength{\cftbeforelottitleskip}{0cm} %remove vertical space above table of contents
\setlength{\cftafterlottitleskip}{0.5cm} %space between title for list of tables and list entries
%two lines below ensure centered title for toc
%needed so that table of contents entry is not indented
\renewcommand{\cftlottitlefont}{\hfill\fontsize{14}{14}\selectfont\bfseries\MakeUppercase}
\renewcommand{\cftafterlottitle}{\hfill} %sometimes another \hfill is needed; depends on some setting in abovce code

%commands need to be redefined so that leading dots go all the way to the page numbers for tables
%%%general framework for command below: cftfigfillnum sets the format for the leading dots (\cftfigleader) and the page number (\cftfigpagefont) such that leading dots proceed all the way to the page number with no spaces between dots and page number (\nobreak) at which point paragraph mode ends (\par) and vertical spacing (defined  below) after item entry is inserted
\setlength{\cftbeforetabskip}{0cm} %removes spacing before each chapter element
\renewcommand{\cfttabafterpnum}{\vskip6pt}

\renewcommand{\cfttabfillnum}[1]{%
  {\cfttableader}\nobreak
  {\cfttabpagefont #1}\par\cfttabafterpnum
}

%remove header that appears in list of tables after first page
\renewcommand{\cftmarklot}{}

%LIST OF FIGURES
\renewcommand{\cftfigfont}{\mdseries} %set font for entries in lof
\renewcommand{\cftfigpagefont}{\mdseries} %set front for page numbers

\setlength{\cftbeforeloftitleskip}{0cm} %remove vertical space above table of contents
\setlength{\cftafterloftitleskip}{0.5cm} %space between title for list of figures and list entries

%two lines below ensure centered title for toc
%needed so that table of contents entry is not indented
\renewcommand{\cftloftitlefont}{\hfill\fontsize{14}{14}\selectfont\bfseries\MakeUppercase}
\renewcommand{\cftafterloftitle}{\hfill} %sometimes another \hfill is needed; depends on some setting in abovce code

%commands need to be redefined so that leading dots go all the way to the page numbers for figures
%%%general framework for command below: cftfigfillnum sets the format for the leading dots (\cftfigleader) and the page number (\cftfigpagefont) such that leading dots proceed all the way to the page number with no spaces between dots and page number (\nobreak) at which wpoint paragraph mode ends (\par) and vertical spacing (defined  below) after item entry is inserted
\setlength{\cftbeforefigskip}{0cm} %removes spacing before each chapter element
\renewcommand{\cftfigafterpnum}{\vskip6pt}

\renewcommand{\cftfigfillnum}[1]{%
  {\cftfigleader}\nobreak
  {\cftfigpagefont #1}\par\cftfigafterpnum
}

%remove header that appears in list of figures after first page
\renewcommand{\cftmarklof}{}

%----------------------------------------------------------------------------------------
% LIST OF APPENDICES
%----------------------------------------------------------------------------------------
\newcommand{\listappname}{List of Appendices}
\newlistof[chapter]{app}{loa}{\listappname} %creates a new appendix counter that will be reset at the start of each \chapter

\setcounter{loadepth}{5} %loa will  go to depth of level 5
\setlength{\cftbeforeloatitleskip}{0cm} %remove vertical space above loa
\setlength{\cftafterloatitleskip}{0.5cm} %space between title for loa and list entries
\renewcommand{\cftmarkloa}{} %remove header titles

%two lines below ensure centered title for loa
%needed so that table of contents entry is not indented
\renewcommand{\cftloatitlefont}{\hfill\fontsize{14}{14}\selectfont\bfseries\MakeUppercase}
\renewcommand{\cftafterloatitle}{\hfill\hfill} %sometimes another \hfill is needed; depends on some setting in above code


%APPENDIX (level 0)
\renewcommand{\theapp}{\Alph{app}} %sets alphabetic counter for appendix
\renewcommand{\cftappfont}{\mdseries} %set font for level 0 entry in loa
\renewcommand{\cftapppagefont}{\mdseries} %set front for page numbers

\renewcommand{\cftapppresnum}{Appendix\space}
\renewcommand{\cftappaftersnum}{:\space}
\settowidth{\cftappnumwidth}{\cftapppresnum\theapp\cftappaftersnum\space}

\setlength{\cftbeforeappskip}{0cm} %removes vertical spacing before each chapter element
\renewcommand{\cftappafterpnum}{\vskip6pt}

%updates appendix counter, modifies chapter title such so that it is Appendix _letter_: #1
\newcommand{\app}[1]{%
  \refstepcounter{app}\pdfbookmark[-1]{\cftapppresnum\theapp\cftappaftersnum#1}{#1\theapp}%
  \chapter*{\fontsize{16}{16}\selectfont\bfseries\cftapppresnum\theapp\cftappaftersnum #1} %formats entry in document
  \addcontentsline{loa}{app}{{\cftapppresnum\theapp\cftappaftersnum}#1}%
  \par
}

% figure and table counting in appendix
\usepackage{chngcntr}


%leading dots for appendix (end immediately before page number)
\renewcommand{\cftappfillnum}[1]{%
 {\cftappleader}\nobreak{\cftapppagefont #1}\par\cftappafterpnum
}

%SECAPPENDIX (level 1; format A.1 : title)
\newlistentry[app]{secapp}{loa}{1}
\renewcommand{\thesecapp}{\theapp.\arabic{secapp}}
\renewcommand{\cftsecappfont}{\mdseries} %set font for level 1 entry in loa
\renewcommand{\cftsecapppagefont}{\mdseries} %set front for page numbers

\renewcommand{\cftsecapppresnum}{} %remove word 'Appendix'
\renewcommand{\cftsecappaftersnum}{\hspace{0.5cm}}  %replicate toc format for sub-level-0 headers \thesubappendix (i.e., A.1   title )

\setlength{\cftbeforesecappskip}{0cm} %removes vertical spacing before each chapter element
\renewcommand{\cftsecappafterpnum}{\vskip6pt}
\setlength{\cftsecappindent}{1.55em} %indentation in loa
\settowidth{\cftsecappnumwidth}{\cftsecapppresnum\thesecapp\cftsecappaftersnum\hspace{0.3cm}}

%updates appendix counter, modifies chapter title such so that it is Appendix _letter_: #1
\newcommand{\secapp}[1]{%
  \refstepcounter{secapp}\pdfbookmark[0]{#1}{#1\thesubapp}%
  \section*{\thesecapp\hspace{0.3cm} #1} %spacing between section number and title in text
  \addcontentsline{loa}{secapp}{{\thesecapp\cftsecappaftersnum}#1}%
  \par
}

%leading dots for appendix (end immediately before page number)
\renewcommand{\cftsecappfillnum}[1]{%
 {\cftsecappleader}\nobreak{\cftsecapppagefont #1}\par\cftsecappafterpnum
}


%SUBAPPENDIX (level 2; format A.1.1 : title)
\newlistentry[app]{subapp}{loa}{1}
\renewcommand{\thesubapp}{\thesecapp.\arabic{subapp}}
\renewcommand{\cftsubappfont}{\mdseries} %set font for level 2 entry in loa
\renewcommand{\cftsubapppagefont}{\mdseries} %set front for page numbers

\renewcommand{\cftsubapppresnum}{} %remove word 'Appendix'
\renewcommand{\cftsubappaftersnum}{\hspace{0.5cm}}  %replicate toc format for sub-level-0 headers \thesubappendix (i.e., A.1   title )

\setlength{\cftbeforesubappskip}{0cm} %removes vertical spacing before each chapter element
\renewcommand{\cftsubappafterpnum}{\vskip6pt}
\setlength{\cftsubappindent}{3.10em} %indentation in loa
%\renewcommand{\cftsubappnumwidth}{1.47cm}
\settowidth{\cftsubappnumwidth}{\thesubapp\cftsubappaftersnum\hspace{0.3cm}}

%updates appendix counter, modifies chapter title such so that it is Appendix _letter_: #1
\newcommand{\subapp}[1]{%
  \refstepcounter{subapp}\pdfbookmark[1]{#1}{#1\thesubapp}%
  \subsection*{\thesubapp\hspace{0.3cm} #1}%
  \addcontentsline{loa}{subapp}{{\thesubapp\cftsubappaftersnum}#1}%
  \par
}

%leading dots for appendix (end immediately before page number)
\renewcommand{\cftsubappfillnum}[1]{%
 {\cftsubappleader}\nobreak{\cftsubapppagefont #1}\par\cftsubappafterpnum
}


% SUBSUBAPPENDIX (level 3; format A.1.1.1  title)
\newlistentry[app]{subsubapp}{loa}{1}
\renewcommand{\thesubsubapp}{\thesubapp.\arabic{subsubapp}}
\renewcommand{\cftsubsubappfont}{\mdseries} %set font for level 3 entry in loa
\renewcommand{\cftsubsubapppagefont}{\mdseries} %set front for page numbers


\renewcommand{\cftsubsubapppresnum}{} %remove word 'Appendix'
\renewcommand{\cftsubsubappaftersnum}{\hspace{0.5cm}}  %space after subsubapp title

\setlength{\cftbeforesubsubappskip}{0cm} %removes vertical spacing before each chapter element
\renewcommand{\cftsubsubappafterpnum}{\vskip6pt}
\setlength{\cftsubsubappindent}{4.65em} %indentation in loa (1.55 *2)
\settowidth{\cftsubsubappnumwidth}{\thesubsubapp\cftsubsubappaftersnum\hspace{0.3cm}}

%updates appendix counter, modifies chapter title such so that it is Appendix _letter_: #1
\newcommand{\subsubapp}[1]{%
  \refstepcounter{subsubapp}\pdfbookmark[2]{#1}{#1\thesubsubapp}%
  \subsubsection*{\thesubsubapp\hspace{0.3cm} #1}%
  \addcontentsline{loa}{subsubapp}{{\thesubsubapp\cftsubsubappaftersnum}#1}%
  \par
}

%leading dots for appendix (end immediately before page number)
\renewcommand{\cftsubsubappfillnum}[1]{%
 {\cftsubsubappleader}\nobreak{\cftsubsubapppagefont #1}\par\cftsubsubappafterpnum
}

% PARA (level 4; format A.1.1.1.1  title)
\newlistentry[app]{paraapp}{loa}{1}
\renewcommand{\theparaapp}{\thesubsubapp.\arabic{paraapp}}
\renewcommand{\cftparaappfont}{\mdseries} %set font for level 4 entry in loa
\renewcommand{\cftparaapppagefont}{\mdseries} %set front for page numbers

\renewcommand{\cftparaapppresnum}{} %remove word 'Appendix'
\renewcommand{\cftparaappaftersnum}{\hspace{0.5cm}}  %space after paraapp title

\setlength{\cftbeforeparaappskip}{0cm} %removes vertical spacing before each chapter element
\renewcommand{\cftparaappafterpnum}{\vskip6pt}
\setlength{\cftparaappindent}{6.2em} %indentation in loa (1.55 *2)
\settowidth{\cftparaappnumwidth}{\theparaapp\cftparaappaftersnum\hspace{0.3cm}}

%updates appendix counter, modifies chapter title such so that it is Appendix _letter_: #1
\newcommand{\paraapp}[1]{%
  \refstepcounter{paraapp}\pdfbookmark[3]{#1}{#1\theparaapp}%
  \paragraph*{\theparaapp\hspace{0.3cm} #1}%
  \addcontentsline{loa}{paraapp}{{\theparaapp\cftparaappaftersnum}#1}%
  \par
}

%leading dots for appendix (end immediately before page number)
\renewcommand{\cftparaappfillnum}[1]{%
 {\cftparaappleader}\nobreak{\cftparaapppagefont #1}\par\cftparaappafterpnum
}

% SUBPARA (level 5; format A.1.1.1.1  title)
\newlistentry[app]{subparaapp}{loa}{1}
\renewcommand{\thesubparaapp}{\theparaapp.\arabic{subparaapp}}
\renewcommand{\cftsubparaappfont}{\mdseries} %set font for level 5 entry in loa
\renewcommand{\cftsubparaapppagefont}{\mdseries} %set front for page numbers

\renewcommand{\cftsubparaapppresnum}{} %remove word 'Appendix'
\renewcommand{\cftsubparaappaftersnum}{\hspace{0.5cm}}  %space after subparaapp title

\setlength{\cftbeforesubparaappskip}{0cm} %removes vertical spacing before each chapter element
\renewcommand{\cftsubparaappafterpnum}{\vskip6pt}
\setlength{\cftsubparaappindent}{7.75em} %indentation in loa (1.55 *2)
\settowidth{\cftsubparaappnumwidth}{\thesubparaapp\cftsubparaappaftersnum\hspace{0.3cm}}

%updates appendix counter, modifies chapter title such so that it is Appendix _letter_: #1
\newcommand{\subparaapp}[1]{%
  \refstepcounter{subparaapp}\pdfbookmark[4]{#1}{#1\thesubparaapp}%
  \paragraph*{\thesubparaapp\hspace{0.3cm} #1} %paragraph is used because subparagraph has weird numbering problem
  \addcontentsline{loa}{subparaapp}{{\thesubparaapp\cftsubparaappaftersnum}#1}%
  \par
}

%SUBSUBPARA (level 6; format A.1.1.1.1.1  title)
\newlistentry[app]{subsubparaapp}{loa}{1}
\renewcommand{\thesubsubparaapp}{\thesubparaapp.\arabic{subsubparaapp}}

\renewcommand{\cftsubsubparaapppresnum}{} %remove word 'Appendix'
\renewcommand{\cftsubsubparaappaftersnum}{\hspace{0.5cm}}  %space after subparaapp title

\setlength{\cftbeforesubsubparaappskip}{0cm} %removes vertical spacing before each chapter element
\renewcommand{\cftsubsubparaappafterpnum}{\vskip6pt}
\setlength{\cftsubsubparaappindent}{9.3em} %indentation in loa (1.55 *2)
\settowidth{\cftsubsubparaappnumwidth}{\thesubsubparaapp\cftsubsubparaappaftersnum\hspace{0.3cm}}

%updates appendix counter, modifies chapter title such so that it is Appendix _letter_: #1
\newcommand{\subsubparaapp}[1]{%
  \refstepcounter{subsubparaapp}\pdfbookmark[5]{#1}{#1\thesubsubparaapp}%
  \subparagraph*{\thesubsubparaapp\hspace{0.3cm} #1} %paragraph is used because subparagraph has weird numbering problem
  \addcontentsline{loa}{subsubparaapp}{{\thesubsubparaapp\cftsubsubparaappaftersnum}#1}%
  \par
}

%leading dots for appendix (end immediately before page number)
\renewcommand{\cftsubsubparaappfillnum}[1]{%
 {\cftsubsubparaappleader}\nobreak{\cftsubsubparaapppagefont #1}\par\cftsubsubparaappafterpnum
}

\newcommand{\listabbname}{List of Abbreviations}
\newlistof[chapter]{abb}{loab}{\listabbname} %creates a new appendix counter that will be reset at the start of each \chapter

\setlength{\cftbeforeloabtitleskip}{0cm} %remove vertical space above loab
\setlength{\cftafterloabtitleskip}{0.2cm} %space between title for loab and list entries

\renewcommand{\cftmarkloab}{} %remove header titles

%two lines below ensure centered title for loa
%needed so that table of contents entry is not indented
\renewcommand{\cftloabtitlefont}{\hfill\fontsize{14}{14}\selectfont\bfseries\MakeUppercase}
\renewcommand{\cftafterloabtitle}{\hfill\hfill} %sometimes another \hfill is needed; depends on some setting in above code



%----------------------------------------------------------------------------------------
% REFERENCES & HYPERLINKING
%----------------------------------------------------------------------------------------

\usepackage{hyperref}

\PassOptionsToPackage{backref=true}{biblatex}




\RequirePackage[autocite=inline, style = apa]{biblatex}
\addbibresource{bib/references.bib}
%\addbibresource{bib/references.bib}
\DeclareSourcemap{\maps[datatype = bibtex]{\map{\step[fieldsource = journal, match = \regexp{\x{26}}, replace = \regexp{\{\\\x{26}\}}] }}}

\hypersetup{pdfpagemode={UseOutlines},
    bookmarksopen=true,
    backref=page}
\usepackage{hypernat}
%%adds escape character to ampersand characters in journal fields of .bib file
\DefineBibliographyStrings{english}{backrefpage={cited on p.},backrefpages={cited on pp.}}




\hypersetup{pdfpagemode={UseOutlines},
bookmarksopen=true, %allows bookmarks in pdf
hypertexnames=true, %enables counting when referencing to sections
colorlinks = true, % Set to true to enable coloring links, a nice option, false to turn them off
%citecolor = blue, % The color of citations
%linkcolor = blue, % The color of references to document elements (sections, figures, etc)
%urlcolor= blue,
%anchorcolor = blue, % The color of hyperlinks (URLs)
allcolors = blue,
pdfstartview={FitV},
breaklinks=true, backref=page
}


%example numbering
\newtheorem{theorem}{Theorem}[section]
\renewcommand{\thetheorem}{\theapp.\arabic{theorem}}
\newtheorem{example}{Example}
\renewcommand{\theexample}{\theapp.\arabic{example}}


%load additional latex packages needed within document
	\usepackage{booktabs}
\usepackage{longtable}
\usepackage{array}
\usepackage{multirow}
\usepackage{wrapfig}
\usepackage{float}
\usepackage{colortbl}
\usepackage{pdflscape}
\usepackage{tabu}
\usepackage{threeparttable}
\usepackage{threeparttablex}
\usepackage[normalem]{ulem}
\usepackage{makecell}
\usepackage{xcolor}

%----------------------------------------------------------------------------------------
% DOCUMENT OUTLINE
%----------------------------------------------------------------------------------------

% BEGIN DOCUMENT
\begin{document}
\frontmatter %pages will be numbered with roman numerals

  \maketitle

\setcounter{page}{2} %ensures abstract page number starts at roman numberal ii

\cleardoublepage
\thispagestyle{empty} %removes page number only for abstract page
  \begin{abstract}{2}{The preface pretty much says it all. This is additional content. The preface pretty much says it all. This is additional content. The preface pretty much says it all. This is additional content. The preface pretty much says it all. This is additional content. The preface pretty much says it all. This is additional content.}  %[linespacing][abstract][

  \end{abstract}

% notice how yaml variables are indexed with dollar signs and then passed into second argument of preambleItem environments
  \cleardoublepage
  \begin{preambleItem}{2}{Dedication}{You can have a dedication here if you wish. You can have a dedication here if you wish.You can have a dedication here if you wish.You can have a dedication here if you wish.You can have a dedication here if you wish.You can have a dedication here if you wish.You can have a dedication here if you wish.}
  \end{preambleItem}
  \cleardoublepage
   \begin{preambleItem}{2}{Acknowledgements}{I want to thank a few people.You can have a dedication here if you wish. You can have a dedication here if you wish.You can have a dedication here if you wish.You can have a dedication here if you wish.You can have a dedication here if you wish.You can have a dedication here if you wish.You can have a dedication here if you wish. I want to thank a few people.You can have a dedication here if you wish. You can have a dedication here if you wish.You can have a dedication here if you wish.You can have a dedication here if you wish.You can have a dedication here if you wish.You can have a dedication here if you wish.You can have a dedication here if you wish. I want to thank a few people.You can have a ded}
  \end{preambleItem}


%move page numbers to top right for list of tables, figures, and tables
\fancypagestyle{plain}{%
  \fancyhf{}% clear all header and footer fields
  \renewcommand{\headrulewidth}{0pt}
  \fancyhead[R]{\thepage}

   }

%table of contents
  \cleardoublepage
  \hypersetup{linkcolor = black, pdfborder= 0 0 0} %remove red borders around toc items
  \setcounter{secnumdepth}{5}
  \setcounter{tocdepth}{5}
  \tableofcontents
  \newpage

%list of tables
  \cleardoublepage
  \listoftables
  \newpage

%list of figures
  \cleardoublepage
  \listoffigures
  \newpage


%list of appendices
  \cleardoublepage
  \phantomsection
  \addcontentsline{toc}{chapter}{\listappname}
  \listofapp

  \newpage

\mainmatter % here the regular arabic numbering starts

\hypertarget{introduction}{%
\chapter{Introduction}\label{introduction}}
\begin{quote}
    ``Neither the behavior of human beings nor the activities of organizations can be defined without reference to time, and temporal aspects are critical for understanding them" \parencite[][p. 136]{navarro2015}.
\end{quote}
The topic of time has received considerable attention in organizational psychology over the past 20 years. Examples of well-received articles published around the beginning of the 21\textsuperscript{st} century discuss how investigating time is important for
understanding patterns of change and boundary conditions of theory
\autocite{zaheer1999}, how longitudinal research is necessary for disentangling
different types of causality \autocite{mitchell2001}, and explicate a pattern
of organizational change \autocite[or institutionalization;][]{lawrence2001}.
Since then, articles have emphasized the need to address time in
specific areas such as performance \autocite{fisher2008,dalal2014}, teams \autocite{roe2012}, and goal setting \autocite{fried2004} and, more generally, throughout organizational research \autocite{george2000,roe2008,ployhart2010,sonnentag2012,navarro2015,shipp2015,kunisch2017,vantilborgh2018,aguinis2021}.

The importance of time has also been recognized in organizational theory. In defining a theoretical contribution, \textcite{whetten1989} discussed that time must be discussed in regard to setting boundary conditions (i.e., under what circumstances does the theory apply) and in specifying relations between variables over time \autocite{mitchell2001,george2000}. Even if a considerable number of organizational theories do not adhere to the definition of \textcite{whetten1989}, theoretical models in organizational psychology consist of path diagrams that delineate the causal underpinnings of a process. Given that temporal precedence is a necessary condition for establishing causality \autocite{mill2011}, time has a role, whether implicitly or explicitly, in organizational theory.

Despite the considerable attention given towards investigating processes over time and its ubiquity in organizational theory, the prevalence of longitudinal research has historically remained low. One study examined the prevalence of longitudinal research from 1970--2006 across five organizational psychology journals and found that 4\% of articles used longitudinal designs (Roe, 2014). Another survey of two applied psychology journals in 2005 found that approximately 10\% (10 of 105 studies) of studies used longitudinal designs \autocite{roe2008}. Similarly, two surveys of studies employing longitudinal designs with mediation analysis found that, across five journals, only about 10\% (7 of 72 studies) did so in 2005 \autocite{maxwell2007} and approximately 16\% (15 of 92 studies) did so in 2006 \autocite{mitchell2013}.\footnote{Note that the definition of a longitudinal design in \textcite{maxwell2007} and \textcite{mitchell2013} required that measurements be taken over at least three time points so that measurements of the predictor, mediator, and outcome variables were separated over time.} Thus, the prevalence of longitudinal research has remained low.

In the seven sections that follow, I will explain why longitudinal research is necessary and the factors that must be considered when conducting such research. In the first section, I will explain why conducting longitudinal research is essential for understanding the dynamics of psychological processes. In the second section, I will overview patterns of change that are likely to emerge over time. In the third section, I will overview design and analytical issues involved in designing longitudinal studies. In the fourth section, I will explain how design and analytical issues encountered in conducting longitudinal research can be investigated. In the fifth section, I will provide a systematic review of the research that has investigated design and analytical issues involved in conducting longitudinal research. Finally, in the sixth and seventh sections, I will, respectively, discuss some methods for modelling nonlinear change and the frameworks in which they can be used. A summary of the three simulation experiments that I conducted in my dissertation will then be provided.

\hypertarget{the-need-to-conduct-longitudinal-research}{%
\section{The Need to Conduct Longitudinal Research}\label{the-need-to-conduct-longitudinal-research}}

Longitudinal research provides substantial advantages over cross-sectional research. Unfortunately, researchers commonly discuss the results of cross-sectional analyses as if they have been obtained with a longitudinal design. However, cross-sectional and longitudinal analyses often produce different results. One example of the assumption that cross-sectional findings are equivalent to longitudinal findings comes from the large number of studies employing mediation analysis. Given that mediation is used to understand chains of causality in psychological processes \autocite{baron1986}, it would thus make sense to pair mediation analysis with a longitudinal design because understanding causality, after all, requires temporal precedence. Unfortunately, the majority of studies that have used mediation analysis have done so using cross-sectional designs---with estimates of approximately 90\% \autocite{maxwell2007} and 84\% \autocite{mitchell2013}---and have often discussed the results as if they were longitudinal. Investigations into whether mediation results remain equivalent across cross-sectional and longitudinal designs have repeatedly concluded that using mediation analysis on cross-sectional data can return different, and sometimes completely opposite, results from using it on longitudinal data \autocite{cole2003,maxwell2007,maxwell2011,mitchell2013,olaughlin2018}. Therefore, mediation analyses based on cross-sectional analyses may be misleading.

The non-equivalence of cross-sectional and longitudinal results that occurs with mediation analysis is, unfortunately, not due to a specific set of circumstances that only arise with mediation analysis, but a consequence of a broader systematic cause that affects the results of many analyses. The concept of ergodicity explains why cross-sectional and longitudinal analyses seldom yield similar results. To understand ergodicity, it is first important to realize that variance is central to many statistical analyses---correlation, regression, factor analysis, and mediation are some examples. Thus, if variance remains unchanged across cross-sectional and longitudinal data sets, then analyses of either data set would return the same results. Importantly, variance only remains equal across cross-sectional and longitudinal data sets if two conditions put forth by ergodic theory are satisfied \autocites[homogeneity and stationarity;][]{molenaar2004,molenaar2009}. If these two conditions are met, then a process is said to be ergodic. Unfortunately, the two conditions required for ergodicity are highly unlikely to be satisfied and so cross-sectional findings will frequently deviate from longitudinal findings (for a detailed discussion, see Appendix \ref{ergodicity}).

Given that cross-sectional and longitudinal analyses are, in general, unlikely to return equivalent findings, it is unsurprising that several investigations in organizational research---and psychology as a whole---have found these analyses to return different results. Beginning with an example from \textcite{curran2011}, heart attacks are less likely to occur in people who exercise regularly (longitudinal finding), but more likely to happen when exercising (cross-sectional finding). Correlational studies find differences in correlation magnitudes between cross-sectional and longitudinal data sets \autocites[for a meta-analytic review, see][]{nixon2011,fisher2018}.\footnote{Note that \textcite{fisher2018} also found the variability of longitudinal correlations to be considerably larger than the variability of cross-sectional correlations.} Moving on to perhaps the most commonly employed analysis in organizational research of mediation, several articles have highlighted cross-sectional data can return different, and sometimes completely opposite, results to longitudinal data \autocite{cole2003,maxwell2007,maxwell2011,olaughlin2018}. Factor analysis is perhaps the most interesting example: The well-documented five-factor model of personality seldom arises when analyzing person-level data that was obtained by measuring personality on 90 consecutive days \autocite{hamaker2005}. Therefore, cross-sectional analyses are rarely equivalent to longitudinal analyses.

Fortunately, technological advancements have allowed researchers to more easily conduct longitudinal research in two ways. First, the use of the experience sampling method \autocite{beal2015} in conjunction with modern information transmission technologies---whether through phone applications or short message services---allows data to sometimes be sampled over time with relative ease. Second, the development of analyses for longitudinal data (along with their integration in commonly used software) that enable person-level data to be modelled such as multilevel models \autocite{raudenbush2002}, growth mixture models \autocite{wang2007}, and dynamic factor analysis \autocite{ram2013} provide researchers with avenues to explore the temporal dynamics of psychological processes. With one recent survey estimating that 43.3\% of mediation studies (26 of 60 studies) used a longitudinal design \autocite{olaughlin2018}, it appears that the prevalence of longitudinal research has increased from the 9.5\% \autocite{roe2008} and 16.3\% \autocite{mitchell2013} values estimated at the beginning of the 21\textsuperscript{st} century. Although the frequency of longitudinal research appears to have increased over the past 20 years, several avenues exist where the quality of longitudinal research can be improved, and in my dissertation, I focus on investigating these avenues.

\hypertarget{understanding-patterns-of-change-that-emerge-over-time}{%
\section{Understanding Patterns of Change That Emerge Over Time}\label{understanding-patterns-of-change-that-emerge-over-time}}

Change can occur in many ways over time. One pattern of change commonly assumed to occur over time is that of linear change. When change follows a linear pattern, the rate of change over time remains constant. Unfortunately, a linear pattern places demanding restrictions on the possible trajectories of change. If change were to follow a linear pattern, then any pauses in change (or plateaus) or changes in direction could not occur: Change would simply grow over time. Unfortunately, effect sizes have been shown to diminish over time \autocites[for meta-analytic examples, see][]{cohen1993,griffeth2000,hom1992,riketta2008,steel1984,steel1990}. Moreover, many variables display cyclic patterns of change over time, with mood \autocite{larsen1990}, daily stress \autocite{bodenmann2010}, and daily drinking behaviour \autocite{huh2015} as some examples. Therefore, change over is unlikely to follow a linear pattern.

A more realistic pattern of change to occur over time is a nonlinear pattern \autocite[for a review, see][]{cudeck2007}. Nonlinear change allows the rate of change to be nonconstant; that is, change may occur more rapidly during certain periods of time, stop altogether, or reverse direction. When looking at patterns of change observed across psychology, several examples of nonlinear change have been found in the declining rate of speech errors throughout child development \autocite{burchinal1991}, rates of forgetting \autocite{murre2015}, development of habits \autocite{fournier2017}, and the formation of opinions \autocite{xia2020}. Given that nonlinear change appears more likely than linear change, my dissertation will assume change over time to be nonlinear.

\hypertarget{challenges-involved-in-conducting-longitudinal-research}{%
\section{Challenges Involved in Conducting Longitudinal Research}\label{challenges-involved-in-conducting-longitudinal-research}}

Conducting longitudinal research presents researchers with several challenges. Many challenges are those from cross-sectional research only amplified \autocite[for a review, see][]{bergman1993}.\footnote{It should be noted that conducting a longitudinal study does alleviate some issues encountered in conducting cross-sectional research. For example, taking measurements over multiple time points likely reduces common method variance \parencites{podsakoff2003}[for an example, see ][]{ostroff2002}.} For example, greater efforts have to be made to to prevent missing data which can increase over time \autocite{newman2008,dillman2014}. Likewise, the adverse effects of well-documented biases such as demand characteristics \autocite{orne1962} and social desirability \autocite{nederhof1985} have to be countered at each time point. Outside of challenges shared with cross-sectional research, conducting longitudinal research also presents new challenges. Analyses of longitudinal data have to consider complications such as how to model error structures \autocite{grimm2010a}, check for measurement non-invariance over time \autocite[the extent to which a construct is measured with the same measurement model over time;][]{mellenbergh1989}, and how to center/process data to appropriately answer research questions \autocite{enders2007,wang2015}.

Although researchers must contend with several issues in conducting longitudinal research, three issues are of particular interest in my dissertation. The first issue concerns how many measurements to use in a longitudinal design. The second issue concerns how to space the measurements. The third issue focuses on how much error is incurred if the time structuredness of the data is overlooked. The sections that follow will review each of these issues.

\hypertarget{number-of-measurements}{%
\subsection{Number of Measurements}\label{number-of-measurements}}

Researchers have to decide on the number of measurements to include in a longitudinal study. Although using more measurements increases the accuracy of results---as noted in the results of several studies \autocites[e.g.,][]{coulombe2016,timmons2015,finch2017,fine2019}---taking additional measurements often comes at a cost that a researcher may be unable account for with a limited budget. One important point to mention is that a researcher designing a longitudinal study must take at least three measurements to obtain a reliable estimate of change and, perhaps more importantly, to allow a nonlinear pattern of change to be modelled \autocite{ployhart2010}. In my dissertation, I hope to determine whether an optimal number of measurements exists when modelling a nonlinear pattern of change.

\hypertarget{spacing-of-measurements}{%
\subsection{Spacing of Measurements}\label{spacing-of-measurements}}

Additionally, a researcher must decide on the spacing of measurements in a longitudinal study. Although discussions of measurement spacing often recommend that researchers use theory and previous studies to determine measurement spacing \autocite{mitchell2001,cole2003,collins2006,dormann2014,dormann2015}, organizational theories seldom delineate periods of time over which a processes unfold, and so the majority of longitudinal research uses intervals of convention and/or convenience to space measurements \autocite{mitchell2001,dormann2014}. Unfortunately, using measurement spacings that do not account for the temporal pattern of change of a psychological process can lead to inaccurate results \autocite[e.g.,][]{chen2014}. As an example, \textcite{cole2009} provide show how correlation magnitudes are affected by the choice of measurement spacing intervals. In my dissertation, I hope to determine whether an optimal measurement spacing schedule exists when modelling a nonlinear pattern of change.

\hypertarget{time-structuredness}{%
\subsection{Time Structuredness}\label{time-structuredness}}

Last, and perhaps most pernicious, latent variable analyses of longitudinal data are likely to incur error from an assumption they make about data collection conditions. Latent variable analyses assume that, across all collection points, participants provide their data at the same time. Unfortunately, such a high level of regularity in the response patterns of participants is unlikely: Participants are more likely to provide their data over some period of time after a data collection window has opened. As an example, consider a study that collects data from participants at the beginning of each month. If participants respond with perfect regularity, then they would all provide their data at the exact same time (e.g., noon on the second day of each month). If the participants respond with imperfect regularity, then they would provide their at different times after the beginning of each month. The regularity of responding observed across participants in a longitudinal study determines the time structuredness of the data and the sections that follow will provide overview of time structuredness.

\hypertarget{time-structured-data}{%
\subsubsection{Time-Structured Data}\label{time-structured-data}}

Many analyses assume that data are \emph{time structured}: Participants provide data at the same time at each collection point. By assuming time-structured data, an analysis can incur error because it will map time intervals of inappropriate lengths onto the time intervals that occurred between participant's responses.\footnote{It should be noted that, although seldom implemented, analyses can be accessorized to handle time-unstructured data by using definition variables \parencites{mehta2000}{mehta2005}.} As an example of the consequences of incorrectly assuming data to be time structured, consider a study that assessed the effects of an intervention on the development of leadership by collecting leadership ratings at four time points each separated by four weeks \autocite{day2011}. The employed analysis assumed time-structured data; that is, each each participant provided ratings on the same day---more specifically, the exact same moment---each time these ratings were collected. Unfortunately, it is unlikely that the data collected from participants were time structured: At any given collection point, some participants may have provided leadership ratings at the beginning of the week, while others may only provide ratings two weeks after the survey opened. Importantly, ratings provided two weeks after the survey opened were likely influenced by changes in leadership that occurred over the two weeks. If an analysis incorrectly assumes time-structured data, then it assumes each participant has the same response rate and, therefore, will incorrectly attribute the amount of time that elapses between most participants' responses. For instance, if a participant only provides a leadership rating two weeks after having received a survey (and six weeks after providing their previous rating), then using an analysis that assumes time-structured data would incorrectly assume that each collection point of this participant is separated by four weeks (the interval used in the experiment) and would, consequently, model the observed change as if it had occurred over four weeks. Therefore, incorrectly assuming data to be time structured leads an analysis to overlook the unique response rates of participants across the collection points and, as a consequence, incur error \autocite{mehta2000,mehta2005,coulombe2016}.

\hypertarget{time-unstructured-data}{%
\subsubsection{Time-Unstructured Data}\label{time-unstructured-data}}

Conversely, some analyses assume that data are \emph{time unstructured}: Participants provide data at different times at each collection point. Given the unlikelihood of one response pattern describing the response rates of all participants in a given study, the data
obtained in a study are unlikely to be time structured. Instead, and because participants are likely to exhibit unique response
patterns in their response rates, data are likely to be time unstructured. One way to conceptualize the distinction between time-structured and time-unstructured data is on a continuum. On one end of the continuum, participants all provide data with identical response patterns, thus giving time-structured data. When participants show unique response patterns, the resulting data are time unstructured, with the extent of time-unstructuredness depending on the length of the response windows. For example, if data are collected at the beginning of each month and participants only have one day to provide data at each time, then, assuming a unique response rate for each participant, the resulting data will have a low amount of time unstructuredness. Alternatively, if data are collected at the beginning of each month and participants have 30 days to provide data each time, then, assuming a unique response rate for each participant, the resulting data will have a high amount of time unstructuredness. Therefore, the continuum of time struturedness has time-structured data on one end and time-unstructured data with long response rates on another end. In my dissertation, I hope to determine how much error is incurred when time-unstructured data are assumed to be time structured.

\hypertarget{summary}{%
\subsection{Summary}\label{summary}}

In summary, researchers must contend with several issues when conducting longitudinal research. In addition to contending with issues encountered in conducting cross-sectional research, researchers must contend with new issues that arise from conducting longitudinal research. Three issues of particular importance in my dissertation are the number of measurements, the spacing of measurements, and incorrectly assuming data to be time structured. These issues will be serve as a basis for a systematic review of the simulation literature.

\hypertarget{using-simulations-to-assess-modelling-accuracy}{%
\section{Using Simulations To Assess Modelling Accuracy}\label{using-simulations-to-assess-modelling-accuracy}}

In the next section, I will present the results of the systematic review of the literature that has investigated the issues of measurement number, measurement spacing, and time structuredness. Before presenting the results of the systematic review, I will provide an overview of the Monte Carlo method used to investigate issues involved in conducting longitudinal research.

To understand how the effects of longitudinal issues on modelling accuracy can be investigated, the inferential method commonly employed in psychological research will first be reviewed with an emphasis on its shortcomings (see Figure \ref{fig:MonteCarlo-comparison}). Consider an example where a researcher wants to understand how sampling error affects the accuracy with which a sample mean (\(\bar{x}\)) estimates a population mean (\(\upmu\)). Using the inferential method, the researcher samples data and then estimates the population mean (\(\upmu\)) by computing the mean of the sampled data (\(\bar{x}_1\)). Because collected samples are almost always contaminated by a variety of methodological and/or statistical deficiencies (such as sampling error, measurement error, assumption violations, etc.), the estimation of the population parameter is likely to be imperfect. Unfortunately, to estimate the effect of sampling error on the accuracy of the population mean estimate (\(\bar{x}_1\)), the researcher would need to know the value of the population mean; without knowing the value of the population mean, it is impossible to know how much error was incurred in estimating the population mean and, as as a result, impossible to know the extent to which sampling error contributed to this error. Therefore, a study following the inferential approach can only provide estimates of population parameters.

The Monte Carlo method has a different goal. Whereas the inferential method focuses on estimating parameters from sample data, the Monte Carlo method is used to understand the factors that influence the accuracy of the inferential approach. Figure \ref{fig:MonteCarlo-comparison} shows that the Monte Carlo method works in the opposite direction of the inferential approach: Instead of collecting a sample, the Monte Carlo method begins by assigning a value to at least one parameter to define a population. Many sample data sets are then generated from the defined population (\(s_1, s_2, ..., s_n\)) and the data from each sample are then modelled by computing a sample mean (\(\bar{x}_1, \bar{x}_2, ..., \bar{x}_n\)). Importantly, manipulations can be for data sampling and/or modelling. In the current example,the population estimates of each statistical model are averaged (\(\bar{\bar{x}}\)) and compared to the pre-determined parameter value (\(\upmu\)). The difference between the average of the estimates and the known population value constitutes bias in parameter estimation (i.e., parameter bias). In the current example, the manipulation causes a systematic underestimation, on average, of the population parameter. By randomly generating data, the Monte Carlo method can determine how a variety of methodological and statistical factors affect the accuracy of a model \autocite[for a review, see][]{robert2010}.
\begin{apaFigure}
[landscape]
[samepage]
[0cm]
{Depiction of Monte Carlo Method}
{MonteCarlo-comparison}
{0.7}
{Figures/Monte_Carlo_comparison}
{Comparison of inferential approach with the Monte Carlo approach. The inferential approach begins with a collected sample and then estimates the population parameter using an appropriate statistical model. The difference between the estimated and population value can be conceptualized as error. Because the population value is generally unknown in the inferential approach, it cannot estimate how much error is introduced by any given methodological or statistical deficiency. To estimate how much error is introduced by any given methodological or statistical deficiency, the Monte Carlo method needs to be used, which constitutes four steps. The Monte Carlo method first defines a population by setting parameter values. Second, many samples are generated from the pre-defined population, with some methodological deficiency built in to each data set (in this case, each sample has a specific amount of missing data). Third, each generated sample is then analyzed and the population estimates of each statistical model are averaged and compared to the pre-determined parameter value. Fourth, the difference between the estimate average and the known population value defines the extent to which the missing data manipulation affected parameter estimation (the difference between the population and average estimated population value is the parameter bias).}
\end{apaFigure}
Monte Carlo simulations have been used to evaluate the effects of a variety of methodological and statistical deficiencies for several decades. Beginning with an early use of the Monte Carlo method, \textcite{boneau1960} used it to evaluate the effects of assumption violations on the fidelity of \emph{t}-value distributions. In more recent years, implementations of the the Monte Carlo method have shown that realistic values of sample size
and measurement accuracy produce considerable variability in estimated correlation values \autocite{stanley2014}. Monte Carlo simulations have also provided valuable insights into more complicated statistical analyses. In investigating more complex statistical analyses, simulations have shown that mediation analyses are biased to produce results of complete mediation because the statistical power to detect direct effects falls well below the statistical power to detect indirect effects \autocite{kenny2014}. Given the ability of the Monte Carlo method to evaluate statistical methods, the experiments in my dissertation used it to evaluate the effects of measurement number, measurement spacing, and time structuredness on modelling accuracy.\footnote{My simulation experiments also investigated the effects of sample size and nature of change on modelling accuracy.}

\hypertarget{systematic-review-of-simulation-literature}{%
\section{Systematic Review of Simulation Literature}\label{systematic-review-of-simulation-literature}}

To understand the extent to which issues involved in conducting longitudinal research had been investigated, I conducted a systematic review of the simulation literature. The sections that follow will first present the method I followed in systematically reviewing the literature and then summarize the findings of the review.

\hypertarget{systematic-review-methodology}{%
\subsection{Systematic Review Methodology}\label{systematic-review-methodology}}

I identified the following keywords through citation searching and independent reading: ``growth curve'', ``time-structured analysis'', ``time structure'', ``temporal design'', ``individual measurement occasions'', ``measurement intervals'', ``methods of timing'', ``longitudinal data analysis'', ``individually-varying time points'', ``measurement timing'', ``latent difference score models'', ``parameter bias'', and ``measurement spacing''. I entered these keywords entered into the PsycINFO database (on July 23, 2021) and any paper that contained any one of these key words and the word ``simulation'' in any field was considered a viable paper (see Figure \ref{fig:prismaDiagram} for a PRISMA diagram illustrating the filtering of the reports). The search returned 165 reports, which I screened by reading the abstracts. Initial screening led to the removal of 60 reports because they did not contain any simulation experiments. Of the remaining 105 papers, I removed 2 more papers because they could not accessed \autocite{stockdale2007,tiberio2008}. Of the remaining 103 identified simulation studies, I deemed a paper as relevant if it investigated the effects of any design and/or analysis factor relating to conducting longitudinal research (i.e., number of measurements, spacing of measurements, and/or time structuredness) and did so using the Monte Carlo simulation method. Of the remaining 103 studies, I removed 89 studies being removed because they did not meet the inclusion criteria, leaving fourteen studies to be included the review, with. I also found an additional 3 studies through citation searching, giving a total of 17 studies.

The findings of my systematic review are summarized in Tables \ref{tab:systematicReviewCount}--\ref{tab:systematicReview}. Tables \ref{tab:systematicReviewCount}--\ref{tab:systematicReview} differ in one way: Table \ref{tab:systematicReviewCount} indicates how many studies investigated each effect, whereas Table \ref{tab:systematicReview} provides the reference of each study and detailed information about
\begin{apaFigure}
[landscape]
{PRISMA Diagram Showing Study Filtering Strategy}
{prismaDiagram}
{0.7}
{Figures/prisma_diagram}
{PRISMA diagram for systematic review of simulation research that investigates longitudinal design and analysis factors.}
\end{apaFigure}
\newpage
\renewcommand\bibname{References}
\phantomsection
\addcontentsline{toc}{chapter}{\bibname}
\printbibliography



\end{document}
